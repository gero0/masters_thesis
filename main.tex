% !TeX root = main.tex
\extrafloats{1000}
\input{src/header.tex}

\usepackage{listings}
\usepackage[ruled]{algorithm2e}
\usepackage{pdfpages}


\lstset{
        %inputencoding=utf8,
        %extendedchars=\true,
        basicstyle=\footnotesize,
        numbers=left,
        literate={ą}{{\k{a}}}1
             {Ą}{{\k{A}}}1
             {ę}{{\k{e}}}1
             {Ę}{{\k{E}}}1
             {ó}{{\'o}}1
             {Ó}{{\'O}}1
             {ś}{{\'s}}1
             {Ś}{{\'S}}1
             {ł}{{\l{}}}1
             {Ł}{{\L{}}}1
             {ż}{{\.z}}1
             {Ż}{{\.Z}}1
             {ź}{{\'z}}1
             {Ź}{{\'Z}}1
             {ć}{{\'c}}1
             {Ć}{{\'C}}1
             {ń}{{\'n}}1
             {Ń}{{\'N}}1
}

\begin{document}
% \bibliographystyle{plabbrv} 
% \bibliographystyle{unsrt} 
\bibliographystyle{siam}

\includepdf{tytul.pdf}
\import{src/}{dedication.tex}
\tableofcontents

\import{chapters/introduction/}{src.tex}
\import{chapters/literature/}{src.tex}
\import{chapters/lon/}{src.tex}
\import{chapters/experiments/}{src.tex}
\import{chapters/solution/}{src.tex}
\import{chapters/ending/}{src.tex}

% \appendixpage % Użyć tylko jeśli występuje więcej niż jeden dodatek!
\begin{appendix}
    \import{appendices/appendix1/}{src.tex}
\end{appendix}

\addcontentsline{toc}{chapter}{\bibname}
\bibliography{bibliography}

\addcontentsline{toc}{chapter}{Spis rysunków}
\listoffigures
\addcontentsline{toc}{chapter}{Spis tablic}
\listoftables

\end{document}

