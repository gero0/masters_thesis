\chapter{Opis implementacji}
Do realizacji badania konieczne było stworzenie odpowiedniego oprogramowania.
Kod źródłowy oprogramowania jest publicznie dostępny na platformie Github pod adresem
\url{https://github.com/gero0/FLA_research}.

Opracowano następujące narzędzia:
\begin{itemize}
    \item Program wykonujący próbkowanie przestrzeni rozwiązań ''tsp\_samplers'',
    \item Program obliczający wartości miar na podstawie spróbkowanej przestrzeni ''stat\_calculator'',
    \item Generatory instancji testowych,
    \item Skrypty pomocnicze.
\end{itemize}

\section{Program próbkujący}
Ponieważ próbkowanie przestrzeni rozwiązań wymaga dużego nakładu obliczeniowego,
implementację algorytmów wykonano w języku Rust.
Rust jest językiem kompilowanym do kodu maszynowego, wspierającym programowanie wielowątkowe.

Program ma postać pliku wykonywalnego uruchamianego z linii poleceń.
Program wymaga podania nazwy pliku wejściowego, nazwę algorytmu oraz wartości parametrów tego algorytmu.
Opcjonalnie jako parametr możliwe jest podanie nazwy folderu, do którego mają zostać zapisane wyniki próbkowania,
jeśli nie zostanie ona podana program zapisze wyniki do folderu o nazwie ''<algorytm>\_latest''.

Format komendy wygląda więc następująco:
\begin{lstlisting}
    tsp_samplers <plik_wejściowy> [plik_wyjściowy] \
    <algorytm> <parametry_algorytmu>
\end{lstlisting}


Akceptowane wartości parametru ''algorytm'' to:
\begin{itemize}
    \item tp - próbkowanie dwufazowe
    \item snowball - próbkowanie snowball
    \item exhaustive - przegląd zupełny
\end{itemize}

Plik wejściowy musi być plikiem txt w następującym formacie:
\begin{itemize}
    \item Linia 1 - nazwa instancji
    \item Linia 2 - liczba miast w instancji
    \item N kolejnych linii - Pełna macierz odległości pomiędzy miastami podanych w postaci liczb całkowitych.
          Kolejne wartości w rzędzie macierzy rozdzielone spacją, wiersze rozdzielone znakiem nowej linii
\end{itemize}


Przykładowy plik wejściowy przedstawiający instancję burma14 ze zbioru TSPLIB został przedstawiony na listingu
\ref{lst:burma}.

\begin{lstlisting}[caption={Instancja burma14 w formacie akceptowanym przez program}, label=lst:burma]
    burma14
    14
    0 153 510 706 966 581 455 70 160 372 157 567 342 398 
    153 0 422 664 997 598 507 197 311 479 310 581 417 376 
    510 422 0 289 744 390 437 491 645 880 618 374 455 211 
    706 664 289 0 491 265 410 664 804 1070 768 259 499 310 
    966 997 744 491 0 400 514 902 990 1261 947 418 635 636 
    581 598 390 265 400 0 168 522 634 910 593 19 284 239 
    455 507 437 410 514 168 0 389 482 757 439 163 124 232 
    70 197 491 664 902 522 389 0 154 406 133 508 273 355 
    160 311 645 804 990 634 482 154 0 276 43 623 358 498 
    372 479 880 1070 1261 910 757 406 276 0 318 898 633 761 
    157 310 618 768 947 593 439 133 43 318 0 582 315 464 
    567 581 374 259 418 19 163 508 623 898 582 0 275 221 
    342 417 455 499 635 284 124 273 358 633 315 275 0 247 
    398 376 211 310 636 239 232 355 498 761 464 221 247 0     
\end{lstlisting}

Wyniki program zapisuje okresowo do docelowego folderu w plikach ''samples\_<i>.json''.
Pliki zawierają następujące informacje:

\begin{itemize}
    \item nodes - lista wierzchołków grafu. Każdy wierzchołek opisywany jest 3-elementową listą:
          [unikalny identyfikator, rozwiązanie, wartość funkcji celu]. Identyfikator oraz wartość funkcji celu
          są liczbami całkowitymi, rozwiązanie ma zaś postać listy identyfikatorów miast w kolejności, w jakiej przebiega przez nie trasa,
    \item edges - lista krawędzi skierowanych. Każda krawędź jest opisane 3-elementową listą - [id. wierzchołka początkowego, id. wierzchołka docelowego, waga],
    \item opt\_count - liczba wywołań funkcji optymalizującej (2opt),
    \item oracle\_count - liczba wykonanych obliczeń długości ścieżki,
    \item time\_ms - czas działania programu podany w ms. Liczony jest jedynie czas spędzony na próbkowaniu
          (bez liczenia czasu spędzonego na zapisywanie wyników),
    \item comment - komentarz, zawiera parametry, z którymi uruchomiono algorytm,
    \item missed - tylko dla próbkowania dwufazowego, licznik prób dodania krawędzi, które zakończyły się
          niepowodzeniem z powodu braku wierzchołka należącego do krawędzi w zbiorze spróbkowanych wierzchołków.
\end{itemize}

Przykładowy plik wyjściowy programu został przedstawiony na listingu \ref{lst:sampleoutput}.

\begin{lstlisting}[caption={Przykład pliku wyjściowego po krótkim próbkowaniu małej instancji}, label=lst:sampleoutput]
{
"opt_count":101802,
"oracle_count":5114163,
"time_ms":199,
"comment":"n_max:100 n_att:100 e_att:100 iters:10 file:cliques_6",
"missed":0,
"nodes": [
[1,[0, 1, 4, 5, 3, 2],1622],
[0,[0, 2, 3, 5, 4, 1],1622]],
"edges": [
[0,0,335],
[1,0,93],
[0,1,665],
[1,1,907]]
}
\end{lstlisting}

\section{Program obliczający wartości miar}

Program obliczający wartości miar został napisany w języku Python.
Większość miar grafu obliczone zostało przy pomocy biblioteki igraph.
Aby zwiększyć wydajność, do obliczeń metryk \textbf{avg\_loop\_weight} i \textbf{num\_subsinks}
wykorzystano moduł natywny napisany w języku Rust.

Program jest uruchamiany przy użyciu interpretera języka Python, jako argument przyjmuje
ścieżkę do folderu zawierającego pliki z próbkami. Opcjonalnymi parametrami są: nazwa pliku wyjściowego
limit plików wejściowych (-l), oraz nazwy miar do policzenia (-s, --stats). Jeśli nazwa pliku wyjściowego nie zostanie podana, wyniki zapisywane
są do pliku ''results.csv''. Jeśli limit nie zostanie podany, obliczenia są wykonywane dla wszystkich plików w folderze.
Jeśli nie podane zostaną żadne nazwy miar do policzenia, domyślnym działaniem jest obliczenie ich wszystkich.

Format komendy wygląda następująco:
\begin{lstlisting}
    python stat_calculator/main.py <plik_wejściowy> \
    [plik_wyjściowy] [ -l <N>] [ --stats ...]
\end{lstlisting}

Dane wyjściowe zapisywane są w postaci pliku csv zgodnego ze strukturami DataFrame biblioteki pandas.
Dodatkowo generowany jest plik ''results\_corr.csv'' zawierający tablicę z wartościami współczynników korelacji
między wartościami miar.

TODO: przykładowy mały plik wyjściowy

\section{Generatory instancji testowych}

Generatory instancji testowych napisano w języku Python.
Są to trzy proste skrypty, każdy generujący inny rodzaj instancji problemu.

\section{Skrypty pomocnicze}
Skrypty pomocnicze wykonano w języku Python.
Należą do nich:
\begin{itemize}
    \item plotting.py - Skrypt rysujący wykresy poszczególnych miar w zależności od liczby wierzchołków. Skrypt przyjmuje na wejście pliki z wartościami miar wygenerowane przez program stat\_calculator
    \item convert\_tsplib.py - Skrypt konwertujący pliki w formacie tsplib na uproszczony format przyjmowany przez program próbkujący. Korzysta z biblioteki tsplib95.
    \item visualize.py - Prosty skrypt tworzący wizualizację grafu LON na podstawie pliku wejściowego json z próbkami.
    \item run\_exhaustive.py - Skrypt uruchamiający po kolei przegląd zupełny dla podanej listy instancji
    \item run\_tp.py - Skrypt uruchamiający po kolei próbkowanie dwufazowe dla podanej listy instancji
    \item run\_snowball.py - Skrypt uruchamiający po kolei próbkowanie \textit{snowball} dla podanej listy instancji
    \item run\_stat\_calc.py - Skrypt uruchamiający program obliczający wartości miar dla podanej listy folderów w próbkami
    \item run\_plotting.py - Skrypt uruchamiający po kolei skrypt plotting.py dla podanej listy plików z obliczonymi wartościami miar
\end{itemize}

\section{Wykorzystane biblioteki}
Najważniejsze zewnętrzne biblioteki wykorzystane w projekcie:
\vspace{1em}

Rust:
\begin{itemize}
    \item rand\_chacha - szybki i deterministyczny generator liczb losowych,
    \item clap - parser argumentów konsoli,
    \item rustc-hash - szybka hashmapa,
    \item permutations\_iter - generator permutacji metodą Steinhausa-Johnsona-Trottera,
    \item PyO3 - narzędzie do tworzenia natywnych modułów Pythona w języku Rust,
    \item tsptools - generator losowego rozwiązania problemu TSP oraz funkcja obliczająca długość ścieżki.
\end{itemize}

Python:
\begin{itemize}
    \item igraph - obliczanie wartości miar grafu LON, generowanie wizualizacji,
    \item matplotlib - rysowanie wykresów,
    \item numpy, pandas - przetwarzanie danych,
    \item Pillow - generowanie obrazów,
    \item tsplib95 - odczytywanie plików w formacie tsplib.
\end{itemize}

