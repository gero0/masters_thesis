\chapter{Podsumowanie}

W pracy wykonano badanie stabilności wartości miar dla problemu komiwojażera.
Wykonano implementację generatorów instancji testowych, algorytmów próbkowania przestrzeni rozwiązań,
oraz programu obliczającego wartości miar.
Wykonano badania dla 35 instancji problemu pochodzących z generatorów oraz ze zbioru tsplib.
Wykorzystano algorytm dwufazowy oraz \textit{snowball}.

Dla małych instancji o rozmiarze do 11 miast, wykonano przegląd zupełny i porównano wartości miar z wartościami uzyskanymi
z próbkowania. Wykazano, że dla większości małych instancji możliwe jest uzyskanie dokładnych wartości
większości miar w czasie znacznie krótszym od przeglądu zupełnego.
Wyniki sugerują również, że dla instancji tego rozmiaru bardziej efektywne obliczeniowo jest wykorzystanie
algorytmu \textit{snowball}.

Większe instancje poddano badaniu stabilności, polegającemu na zapisywaniu kolejnych wartości miar
w ramach zwiększania się liczby wierzchołków w spróbkowanej przestrzeni.
Stwierdzono, że wartość większości miar jest niemożliwa do przewidzenia z dużą dokładnością.
Zauważono duży wpływ długości próbkowania oraz wybranego algorytmu na wartości miar przestrzeni rozwiązań.
Zidentyfikowano słabe punkty obu algorytmów próbkowania.
Algorytm dwufazowy nie radzi sobie z odnajdywaniem krawędzi w instancjach z dużą liczbą optimów lokalnych przy niedostatecznie długim próbkowaniu.
Algorytm \textit{snowball} może mieć trudności ze znalezieniem wszystkich optimów lokalnych w przestrzeni.

Zauważono różnice między różnymi typami instancji.
Instancje z miastami ułożonymi na siatce cechują się dużym stosunkiem liczby optimów lokalnych do liczby miast.
Instancje z miastami rozmieszczonymi w klikach mają zwykle mniej optimów lokalnych od podobnych rozmiarem instancji innego typu.

Ze względu na ograniczony czas i moc obliczeniową, eksperymenty zostały przeprowadzone dla niewielkiej liczby instancji.
Rozwój badań w tym kierunku powinien obejmować przeprowadzenie analizy dla dużej liczby instancji tego samego typu i rozmiaru,
np. kilkanaście instancji o rozmiarze 100 i rozkładzie równomiernym.
Wtedy będzie można stwierdzić z większą pewnością, na które cechy przestrzeni wpływa dany sposób rozmieszczenia miast.
Należy również przeprowadzić dłuższe próbkowanie, szczególnie dla instancji z dużą ilością lokalnych optimów.

Aby dłuższe próbkowanie było możliwe, należy zadbać o wydajną implementację algorytmów.
Oprócz wymyślania nowych algorytmów, można również dokonywać prób zrównoleglenia istniejących. 
Na potrzeby tej pracy zaimplementowano algorytm dwufazowy w sposób umożliwiający wykorzystanie wielu procesorów.
Kolejnym krokiem może być uruchomienie algorytmu w systemie rozproszonym i/lub na kartach GPU.