\chapter{Wstęp}

Analiza przestrzeni rozwiązań jest jednym z rzadziej rozważanych aspektów badań nad problemami optymalizacyjnymi.
Dane uzyskane z analizy przestrzeni rozwiązań, oraz otrzymane z nich wartości miar, mogą nosić w sobie
informacje o właściwościach danego problemu, lub jego konkretnej instancji.
Z powodu dużej złożoności obliczeniowej problemów optymalizacyjnych, zwykle nie jest możliwy przegląd całej przestrzeni.
Do badań wykorzystuje się więc próbkowanie, które można przeprowadzić na wiele różnych sposobów.
Wybór algorytmu, oraz długość procesu próbkowania może mieć istotny wpływ na wartości miar obliczonych na podstawie spróbkowanej przestrzeni.

O ile większość najnowszych publikacji z dziedziny zajmuje się badaniami nad możliwościami praktycznego zastosowania analizy przestrzeni rozwiązań w różnych zadaniach,
niniejsza praca skupia się głównie na samym aspekcie próbkowania przestrzeni.
Celem jest zbadanie wpływu metody próbkowania na wartości miar dla jednego z najbardziej znanych 
problemów optymalizacji dyskretnej - problemu komiwojażera.
Wykonane zostanie długotrwałe próbkowanie przestrzeni rozwiązań kilkudziesięciu instancji problemu, przy użyciu dwóch różnych algorytmów.
Uzyskane dane posłużą do obliczenia wartości miar i przedstawieniu ich w relacji do wielkości spróbkowanej przestrzeni.
Umożliwi to ocenę wpływu metody na wartości poszczególnych miar oraz ich stabilność.

Praca składa się z pięciu rozdziałów.
W rozdziale pierwszym znajduje się krótkie wprowadzenie do tematyki pracy.
Rozdział drugi zawiera przegląd najnowszych publikacji z dziedziny, oraz sekcje poświęcone objaśnieniu niektórych terminów związanych z tematyką
analizy przestrzeni rozwiązań. Znajduje się w nim również opis miar analizowanych w reszcie pracy.
Rozdział trzeci zawiera opis eksperymentu. Przedstawione są w nim testowe instancje problemu i wykorzystane algorytmy próbkowania.
W dalszej części rozdziału opisany jest przebieg eksperymentu oraz uzyskane wyniki.
Rozdział czwarty jest technicznym opisem implementacji i zawiera informacje o programach i skryptach napisanych na potrzeby pracy.
Przedstawiono w nim krótkie instrukcje uruchomienia najważniejszych programów. Opisane zostały również wykorzystywane przez nie formaty plików.
W rozdziale znajduje się również spis najważniejszych zewnętrznych bibliotek i narzędzi wykorzystanych w tworzeniu projektu.
W rozdziale piątym znajduje się krótkie podsumowanie całości pracy oraz wnioski wyciągnięte na podstawie uzyskanych wyników.
