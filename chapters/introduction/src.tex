\chapter{Wprowadzenie}

Analiza przestrzeni rozwiązań, znanej również jako krajobraz adaptacyjny, jest jednym z rzadziej rozważanych aspektów badań nad problemami optymalizacyjnymi.
Przestrzeń ta pozwala na reprezentację problemu jako ,,krajobraz'', w którym każde rozwiązanie jest punktem na płaszczyźnie, a wysokość punktu wartością funkcji celu
danego rozwiązania.
Niektóre z cech tej przestrzeni, takie jak liczba optimów lokalnych i ich rozkład, obecność rozległych obszarów płaskich,
licznych wzniesień lub wgłębień, mogą dostarczać informacji o właściwościach danego problemu, lub jego konkretnej instancji \cite{DBLP:journals/corr/OchoaVDT14}.
Dokładna reprezentacja przestrzeni rozwiązań, obejmująca wszystkie rozwiązania,
wymagałaby przechowywania olbrzymiej ilości danych.
Jednym z bardziej kompaktowych formatów jej reprezentacji jest sieć optimów lokalnych (ang. Local Optima Network, LON) --- graf, którego wierzchołki reprezentują
optima lokalne przestrzeni, a krawędzie relacje pomiędzy nimi.
Wartości miar opisujące graf LON, takie jak wagi krawędzi, czy długości ścieżek między wierzchołkami, można wykorzystać
m.in do predykcji jakości wyników uzyskiwanych przez techniki heurystyczne, dostrajania ich parametrów, lub do projektowania całkiem nowych algorytmów.
Z powodu dużego rozmiaru przestrzeni rozwiązań zbudowanie grafu stanowiącego jej dokładną reprezentację jest trudne, czasochłonne i zwykle niemożliwe
do wykonania w akceptowalnym czasie.
W badaniach zamiast przeglądu zupełnego wykorzystuje się zatem próbkowanie, zakładając przy tym, że wartości miar uzyskane z obszaru nim objętego
są w pewnym stopniu reprezentatywne dla całej przestrzeni. Próbkowanie to można wykonać na wiele różnych sposobów,
a wybór algorytmu, oraz długość procesu może mieć istotny wpływ na wartości miar obliczonych na podstawie spróbkowanej przestrzeni.

Większość najnowszych publikacji z dziedziny zajmuje się badaniami nad możliwościami praktycznego zastosowania analizy przestrzeni rozwiązań w różnych zadaniach.
Niniejsza praca skupia się zaś głównie na samym aspekcie próbkowania.
Uzyskane z eksperymentów dane pozwolą na sprawdzenie, w jaki sposób wartości miar zmieniają się wraz z postępem procesu próbkowania.
Pozwoli to ocenić, które z nich zmieniają się w sposób na tyle przewidywalny, by możliwe było szacowanie ich wartości dla pełnej przestrzeni na podstawie danych z jej spróbkowanego fragmentu.
Wyniki ukażą również różnice pomiędzy sieciami LON budowanymi przez różne techniki próbkowania i ujawnią mocne oraz słabe strony poszczególnych algorytmów.

\section{Cel pracy}
W pracy zostaną przeprowadzone badania nad przestrzenią rozwiązań jednego z najbardziej znanych problemów optymalizacji dyskretnej --- problemu komiwojażera.

Do poszczególnych celów pracy należą:
\begin{itemize}
    \item Sprawdzenie dokładności, z jaką algorytmy próbkowania odwzorowują rzeczywistą przestrzeń rozwiązań,
    \item Zbadanie wpływu metody próbkowania, oraz przeznaczonego na nie nakładu obliczeniowego na wartości miar uzyskanej spróbkowanej przestrzeni,
    \item Zbadanie korelacji wzajemnej miar przestrzeni rozwiązań oraz wpływu metody próbkowania na wartości współczynników korelacji.
\end{itemize}

Cel zostanie zrealizowany poprzez wykonanie eksperymentów na kilkudziesięciu różnorodnych --- pod względem liczby miast i ich ułożenia --- instancjach problemu.

\section{Zakres Pracy}

Praca składa się z pięciu rozdziałów.
W rozdziale pierwszym znajduje się krótkie wprowadzenie do tematyki pracy.
Rozdział drugi zawiera przegląd najnowszych publikacji z dziedziny, opisane są w nim zastosowania analizy przestrzeni rozwiązań, oraz algorytmy
próbkowania znalezione w literaturze.
Rozdział trzeci poświęcony jest objaśnieniu niektórych terminów związanych z tematyką
analizy przestrzeni rozwiązań. Znajduje się w nim również opis miar analizowanych w reszcie pracy.
Rozdział czwarty zawiera opis eksperymentu. Przedstawione są w nim testowe instancje problemu i wykorzystane algorytmy próbkowania.
W dalszej części rozdziału opisany jest przebieg eksperymentu oraz uzyskane wyniki.
Dla małych instancji problemu, dla których możliwe jest wykonanie przeglądu zupełnego zostanie wykonane porównanie wartości miar
uzyskanych z próbkowania do tych uzyskanych z przeglądu zupełnego. Pozwoli to na ocenę dokładności próbkowania.
Dla kilkudziesięciu większych instancji wykonane zostanie długotrwałe próbkowanie przestrzeni rozwiązań, przy użyciu dwóch różnych algorytmów.
Uzyskane dane posłużą do obliczenia wartości miar i przedstawieniu ich w relacji do wielkości spróbkowanej przestrzeni.
Umożliwi to ocenę wpływu algorytmu na wartości poszczególnych miar oraz ich stabilność.
Dane te posłużą również do obliczenia współczynników wzajemnej korelacji Pearsona poszczególnych miar.
Współczynniki zostaną obliczone osobno dla dwóch algorytmów, co pozwoli na ocenę wpływu metody próbkowania na ich wartość.
Rozdział piąty jest technicznym opisem implementacji i zawiera informacje o programach i skryptach napisanych na potrzeby pracy.
Przedstawiono w nim krótkie instrukcje uruchomienia najważniejszych programów. Opisane zostały również wykorzystywane przez nie formaty plików.
W rozdziale znajduje się także spis najważniejszych zewnętrznych bibliotek i narzędzi wykorzystanych w tworzeniu projektu.
W rozdziale szóstym znajduje się krótkie podsumowanie całości pracy oraz wnioski wyciągnięte na podstawie uzyskanych wyników.
