\chapter{Badania eksperymentalne}

\section{Zaimplementowane algorytmy}
\subsection{Próbkowanie dwufazowe} \label{section:tp}
Próbkowanie dwufazowe swoją nazwę zawdzięcza procesowi próbkowania składającemu się z dwóch oddzielnych faz
- próbkowania wierzchołków oraz próbkowania krawędzi - wykonywanych jedna po drugiej.
Istotną zaletą tego podejścia jest jego stosunkowo prosta implementacja.

Zaimplementowany algorytm pochodzi z pracy\cite{DBLP:conf/depcos/BozejkoGNAB18}.
Został on przygotowany specjalnie do próbkowania przestrzeni rozwiązań problemu komiwojażera.
Próbkowanie wierzchołków odbywa się poprzez generowanie losowych rozwiązań, a następnie ich optymalizacji algorytmem 2-opt.
Próbkowanie krawędzi polega na wielokrotnym poddaniu każdego ze znalezionych wcześniej lokalnych optimów $n_i$ operacji perturbacji typu 2-exchange,
a następnie poddaniu powstałego rozwiązania optymalizacji algorytmem 2-opt typu \textit{first-improvement} uzyskując w ten sposób lokalne optimum $n_j$.
Następnie dodawana jest krawędź między $n_i$ a $n_j$, lub - jeśli już taka istnieje - jej waga jest zwiększana o 1.

Algorytm przyjmuje trzy parametry: pożądaną liczbę wierzchołków do wygenerowania ($n_{max}$), maksymalną liczbę prób generowania wierzchołka ($n_{att}$)
oraz maksymalną liczbę prób generowania krawędzi($e_{att}$).
Implementacja zastosowana w tej pracy dodatkowo powtarza cały proces kilkukrotnie, za każdym razem zapisując zebrane próbki do pliku.

Algorytm w postaci pseudokodu został przedstawiony na listingu \ref{alg:tp}.

\subsection{Snowball} \label{section:snowball}
Próbkowanie typu Snowball wywodzi się z techniki używanej w badaniach z dziedziny socjologii, w której ludzie należący do próby z populacji
rekrutują kolejnych uczestników badania spośród swoich znajomych.
W kontekście badania przestrzeni rozwiązań technika ta została zaprezentowana w pracy\cite{DBLP:conf/ppsn/VerelDOT18}, gdzie została wykorzystana
do próbkowanie przestrzeni problemu kwadratowego przypisania (QAP).

Próbkowanie składa się z etapów procedury \textit{snowball} próbkującej "wgłąb" i losowego spaceru(ang. \textit{random walk}).
Próbkowanie \textit{snowball} polega na wybraniu rozwiązania startowego i przeszukaniu jego najbliższego sąsiedztwa.
Następnie operacja ta jest powtarzana dla każdego rozwiązania w tym sąsiedztwie. Proces powtarza się aż do osiągnięcia z góry ustalonej głebokości
przeszukiwania. Następnie rozpoczyna się procedura losowego spaceru - wybierane jest kolejne rozwiązanie startowe
ze zbioru sąsiadów poprzedniego rozwiązywania startowego (lub rozwiązanie losowe, jeśli to sąsiedztwo jest puste) i proces \textit{snowball}
rozpoczyna się od nowa. Procedura jest powtarzana aż osiągnięty zostanie z góry ustalony limit długości spaceru.

Zaimplementowany algorytm jest próbą adaptacji tej techniki do zadania przeszukiwania przestrzeni
problemu komiwojażera. Do najważniejszych modyfikacji należy zastąpienie funkcji optymalizacji lokalnej \textit{hillclimb}
optymalizacją 2opt, implementacja odpowiedniej funkcji celu oraz operacji mutacji typu 2-exchange.

Algorytm w postaci pseudokodu został przedstawiony na listingu \ref{alg:snowball}.

\vspace{1em}

\begin{algorithm}[H]
    \caption{Próbkowanie dwufazowe - pseudokod}\label{alg:tp}

    \KwData{\\
        \Indp \Indp
        $n_{max}$ - żądana liczba wierzchołków\\
        $n_{att}$ - liczba prób generowania wierzchołków\\
        $e_{att}$ - liczba prób generowania krawędzi\\
        $n_{runs}$ - liczba powtórzeń\\
        $D$ - stała D krawędzi
    }

    \vspace{1em}

    $N \gets \{\}$\;
    $E \gets \{\}$\;
    \For{$i\gets1$ \KwTo $n_{runs}$}{
        $probkujWierzcholki(N, n_{max}, n_{att})$\;
        $probkujKrawedzie(N, E, e_{att})$\;
        $zapiszDoPliku(N, E)$\;
    }
    \vspace{1em}

    \SetKwFunction{FV}{probkujWierzcholki}
    \SetKwProg{Fn}{function}{:}{}

    \Fn{\FV{$N, n_{max}, n_{att}$}}{
        \For{$i\gets1$ \KwTo $n_{max}$}{
            \For{$i\gets1$ \KwTo $n_{att}$}{
                $s \gets losoweRozwiazanie()$\;
                $s \gets 2opt(s)$\;
                $N \gets N \cup \{s\}$\;
            }
        }
    }
    \textbf{end}

    \vspace{1em}

    \SetKwFunction{FE}{probkujKrawedzie}

    \Fn{\FE{$N, E, e_{att}$}}{
        \ForEach{$n \in{N}$}{
            \For{$i\gets1$ \KwTo $e_{att}$}{
                $s \gets 2exchangeMutacja(n, D)$\;
                $s \gets 2optFirstImprovement(s)$\;
                \If{$s \in{N}$}{
                    $E \gets E \cup \{(n, s)\}$\;
                    $w_{ns} \gets w_{ns} + 1$\;
                }
            }
        }
    }
    \textbf{end}

    \vspace{1em}

\end{algorithm}

\begin{algorithm}[p]
    \caption{Próbkowanie snowball - pseudokod}\label{alg:snowball}

    \KwData{\\
        \Indp \Indp
        $w_{len}$ - długość losowego spaceru\\
        $m$ - liczba prób przeszukania sąsiedztwa\\
        $depth$ - głębokość przeszukiwania\\
        $D$ - stała D krawędzi\\
        $s_{tresh}$ - interwał zapisu
    }

    \vspace{1em}
    $s_1 \gets losoweRozwiazanie()$\;
    $n_1 \gets 2opt(s_1)$\;

    $N \gets \{ n_1 \}$\;
    $E \gets \{\}$\;
    \For{$j\gets1$ \KwTo $n_{runs}$}{
        \For{$i\gets1$ \KwTo $w_{len}$}{
            $snowball(n_i, m, depth)$\;
            $n_{i+1} \gets losowySpacer(n_i)$\;
        }
    }
    $zapiszDoPliku(N, E)$\;
    \vspace{1em}

    \SetKwFunction{FS}{snowball}
    \SetKwProg{Fn}{function}{:}{}

    \Fn{\FS{$n, m, depth$}}{
        \If{$d > 0$}{
            \For{$i\gets1$ \KwTo $m$}{
                $s \gets 2opt(2exchangeMutacja(n, D))$\;
                $N \gets N \cup \{ s \}$\;
                \If{$|N| \hspace{0.5em} mod \hspace{0.5em} s_{tresh} = 0$}{
                    $zapiszDoPliku(N, E)$\;
                }
                \uIf{$(n, s) \in{E}$}{
                    $w_{ns} \gets w_{ns} + 1$\;
                }
                \Else{
                    $E \gets E \cup \{ (n, s) \}$\;
                    $w_{ns} \gets 1$\;
                    $snowball(s, m, d-1)$\;
                }
            }
        }
    }
    \textbf{end}

    \vspace{1em}

    \SetKwFunction{FW}{losowySpacer}

    \Fn{\FW{$n_i$}}{
        $neighbours \gets \{ s: (n_i, s) \in E \land s \notin \{ n_0...n_i \} \}$\;
        \uIf{$neighbours \neq \emptyset $}{
            $n_{i+1} \gets losowyElementZeZbioru(neighbours)$\;
        }
        \Else{
            $s \gets losoweRozwiazanie()$\;
            $n_{i+1} \gets 2opt(s)$\;
            $N \gets N \cup \{ n_{i+1} \}$\;
            \If{$|N| \hspace{0.5em} mod \hspace{0.5em} s_{tresh} = 0$}{
                $zapiszDoPliku(N, E)$\;
            }
        }
        \Return $n_{i+1}$
    }
    \textbf{end}

    \vspace{1em}

\end{algorithm}

\newpage

\subsection{Przegląd zupełny}
Ze względu na złożoność problemu komiwojażera przegląd zupełny można zastosować tylko do bardzo małych instancji problemu.
Przegląd polega na wygenerowaniu wszystkich możliwych rozwiązań danej instancji, wykonaniu na nich optymalizacji 2-opt w celu znalezienia
optimów lokalnych a następnie znalezieniu krawędzi oraz obliczeniu ich wag. Dla każdego z rozwiązań generowane są wszystkie
permutacje, które mogą powstać poprzez D-krotne wykonanie na rozwiązaniu operacji 2-exchange. Jeśli wśród tych permutacji znajduje się jedno ze znalezionych wcześniej
lokalnych optimów, oznacza to, że spełniony jest warunek \ref{eq:escape_edge_cond} i dodawana jest nowa krawędź lub zwiększona zostaje waga istniejącej.

\begin{algorithm}[]
    \caption{Przegląd zupełny}\label{alg:exhaustive}
    $S \gets \{\}$\;
    $P \gets wygenerujWszystkiePermutacje()$\;
    \ForEach{$p \in P$}{
        $lo \gets 2opt(p)$\;
        $S \gets S \cup \{(p, lo)\}$\;
    }

    \SetKwProg{Fn}{function}{:}{}

    \ForEach{$(p, lo) \in S$}{
        \ForEach{$n \in N$}{
            \If{$wZasiegu2Exchange(p, n, D)$}{
                \uIf{$(n, lo) \in E$}{
                    $w_{n, lo} \gets w_{n, lo} + 1$\;
                }\Else{
                    $E \gets E \cup \{(n, lo)\}$\;
                    $w_{n, lo} = 1$\;
                }
            }
        }
    }

    \SetKwFunction{FI}{wZasiegu2Exchange}

    \vspace{1em}

    \Fn{\FI{$p, n, D$}}{
        $permutacje \gets \{p\}$\;
        \For{$i \in 1..D$}{
            $nowe\_perm \gets \{\}$\;
            \ForEach{$perm \in permutacje$}{
                $pochodne\_perm \gets 2exchangeWszystkiePermutacje(permutacje)$\;
                \ForEach{$poch\in pochodne\_perm$}{
                    \If{poch = n}{
                        \Return{true}\;
                    }
                    $nowe\_perm \gets nowe\_perm \cup \{poch\}$\;
                }
            }
            $permutacje \gets nowe\_perm$\;
        }
        \Return{false}\;
    }
    \textbf{end}

\end{algorithm}

\section{Instancje testowe}
Do badań wykorzystano instancje testowe wygenerowane losowo oraz wybrane instancje ze zbioru TSPLIB.
Zaimplementowano trzy generatory tworzące różne typy instancji testowych: z miastami rozłożonymi równomiernie,
z miastami rozłożonymi w klastrach oraz z miastami ułożonymi na siatce.
Wygenerowano instancje testowe każdego z trzech typów instancji losowych o rozmiarach 7, 8, 9, 10, 11 oraz 20, 50, 80 i 100.
Uzyskano w ten sposób 27 instancji problemu.
Ze zbioru TSPLIB wybrano instancje o podobnych rozmiarach: \textbf{burma14}, \textbf{ulysses22}, \textbf{att48}, \textbf{berlin52}, \textbf{pr76}, \textbf{eil76}, \textbf{rat99},
\textbf{bier127}.
W sumie badanie przeprowadzono na 35 instancjach problemu.

\subsection*{Miasta rozmieszczone równomiernie}
Generator losowo rozmieszcza miasta na wirtualnej planszy o ustalonym rozmiarze.\\
Współrzędne miast generowane są losowo z rozkładu równomiernego.
W dalszej części dokumentu instancje wygenerowane tym generatorem będą nazywane \textbf{uniform\_<liczba miast>}.

Przykład wygenerowanej instancji został przedstawiony na rysunku \ref{fig:uniform_example}.

\subsection*{Miasta rozmieszczone w klastrach}
Miasta umieszczane są blisko siebie w kilku grupach oddzielonych większymi odległościami.
W dalszej części dokumentu instancje wygenerowane tym generatorem będą nazywane \textbf{cliques\_<liczba miast>}.
Przykład wygenerowanej instancji został przedstawiony na rysunku \ref{fig:clique_example}.


\subsection*{Miasta rozmieszczone na siatce}
Miasta umieszczane są na siatce, w stałej odległości od swoich sąsiadów.
W dalszej części dokumentu instancje wygenerowane tym generatorem będą nazywane \textbf{grid\_<liczba miast>}.
Przykład wygenerowanej instancji został przedstawiony na rysunku \ref{fig:grid_example}.

\begin{figure}[h!]
    \centering
    \includegraphics[width=0.46\textwidth]{chapters/experiments/img/uniform_example.png}
    \caption{Wizualizacja przykładowej wygenerowanej instancji z miastami rozmieszczonymi równomiernie dla 100 miast}
    \label{fig:uniform_example}
\end{figure}

\newpage

\begin{figure}[h!]
    \centering
    \includegraphics[width=0.60\textwidth]{chapters/experiments/img/clique_example.png}
    \caption{Wizualizacja przykładowej wygenerowanej instancji z miastami rozmieszczonymi w klastrach dla 100 miast}
    \label{fig:clique_example}
\end{figure}

\begin{figure}[h!]
    \centering
    \includegraphics[width=0.60\textwidth]{chapters/experiments/img/grid_example.png}
    \caption{Wizualizacja przykładowej wygenerowanej instancji z miastami rozmieszczonymi na siatce dla 100 miast}
    \label{fig:grid_example}
\end{figure}

\section{Opis badań}

Badania podzielono na trzy etapy: Porównanie wartości metryk z przeglądem zupełnym dla małych instancji,
badanie stabilności średnich instancji, oraz badanie stabilności dużych instancji.

W tym rozdziale przedstawione zostaną tabele i wykresy wygenerowane z wyników próbkowania.
Nazwy metryk są identyczne z nazwami objaśnionymi w rozdziale \ref{sec:metrics}.
Ponadto występują pojęcia:

\begin{itemize}
    \item node\_count - liczba wierzchołków,
    \item edge\_count - liczba krawędzi,
    \item opt\_count - liczba wywołań funkcji 2opt,
    \item oracle\_count - liczba wykonanych obliczeń długości ścieżki (metodą obliczenia przyrostu długości stosowanej w 2opt).
    \item twophase - algorytm dwufazowy
\end{itemize}

Wszystkie badania wykonano wykorzystując model krawędzi typu \textit{Escape Edges} z parametrem $D$ o wartości 2.

\subsection{Porównanie wartości metryk dla małych instancji}
Dla małych instancji wygenerowanych losowo (uniform, grid, cliques) o rozmiarach 7 do 11
możliwe jest wykonanie przeglądu zupełnego.
Na tych instancjach wykonano próbkowanie algorytmami dwufazowym oraz snowball, z różnym zestawem parametrów.

Dla algorytmu dwufazowego wykonano próbkowanie dla różnych wartości parametru $n_{max}$ (liczby żądanych wierzchołków): 10, 100, 1000 oraz 10000.
Parametry $n_{att}$, $e_{att}$ ustawiono na równe $n_{max}$, a parametr $n_{runs}$ na 1.

Dla algorytmu \textit{snowball} wykonano próbkowanie dla różnych wartości parametru $w_{len}$ (długości losowego spaceru): 1, 10, 100 oraz 1000.
Pozostałe parametry ustawiono następująco: $m$ = 100, $depth$ = 3.

Uzyskane wartości porównano z wartościami uzyskanymi z przeglądu zupełnego.
Wyniki przedstawiono w tabelach od \ref{tab:comp_uniform_7_snowball} do \ref{tab:comp_grid_11_twophase}.
W tabeli przedstawiono wartości metryk uzyskane w przeglądzie zupełnym oraz wartości uzyskane z próbkowania.
Dodatkowo w nawiasie podana została wartość błędu obliczona z wzoru \ref{eq:relError}.
\begin{equation}
    \label{eq:relError}
    E = \frac{|R - S|}{R} \cdot 100\%
\end{equation}
Gdzie:
\begin{itemize}
    \item E - wartość błędu
    \item R - wartość rzeczywista (z przegl. zupełnego)
    \item S - wartość otrzymana z próbkowania.
\end{itemize}

Wartości, dla których błąd nie wyniósł 0 zostały dodatkowo oznaczone w tabeli grubą czcionką.

\begin{table}
\centering
\caption{Instancja uniform\_7 - porównanie wartości miar uzyskanych z próbkowania snowball z wartościami z przeglądu zupełnego.}
\label{tab:comp_uniform_7_snowball}
\footnotesize
\resizebox{\textwidth}{!}{
\begin{tabular}{|l|l|l|l|l|l|}
\hline
\textbf{metryka} & \textbf{p. zupełny} & \textbf{snowball\_1} & \textbf{snowball\_10} & \textbf{snowball\_100} & \textbf{snowball\_1000} \\ \hline
opt\_count & 720.00 & 501.00 & 1608.00 & 10698.00 & 101798.00 \\ \hline
oracle\_count & 42362.00 & 30912.00 & 97125.00 & 650391.00 & 6179229.00 \\ \hline
node\_count & 4.00 & 4.00 (0.00\%) & 4.00 (0.00\%) & 4.00 (0.00\%) & 4.00 (0.00\%) \\ \hline
edge\_count & 16.00 & \textbf{15.00 (6.25\%)} & 16.00 (0.00\%) & 16.00 (0.00\%) & 16.00 (0.00\%) \\ \hline
num\_subsinks & 2.00 & 2.00 (0.00\%) & 2.00 (0.00\%) & 2.00 (0.00\%) & 2.00 (0.00\%) \\ \hline
edge\_to\_node & 4.00 & \textbf{3.75 (6.25\%)} & 4.00 (0.00\%) & 4.00 (0.00\%) & 4.00 (0.00\%) \\ \hline
avg\_fitness & 2720.50 & 2720.50 (0.00\%) & 2720.50 (0.00\%) & 2720.50 (0.00\%) & 2720.50 (0.00\%) \\ \hline
distLO & 0.02 & \textbf{0.05 (203.37\%)} & \textbf{0.02 (2.24\%)} & \textbf{0.00 (79.17\%)} & \textbf{0.00 (97.45\%)} \\ \hline
conrel & 3.00 & 3.00 (0.00\%) & 3.00 (0.00\%) & 3.00 (0.00\%) & 3.00 (0.00\%) \\ \hline
avg\_loop\_weight & 57.00 & \textbf{83.75 (46.93\%)} & \textbf{257.00 (350.88\%)} & \textbf{1863.00 (3168.42\%)} & \textbf{17944.75 (31382.02\%)} \\ \hline
go\_path\_ratio & 0.75 & 0.75 (0.00\%) & 0.75 (0.00\%) & 0.75 (0.00\%) & 0.75 (0.00\%) \\ \hline
avg\_go\_path\_len & 1.00 & 1.00 (0.00\%) & 1.00 (0.00\%) & 1.00 (0.00\%) & 1.00 (0.00\%) \\ \hline
max\_go\_path\_len & 1.00 & 1.00 (0.00\%) & 1.00 (0.00\%) & 1.00 (0.00\%) & 1.00 (0.00\%) \\ \hline
num\_sinks & 2.00 & 2.00 (0.00\%) & 2.00 (0.00\%) & 2.00 (0.00\%) & 2.00 (0.00\%) \\ \hline
num\_sources & 2.00 & 2.00 (0.00\%) & 2.00 (0.00\%) & 2.00 (0.00\%) & 2.00 (0.00\%) \\ \hline
funnel\_num & 2.00 & 2.00 (0.00\%) & 2.00 (0.00\%) & 2.00 (0.00\%) & 2.00 (0.00\%) \\ \hline
mean\_funnel\_size & 3.00 & 3.00 (0.00\%) & 3.00 (0.00\%) & 3.00 (0.00\%) & 3.00 (0.00\%) \\ \hline
max\_funnel\_size & 3.00 & 3.00 (0.00\%) & 3.00 (0.00\%) & 3.00 (0.00\%) & 3.00 (0.00\%) \\ \hline
max\_out\_degree & 3.00 & 3.00 (0.00\%) & 3.00 (0.00\%) & 3.00 (0.00\%) & 3.00 (0.00\%) \\ \hline
avg\_out\_degree & 3.00 & \textbf{2.75 (8.33\%)} & 3.00 (0.00\%) & 3.00 (0.00\%) & 3.00 (0.00\%) \\ \hline
max\_in\_degree & 3.00 & 3.00 (0.00\%) & 3.00 (0.00\%) & 3.00 (0.00\%) & 3.00 (0.00\%) \\ \hline
avg\_in\_degree & 3.00 & \textbf{2.75 (8.33\%)} & 3.00 (0.00\%) & 3.00 (0.00\%) & 3.00 (0.00\%) \\ \hline
density & 1.33 & \textbf{1.25 (6.25\%)} & 1.33 (0.00\%) & 1.33 (0.00\%) & 1.33 (0.00\%) \\ \hline
num\_cc & 1.00 & 1.00 (0.00\%) & 1.00 (0.00\%) & 1.00 (0.00\%) & 1.00 (0.00\%) \\ \hline
largest\_cc & 4.00 & 4.00 (0.00\%) & 4.00 (0.00\%) & 4.00 (0.00\%) & 4.00 (0.00\%) \\ \hline
largest\_cc\_radius & 1.00 & 1.00 (0.00\%) & 1.00 (0.00\%) & 1.00 (0.00\%) & 1.00 (0.00\%) \\ \hline
avg\_path\_len & 1.00 & \textbf{1.08 (8.33\%)} & 1.00 (0.00\%) & 1.00 (0.00\%) & 1.00 (0.00\%) \\ \hline
largest\_clique\_size & 4.00 & 4.00 (0.00\%) & 4.00 (0.00\%) & 4.00 (0.00\%) & 4.00 (0.00\%) \\ \hline
reciprocity & 1.00 & \textbf{0.91 (9.09\%)} & 1.00 (0.00\%) & 1.00 (0.00\%) & 1.00 (0.00\%) \\ \hline
\end{tabular}
}
\end{table}

\begin{table}
\centering
\caption{Instancja uniform\_7 - porównanie wartości miar uzyskanych z próbkowania dwufazowego z wartościami z przeglądu zupełnego.}
\label{tab:comp_uniform_7_twophase}
\footnotesize
\resizebox{\textwidth}{!}{
\begin{tabular}{|l|l|l|l|l|l|}
\hline
\textbf{metryka} & \textbf{p. zupełny} & \textbf{dwufazowy\_10} & \textbf{dwufazowy\_100} & \textbf{dwufazowy\_1000} & \textbf{dwufazowy\_10000} \\ \hline
opt\_count & 720.00 & 107.00 & 10007.00 & 1000014.00 & 100000005.00 \\ \hline
oracle\_count & 42362.00 & 7157.00 & 787632.00 & 79970482.00 & 8007871548.00 \\ \hline
node\_count & 4.00 & 4.00 (0.00\%) & 4.00 (0.00\%) & 4.00 (0.00\%) & 4.00 (0.00\%) \\ \hline
edge\_count & 16.00 & \textbf{10.00 (37.50\%)} & \textbf{15.00 (6.25\%)} & 16.00 (0.00\%) & 16.00 (0.00\%) \\ \hline
num\_subsinks & 2.00 & 2.00 (0.00\%) & 2.00 (0.00\%) & 2.00 (0.00\%) & 2.00 (0.00\%) \\ \hline
edge\_to\_node & 4.00 & \textbf{2.50 (37.50\%)} & \textbf{3.75 (6.25\%)} & 4.00 (0.00\%) & 4.00 (0.00\%) \\ \hline
avg\_fitness & 2720.50 & 2720.50 (0.00\%) & 2720.50 (0.00\%) & 2720.50 (0.00\%) & 2720.50 (0.00\%) \\ \hline
distLO & 0.02 & \textbf{0.42 (2237.80\%)} & \textbf{0.33 (1748.65\%)} & \textbf{0.02 (2.72\%)} & \textbf{0.00 (92.46\%)} \\ \hline
conrel & 3.00 & \textbf{1.00 (66.67\%)} & 3.00 (0.00\%) & 3.00 (0.00\%) & 3.00 (0.00\%) \\ \hline
avg\_loop\_weight & 57.00 & \textbf{4.50 (92.11\%)} & \textbf{44.00 (22.81\%)} & \textbf{387.00 (578.95\%)} & \textbf{3772.00 (6517.54\%)} \\ \hline
go\_path\_ratio & 0.75 & \textbf{0.25 (66.67\%)} & 0.75 (0.00\%) & 0.75 (0.00\%) & 0.75 (0.00\%) \\ \hline
avg\_go\_path\_len & 1.00 & 1.00 (0.00\%) & 1.00 (0.00\%) & 1.00 (0.00\%) & 1.00 (0.00\%) \\ \hline
max\_go\_path\_len & 1.00 & 1.00 (0.00\%) & 1.00 (0.00\%) & 1.00 (0.00\%) & 1.00 (0.00\%) \\ \hline
num\_sinks & 2.00 & 2.00 (0.00\%) & 2.00 (0.00\%) & 2.00 (0.00\%) & 2.00 (0.00\%) \\ \hline
num\_sources & 2.00 & 2.00 (0.00\%) & 2.00 (0.00\%) & 2.00 (0.00\%) & 2.00 (0.00\%) \\ \hline
funnel\_num & 2.00 & 2.00 (0.00\%) & 2.00 (0.00\%) & 2.00 (0.00\%) & 2.00 (0.00\%) \\ \hline
mean\_funnel\_size & 3.00 & \textbf{2.50 (16.67\%)} & 3.00 (0.00\%) & 3.00 (0.00\%) & 3.00 (0.00\%) \\ \hline
max\_funnel\_size & 3.00 & 3.00 (0.00\%) & 3.00 (0.00\%) & 3.00 (0.00\%) & 3.00 (0.00\%) \\ \hline
max\_out\_degree & 3.00 & 3.00 (0.00\%) & 3.00 (0.00\%) & 3.00 (0.00\%) & 3.00 (0.00\%) \\ \hline
avg\_out\_degree & 3.00 & \textbf{1.50 (50.00\%)} & \textbf{2.75 (8.33\%)} & 3.00 (0.00\%) & 3.00 (0.00\%) \\ \hline
max\_in\_degree & 3.00 & 3.00 (0.00\%) & 3.00 (0.00\%) & 3.00 (0.00\%) & 3.00 (0.00\%) \\ \hline
avg\_in\_degree & 3.00 & \textbf{1.50 (50.00\%)} & \textbf{2.75 (8.33\%)} & 3.00 (0.00\%) & 3.00 (0.00\%) \\ \hline
density & 1.33 & \textbf{0.83 (37.50\%)} & \textbf{1.25 (6.25\%)} & 1.33 (0.00\%) & 1.33 (0.00\%) \\ \hline
num\_cc & 1.00 & 1.00 (0.00\%) & 1.00 (0.00\%) & 1.00 (0.00\%) & 1.00 (0.00\%) \\ \hline
largest\_cc & 4.00 & 4.00 (0.00\%) & 4.00 (0.00\%) & 4.00 (0.00\%) & 4.00 (0.00\%) \\ \hline
largest\_cc\_radius & 1.00 & 1.00 (0.00\%) & 1.00 (0.00\%) & 1.00 (0.00\%) & 1.00 (0.00\%) \\ \hline
avg\_path\_len & 1.00 & \textbf{1.14 (14.29\%)} & \textbf{1.08 (8.33\%)} & 1.00 (0.00\%) & 1.00 (0.00\%) \\ \hline
largest\_clique\_size & 4.00 & \textbf{3.00 (25.00\%)} & 4.00 (0.00\%) & 4.00 (0.00\%) & 4.00 (0.00\%) \\ \hline
reciprocity & 1.00 & \textbf{0.33 (66.67\%)} & \textbf{0.91 (9.09\%)} & 1.00 (0.00\%) & 1.00 (0.00\%) \\ \hline
\end{tabular}
}
\end{table}

\begin{table}
\centering
\caption{Instancja uniform\_8 - porównanie wartości miar uzyskanych z próbkowania snowball z wartościami z przeglądu zupełnego.}
\label{tab:comp_uniform_8_snowball}
\footnotesize
\resizebox{\textwidth}{!}{
\begin{tabular}{|l|l|l|l|l|l|}
\hline
\textbf{metryka} & \textbf{p. zupełny} & \textbf{snowball\_1} & \textbf{snowball\_10} & \textbf{snowball\_100} & \textbf{snowball\_1000} \\ \hline
opt\_count & 5040.00 & 401.00 & 1210.00 & 10300.00 & 101200.00 \\ \hline
oracle\_count & 570090.00 & 34384.00 & 106120.00 & 905660.00 & 8916992.00 \\ \hline
node\_count & 2.00 & 2.00 (0.00\%) & 2.00 (0.00\%) & 2.00 (0.00\%) & 2.00 (0.00\%) \\ \hline
edge\_count & 4.00 & 4.00 (0.00\%) & 4.00 (0.00\%) & 4.00 (0.00\%) & 4.00 (0.00\%) \\ \hline
num\_subsinks & 2.00 & 2.00 (0.00\%) & 2.00 (0.00\%) & 2.00 (0.00\%) & 2.00 (0.00\%) \\ \hline
edge\_to\_node & 2.00 & 2.00 (0.00\%) & 2.00 (0.00\%) & 2.00 (0.00\%) & 2.00 (0.00\%) \\ \hline
avg\_fitness & 3224.00 & 3224.00 (0.00\%) & 3224.00 (0.00\%) & 3224.00 (0.00\%) & 3224.00 (0.00\%) \\ \hline
distLO & 0.03 & \textbf{0.02 (15.71\%)} & \textbf{0.01 (65.97\%)} & \textbf{0.00 (96.21\%)} & \textbf{0.00 (99.64\%)} \\ \hline
conrel & 1.00 & 1.00 (0.00\%) & 1.00 (0.00\%) & 1.00 (0.00\%) & 1.00 (0.00\%) \\ \hline
avg\_loop\_weight & 206.00 & \textbf{171.50 (16.75\%)} & \textbf{532.50 (158.50\%)} & \textbf{4520.00 (2094.17\%)} & \textbf{44341.50 (21425.00\%)} \\ \hline
go\_path\_ratio & 0.50 & 0.50 (0.00\%) & 0.50 (0.00\%) & 0.50 (0.00\%) & 0.50 (0.00\%) \\ \hline
avg\_go\_path\_len & 1.00 & 1.00 (0.00\%) & 1.00 (0.00\%) & 1.00 (0.00\%) & 1.00 (0.00\%) \\ \hline
max\_go\_path\_len & 1.00 & 1.00 (0.00\%) & 1.00 (0.00\%) & 1.00 (0.00\%) & 1.00 (0.00\%) \\ \hline
num\_sinks & 2.00 & 2.00 (0.00\%) & 2.00 (0.00\%) & 2.00 (0.00\%) & 2.00 (0.00\%) \\ \hline
num\_sources & 2.00 & 2.00 (0.00\%) & 2.00 (0.00\%) & 2.00 (0.00\%) & 2.00 (0.00\%) \\ \hline
funnel\_num & 2.00 & 2.00 (0.00\%) & 2.00 (0.00\%) & 2.00 (0.00\%) & 2.00 (0.00\%) \\ \hline
mean\_funnel\_size & 1.00 & 1.00 (0.00\%) & 1.00 (0.00\%) & 1.00 (0.00\%) & 1.00 (0.00\%) \\ \hline
max\_funnel\_size & 1.00 & 1.00 (0.00\%) & 1.00 (0.00\%) & 1.00 (0.00\%) & 1.00 (0.00\%) \\ \hline
max\_out\_degree & 1.00 & 1.00 (0.00\%) & 1.00 (0.00\%) & 1.00 (0.00\%) & 1.00 (0.00\%) \\ \hline
avg\_out\_degree & 1.00 & 1.00 (0.00\%) & 1.00 (0.00\%) & 1.00 (0.00\%) & 1.00 (0.00\%) \\ \hline
max\_in\_degree & 1.00 & 1.00 (0.00\%) & 1.00 (0.00\%) & 1.00 (0.00\%) & 1.00 (0.00\%) \\ \hline
avg\_in\_degree & 1.00 & 1.00 (0.00\%) & 1.00 (0.00\%) & 1.00 (0.00\%) & 1.00 (0.00\%) \\ \hline
density & 2.00 & 2.00 (0.00\%) & 2.00 (0.00\%) & 2.00 (0.00\%) & 2.00 (0.00\%) \\ \hline
num\_cc & 1.00 & 1.00 (0.00\%) & 1.00 (0.00\%) & 1.00 (0.00\%) & 1.00 (0.00\%) \\ \hline
largest\_cc & 2.00 & 2.00 (0.00\%) & 2.00 (0.00\%) & 2.00 (0.00\%) & 2.00 (0.00\%) \\ \hline
largest\_cc\_radius & 1.00 & 1.00 (0.00\%) & 1.00 (0.00\%) & 1.00 (0.00\%) & 1.00 (0.00\%) \\ \hline
avg\_path\_len & 1.00 & 1.00 (0.00\%) & 1.00 (0.00\%) & 1.00 (0.00\%) & 1.00 (0.00\%) \\ \hline
largest\_clique\_size & 2.00 & 2.00 (0.00\%) & 2.00 (0.00\%) & 2.00 (0.00\%) & 2.00 (0.00\%) \\ \hline
reciprocity & 1.00 & 1.00 (0.00\%) & 1.00 (0.00\%) & 1.00 (0.00\%) & 1.00 (0.00\%) \\ \hline
\end{tabular}
}
\end{table}

\begin{table}
\centering
\caption{Instancja uniform\_8 - porównanie wartości miar uzyskanych z próbkowania dwufazowego z wartościami z przeglądu zupełnego.}
\label{tab:comp_uniform_8_twophase}
\footnotesize
\resizebox{\textwidth}{!}{
\begin{tabular}{|l|l|l|l|l|l|}
\hline
\textbf{metryka} & \textbf{p. zupełny} & \textbf{dwufazowy\_10} & \textbf{dwufazowy\_100} & \textbf{dwufazowy\_1000} & \textbf{dwufazowy\_10000} \\ \hline
opt\_count & 5040.00 & 102.00 & 10003.00 & 1000002.00 & 100000007.00 \\ \hline
oracle\_count & 570090.00 & 13443.00 & 1461069.00 & 146837491.00 & 14695563426.00 \\ \hline
node\_count & 2.00 & 2.00 (0.00\%) & 2.00 (0.00\%) & 2.00 (0.00\%) & 2.00 (0.00\%) \\ \hline
edge\_count & 4.00 & 4.00 (0.00\%) & 4.00 (0.00\%) & 4.00 (0.00\%) & 4.00 (0.00\%) \\ \hline
num\_subsinks & 2.00 & 2.00 (0.00\%) & 2.00 (0.00\%) & 2.00 (0.00\%) & 2.00 (0.00\%) \\ \hline
edge\_to\_node & 2.00 & 2.00 (0.00\%) & 2.00 (0.00\%) & 2.00 (0.00\%) & 2.00 (0.00\%) \\ \hline
avg\_fitness & 3224.00 & 3224.00 (0.00\%) & 3224.00 (0.00\%) & 3224.00 (0.00\%) & 3224.00 (0.00\%) \\ \hline
distLO & 0.03 & \textbf{0.31 (1020.98\%)} & \textbf{0.10 (256.68\%)} & \textbf{0.01 (48.24\%)} & \textbf{0.00 (99.50\%)} \\ \hline
conrel & 1.00 & 1.00 (0.00\%) & 1.00 (0.00\%) & 1.00 (0.00\%) & 1.00 (0.00\%) \\ \hline
avg\_loop\_weight & 206.00 & \textbf{8.50 (95.87\%)} & \textbf{65.00 (68.45\%)} & \textbf{674.50 (227.43\%)} & \textbf{6772.00 (3187.38\%)} \\ \hline
go\_path\_ratio & 0.50 & 0.50 (0.00\%) & 0.50 (0.00\%) & 0.50 (0.00\%) & 0.50 (0.00\%) \\ \hline
avg\_go\_path\_len & 1.00 & 1.00 (0.00\%) & 1.00 (0.00\%) & 1.00 (0.00\%) & 1.00 (0.00\%) \\ \hline
max\_go\_path\_len & 1.00 & 1.00 (0.00\%) & 1.00 (0.00\%) & 1.00 (0.00\%) & 1.00 (0.00\%) \\ \hline
num\_sinks & 2.00 & 2.00 (0.00\%) & 2.00 (0.00\%) & 2.00 (0.00\%) & 2.00 (0.00\%) \\ \hline
num\_sources & 2.00 & 2.00 (0.00\%) & 2.00 (0.00\%) & 2.00 (0.00\%) & 2.00 (0.00\%) \\ \hline
funnel\_num & 2.00 & 2.00 (0.00\%) & 2.00 (0.00\%) & 2.00 (0.00\%) & 2.00 (0.00\%) \\ \hline
mean\_funnel\_size & 1.00 & 1.00 (0.00\%) & 1.00 (0.00\%) & 1.00 (0.00\%) & 1.00 (0.00\%) \\ \hline
max\_funnel\_size & 1.00 & 1.00 (0.00\%) & 1.00 (0.00\%) & 1.00 (0.00\%) & 1.00 (0.00\%) \\ \hline
max\_out\_degree & 1.00 & 1.00 (0.00\%) & 1.00 (0.00\%) & 1.00 (0.00\%) & 1.00 (0.00\%) \\ \hline
avg\_out\_degree & 1.00 & 1.00 (0.00\%) & 1.00 (0.00\%) & 1.00 (0.00\%) & 1.00 (0.00\%) \\ \hline
max\_in\_degree & 1.00 & 1.00 (0.00\%) & 1.00 (0.00\%) & 1.00 (0.00\%) & 1.00 (0.00\%) \\ \hline
avg\_in\_degree & 1.00 & 1.00 (0.00\%) & 1.00 (0.00\%) & 1.00 (0.00\%) & 1.00 (0.00\%) \\ \hline
density & 2.00 & 2.00 (0.00\%) & 2.00 (0.00\%) & 2.00 (0.00\%) & 2.00 (0.00\%) \\ \hline
num\_cc & 1.00 & 1.00 (0.00\%) & 1.00 (0.00\%) & 1.00 (0.00\%) & 1.00 (0.00\%) \\ \hline
largest\_cc & 2.00 & 2.00 (0.00\%) & 2.00 (0.00\%) & 2.00 (0.00\%) & 2.00 (0.00\%) \\ \hline
largest\_cc\_radius & 1.00 & 1.00 (0.00\%) & 1.00 (0.00\%) & 1.00 (0.00\%) & 1.00 (0.00\%) \\ \hline
avg\_path\_len & 1.00 & 1.00 (0.00\%) & 1.00 (0.00\%) & 1.00 (0.00\%) & 1.00 (0.00\%) \\ \hline
largest\_clique\_size & 2.00 & 2.00 (0.00\%) & 2.00 (0.00\%) & 2.00 (0.00\%) & 2.00 (0.00\%) \\ \hline
reciprocity & 1.00 & 1.00 (0.00\%) & 1.00 (0.00\%) & 1.00 (0.00\%) & 1.00 (0.00\%) \\ \hline
\end{tabular}
}
\end{table}

\begin{table}
\centering
\caption{Instancja uniform\_9 - porównanie wartości miar uzyskanych z próbkowania snowball z wartościami z przeglądu zupełnego.}
\label{tab:comp_uniform_9_snowball}
\footnotesize
\resizebox{\textwidth}{!}{
\begin{tabular}{|l|l|l|l|l|l|}
\hline
\textbf{metryka} & \textbf{p. zupełny} & \textbf{snowball\_1} & \textbf{snowball\_10} & \textbf{snowball\_100} & \textbf{snowball\_1000} \\ \hline
opt\_count & 40320.00 & 401.00 & 1210.00 & 10300.00 & 101200.00 \\ \hline
oracle\_count & 6071702.00 & 44460.00 & 136116.00 & 1165968.00 & 11466720.00 \\ \hline
node\_count & 2.00 & 2.00 (0.00\%) & 2.00 (0.00\%) & 2.00 (0.00\%) & 2.00 (0.00\%) \\ \hline
edge\_count & 4.00 & 4.00 (0.00\%) & 4.00 (0.00\%) & 4.00 (0.00\%) & 4.00 (0.00\%) \\ \hline
num\_subsinks & 2.00 & 2.00 (0.00\%) & 2.00 (0.00\%) & 2.00 (0.00\%) & 2.00 (0.00\%) \\ \hline
edge\_to\_node & 2.00 & 2.00 (0.00\%) & 2.00 (0.00\%) & 2.00 (0.00\%) & 2.00 (0.00\%) \\ \hline
avg\_fitness & 2482.00 & 2482.00 (0.00\%) & 2482.00 (0.00\%) & 2482.00 (0.00\%) & 2482.00 (0.00\%) \\ \hline
distLO & 0.00 & \textbf{0.04 (730.53\%)} & \textbf{0.02 (252.09\%)} & \textbf{0.00 (73.72\%)} & \textbf{0.00 (98.09\%)} \\ \hline
conrel & 1.00 & 1.00 (0.00\%) & 1.00 (0.00\%) & 1.00 (0.00\%) & 1.00 (0.00\%) \\ \hline
avg\_loop\_weight & 390.00 & \textbf{180.50 (53.72\%)} & \textbf{528.00 (35.38\%)} & \textbf{4527.50 (1060.90\%)} & \textbf{44516.00 (11314.36\%)} \\ \hline
go\_path\_ratio & 0.50 & 0.50 (0.00\%) & 0.50 (0.00\%) & 0.50 (0.00\%) & 0.50 (0.00\%) \\ \hline
avg\_go\_path\_len & 1.00 & 1.00 (0.00\%) & 1.00 (0.00\%) & 1.00 (0.00\%) & 1.00 (0.00\%) \\ \hline
max\_go\_path\_len & 1.00 & 1.00 (0.00\%) & 1.00 (0.00\%) & 1.00 (0.00\%) & 1.00 (0.00\%) \\ \hline
num\_sinks & 2.00 & 2.00 (0.00\%) & 2.00 (0.00\%) & 2.00 (0.00\%) & 2.00 (0.00\%) \\ \hline
num\_sources & 2.00 & 2.00 (0.00\%) & 2.00 (0.00\%) & 2.00 (0.00\%) & 2.00 (0.00\%) \\ \hline
funnel\_num & 2.00 & 2.00 (0.00\%) & 2.00 (0.00\%) & 2.00 (0.00\%) & 2.00 (0.00\%) \\ \hline
mean\_funnel\_size & 1.00 & 1.00 (0.00\%) & 1.00 (0.00\%) & 1.00 (0.00\%) & 1.00 (0.00\%) \\ \hline
max\_funnel\_size & 1.00 & 1.00 (0.00\%) & 1.00 (0.00\%) & 1.00 (0.00\%) & 1.00 (0.00\%) \\ \hline
max\_out\_degree & 1.00 & 1.00 (0.00\%) & 1.00 (0.00\%) & 1.00 (0.00\%) & 1.00 (0.00\%) \\ \hline
avg\_out\_degree & 1.00 & 1.00 (0.00\%) & 1.00 (0.00\%) & 1.00 (0.00\%) & 1.00 (0.00\%) \\ \hline
max\_in\_degree & 1.00 & 1.00 (0.00\%) & 1.00 (0.00\%) & 1.00 (0.00\%) & 1.00 (0.00\%) \\ \hline
avg\_in\_degree & 1.00 & 1.00 (0.00\%) & 1.00 (0.00\%) & 1.00 (0.00\%) & 1.00 (0.00\%) \\ \hline
density & 2.00 & 2.00 (0.00\%) & 2.00 (0.00\%) & 2.00 (0.00\%) & 2.00 (0.00\%) \\ \hline
num\_cc & 1.00 & 1.00 (0.00\%) & 1.00 (0.00\%) & 1.00 (0.00\%) & 1.00 (0.00\%) \\ \hline
largest\_cc & 2.00 & 2.00 (0.00\%) & 2.00 (0.00\%) & 2.00 (0.00\%) & 2.00 (0.00\%) \\ \hline
largest\_cc\_radius & 1.00 & 1.00 (0.00\%) & 1.00 (0.00\%) & 1.00 (0.00\%) & 1.00 (0.00\%) \\ \hline
avg\_path\_len & 1.00 & 1.00 (0.00\%) & 1.00 (0.00\%) & 1.00 (0.00\%) & 1.00 (0.00\%) \\ \hline
largest\_clique\_size & 2.00 & 2.00 (0.00\%) & 2.00 (0.00\%) & 2.00 (0.00\%) & 2.00 (0.00\%) \\ \hline
reciprocity & 1.00 & 1.00 (0.00\%) & 1.00 (0.00\%) & 1.00 (0.00\%) & 1.00 (0.00\%) \\ \hline
\end{tabular}
}
\end{table}

\begin{table}
\centering
\caption{Instancja uniform\_9 - porównanie wartości miar uzyskanych z próbkowania dwufazowego z wartościami z przeglądu zupełnego.}
\label{tab:comp_uniform_9_twophase}
\footnotesize
\resizebox{\textwidth}{!}{
\begin{tabular}{|l|l|l|l|l|l|}
\hline
\textbf{metryka} & \textbf{p. zupełny} & \textbf{dwufazowy\_10} & \textbf{dwufazowy\_100} & \textbf{dwufazowy\_1000} & \textbf{dwufazowy\_10000} \\ \hline
opt\_count & 40320.00 & 102.00 & 10002.00 & 1000004.00 & 100000003.00 \\ \hline
oracle\_count & 6071702.00 & 19826.00 & 2100542.00 & 211780859.00 & 21201408685.00 \\ \hline
node\_count & 2.00 & 2.00 (0.00\%) & 2.00 (0.00\%) & 2.00 (0.00\%) & 2.00 (0.00\%) \\ \hline
edge\_count & 4.00 & 4.00 (0.00\%) & 4.00 (0.00\%) & 4.00 (0.00\%) & 4.00 (0.00\%) \\ \hline
num\_subsinks & 2.00 & 2.00 (0.00\%) & 2.00 (0.00\%) & 2.00 (0.00\%) & 2.00 (0.00\%) \\ \hline
edge\_to\_node & 2.00 & 2.00 (0.00\%) & 2.00 (0.00\%) & 2.00 (0.00\%) & 2.00 (0.00\%) \\ \hline
avg\_fitness & 2482.00 & 2482.00 (0.00\%) & 2482.00 (0.00\%) & 2482.00 (0.00\%) & 2482.00 (0.00\%) \\ \hline
distLO & 0.00 & \textbf{0.31 (6479.34\%)} & \textbf{0.03 (542.39\%)} & \textbf{0.00 (40.41\%)} & \textbf{0.00 (93.99\%)} \\ \hline
conrel & 1.00 & 1.00 (0.00\%) & 1.00 (0.00\%) & 1.00 (0.00\%) & 1.00 (0.00\%) \\ \hline
avg\_loop\_weight & 390.00 & \textbf{8.50 (97.82\%)} & \textbf{81.00 (79.23\%)} & \textbf{799.00 (104.87\%)} & \textbf{8078.00 (1971.28\%)} \\ \hline
go\_path\_ratio & 0.50 & 0.50 (0.00\%) & 0.50 (0.00\%) & 0.50 (0.00\%) & 0.50 (0.00\%) \\ \hline
avg\_go\_path\_len & 1.00 & 1.00 (0.00\%) & 1.00 (0.00\%) & 1.00 (0.00\%) & 1.00 (0.00\%) \\ \hline
max\_go\_path\_len & 1.00 & 1.00 (0.00\%) & 1.00 (0.00\%) & 1.00 (0.00\%) & 1.00 (0.00\%) \\ \hline
num\_sinks & 2.00 & 2.00 (0.00\%) & 2.00 (0.00\%) & 2.00 (0.00\%) & 2.00 (0.00\%) \\ \hline
num\_sources & 2.00 & 2.00 (0.00\%) & 2.00 (0.00\%) & 2.00 (0.00\%) & 2.00 (0.00\%) \\ \hline
funnel\_num & 2.00 & 2.00 (0.00\%) & 2.00 (0.00\%) & 2.00 (0.00\%) & 2.00 (0.00\%) \\ \hline
mean\_funnel\_size & 1.00 & 1.00 (0.00\%) & 1.00 (0.00\%) & 1.00 (0.00\%) & 1.00 (0.00\%) \\ \hline
max\_funnel\_size & 1.00 & 1.00 (0.00\%) & 1.00 (0.00\%) & 1.00 (0.00\%) & 1.00 (0.00\%) \\ \hline
max\_out\_degree & 1.00 & 1.00 (0.00\%) & 1.00 (0.00\%) & 1.00 (0.00\%) & 1.00 (0.00\%) \\ \hline
avg\_out\_degree & 1.00 & 1.00 (0.00\%) & 1.00 (0.00\%) & 1.00 (0.00\%) & 1.00 (0.00\%) \\ \hline
max\_in\_degree & 1.00 & 1.00 (0.00\%) & 1.00 (0.00\%) & 1.00 (0.00\%) & 1.00 (0.00\%) \\ \hline
avg\_in\_degree & 1.00 & 1.00 (0.00\%) & 1.00 (0.00\%) & 1.00 (0.00\%) & 1.00 (0.00\%) \\ \hline
density & 2.00 & 2.00 (0.00\%) & 2.00 (0.00\%) & 2.00 (0.00\%) & 2.00 (0.00\%) \\ \hline
num\_cc & 1.00 & 1.00 (0.00\%) & 1.00 (0.00\%) & 1.00 (0.00\%) & 1.00 (0.00\%) \\ \hline
largest\_cc & 2.00 & 2.00 (0.00\%) & 2.00 (0.00\%) & 2.00 (0.00\%) & 2.00 (0.00\%) \\ \hline
largest\_cc\_radius & 1.00 & 1.00 (0.00\%) & 1.00 (0.00\%) & 1.00 (0.00\%) & 1.00 (0.00\%) \\ \hline
avg\_path\_len & 1.00 & 1.00 (0.00\%) & 1.00 (0.00\%) & 1.00 (0.00\%) & 1.00 (0.00\%) \\ \hline
largest\_clique\_size & 2.00 & 2.00 (0.00\%) & 2.00 (0.00\%) & 2.00 (0.00\%) & 2.00 (0.00\%) \\ \hline
reciprocity & 1.00 & 1.00 (0.00\%) & 1.00 (0.00\%) & 1.00 (0.00\%) & 1.00 (0.00\%) \\ \hline
\end{tabular}
}
\end{table}

\begin{table}
\centering
\caption{Instancja uniform\_10 - porównanie wartości miar uzyskanych z próbkowania snowball z wartościami z przeglądu zupełnego.}
\label{tab:comp_uniform_10_snowball}
\footnotesize
\resizebox{\textwidth}{!}{
\begin{tabular}{|l|l|l|l|l|l|}
\hline
\textbf{metryka} & \textbf{p. zupełny} & \textbf{snowball\_1} & \textbf{snowball\_10} & \textbf{snowball\_100} & \textbf{snowball\_1000} \\ \hline
opt\_count & 362880.00 & 601.00 & 2006.00 & 11496.00 & 102496.00 \\ \hline
oracle\_count & 70346836.00 & 86985.00 & 293130.00 & 1674675.00 & 14796000.00 \\ \hline
node\_count & 6.00 & 6.00 (0.00\%) & 6.00 (0.00\%) & 6.00 (0.00\%) & 6.00 (0.00\%) \\ \hline
edge\_count & 35.00 & \textbf{20.00 (42.86\%)} & \textbf{30.00 (14.29\%)} & \textbf{32.00 (8.57\%)} & \textbf{32.00 (8.57\%)} \\ \hline
num\_subsinks & 2.00 & \textbf{3.00 (50.00\%)} & 2.00 (0.00\%) & 2.00 (0.00\%) & 2.00 (0.00\%) \\ \hline
edge\_to\_node & 5.83 & \textbf{3.33 (42.86\%)} & \textbf{5.00 (14.29\%)} & \textbf{5.33 (8.57\%)} & \textbf{5.33 (8.57\%)} \\ \hline
avg\_fitness & 2111.67 & 2111.67 (0.00\%) & 2111.67 (0.00\%) & 2111.67 (0.00\%) & 2111.67 (0.00\%) \\ \hline
distLO & 0.01 & \textbf{0.08 (1073.45\%)} & \textbf{0.13 (1844.98\%)} & \textbf{0.03 (301.38\%)} & \textbf{0.00 (38.98\%)} \\ \hline
conrel & 5.00 & \textbf{1.00 (80.00\%)} & 5.00 (0.00\%) & 5.00 (0.00\%) & 5.00 (0.00\%) \\ \hline
avg\_loop\_weight & 352.50 & \textbf{98.60 (72.03\%)} & \textbf{262.33 (25.58\%)} & \textbf{1550.00 (339.72\%)} & \textbf{14307.83 (3958.96\%)} \\ \hline
go\_path\_ratio & 0.83 & \textbf{0.67 (20.00\%)} & 0.83 (0.00\%) & 0.83 (0.00\%) & 0.83 (0.00\%) \\ \hline
avg\_go\_path\_len & 1.00 & \textbf{1.25 (25.00\%)} & 1.00 (0.00\%) & 1.00 (0.00\%) & 1.00 (0.00\%) \\ \hline
max\_go\_path\_len & 1.00 & \textbf{2.00 (100.00\%)} & 1.00 (0.00\%) & 1.00 (0.00\%) & 1.00 (0.00\%) \\ \hline
num\_sinks & 2.00 & \textbf{3.00 (50.00\%)} & 2.00 (0.00\%) & 2.00 (0.00\%) & 2.00 (0.00\%) \\ \hline
num\_sources & 2.00 & 2.00 (0.00\%) & 2.00 (0.00\%) & 2.00 (0.00\%) & 2.00 (0.00\%) \\ \hline
funnel\_num & 2.00 & \textbf{3.00 (50.00\%)} & 2.00 (0.00\%) & 2.00 (0.00\%) & 2.00 (0.00\%) \\ \hline
mean\_funnel\_size & 5.00 & \textbf{2.67 (46.67\%)} & 5.00 (0.00\%) & 5.00 (0.00\%) & 5.00 (0.00\%) \\ \hline
max\_funnel\_size & 5.00 & \textbf{4.00 (20.00\%)} & 5.00 (0.00\%) & 5.00 (0.00\%) & 5.00 (0.00\%) \\ \hline
max\_out\_degree & 5.00 & 5.00 (0.00\%) & 5.00 (0.00\%) & 5.00 (0.00\%) & 5.00 (0.00\%) \\ \hline
avg\_out\_degree & 4.83 & \textbf{2.50 (48.28\%)} & \textbf{4.00 (17.24\%)} & \textbf{4.33 (10.34\%)} & \textbf{4.33 (10.34\%)} \\ \hline
max\_in\_degree & 5.00 & \textbf{3.00 (40.00\%)} & 5.00 (0.00\%) & 5.00 (0.00\%) & 5.00 (0.00\%) \\ \hline
avg\_in\_degree & 4.83 & \textbf{2.50 (48.28\%)} & \textbf{4.00 (17.24\%)} & \textbf{4.33 (10.34\%)} & \textbf{4.33 (10.34\%)} \\ \hline
density & 1.17 & \textbf{0.67 (42.86\%)} & \textbf{1.00 (14.29\%)} & \textbf{1.07 (8.57\%)} & \textbf{1.07 (8.57\%)} \\ \hline
num\_cc & 1.00 & 1.00 (0.00\%) & 1.00 (0.00\%) & 1.00 (0.00\%) & 1.00 (0.00\%) \\ \hline
largest\_cc & 6.00 & 6.00 (0.00\%) & 6.00 (0.00\%) & 6.00 (0.00\%) & 6.00 (0.00\%) \\ \hline
largest\_cc\_radius & 1.00 & 1.00 (0.00\%) & 1.00 (0.00\%) & 1.00 (0.00\%) & 1.00 (0.00\%) \\ \hline
avg\_path\_len & 1.03 & \textbf{1.48 (43.23\%)} & \textbf{1.20 (16.13\%)} & \textbf{1.13 (9.68\%)} & \textbf{1.13 (9.68\%)} \\ \hline
largest\_clique\_size & 6.00 & \textbf{4.00 (33.33\%)} & 6.00 (0.00\%) & 6.00 (0.00\%) & 6.00 (0.00\%) \\ \hline
reciprocity & 0.97 & \textbf{0.67 (30.95\%)} & \textbf{0.75 (22.32\%)} & \textbf{0.85 (12.36\%)} & \textbf{0.85 (12.36\%)} \\ \hline
\end{tabular}
}
\end{table}

\begin{table}
\centering
\caption{Instancja uniform\_10 - porównanie wartości miar uzyskanych z próbkowania dwufazowego z wartościami z przeglądu zupełnego.}
\label{tab:comp_uniform_10_twophase}
\footnotesize
\resizebox{\textwidth}{!}{
\begin{tabular}{|l|l|l|l|l|l|}
\hline
\textbf{metryka} & \textbf{p. zupełny} & \textbf{dwufazowy\_10} & \textbf{dwufazowy\_100} & \textbf{dwufazowy\_1000} & \textbf{dwufazowy\_10000} \\ \hline
opt\_count & 362880.00 & 122.00 & 10014.00 & 1000027.00 & 100000035.00 \\ \hline
oracle\_count & 70346836.00 & 26110.00 & 2795633.00 & 287853142.00 & 28873812637.00 \\ \hline
node\_count & 6.00 & 6.00 (0.00\%) & 6.00 (0.00\%) & 6.00 (0.00\%) & 6.00 (0.00\%) \\ \hline
edge\_count & 35.00 & \textbf{22.00 (37.14\%)} & \textbf{33.00 (5.71\%)} & \textbf{34.00 (2.86\%)} & 35.00 (0.00\%) \\ \hline
num\_subsinks & 2.00 & 2.00 (0.00\%) & 2.00 (0.00\%) & 2.00 (0.00\%) & 2.00 (0.00\%) \\ \hline
edge\_to\_node & 5.83 & \textbf{3.67 (37.14\%)} & \textbf{5.50 (5.71\%)} & \textbf{5.67 (2.86\%)} & 5.83 (0.00\%) \\ \hline
avg\_fitness & 2111.67 & 2111.67 (0.00\%) & 2111.67 (0.00\%) & 2111.67 (0.00\%) & 2111.67 (0.00\%) \\ \hline
distLO & 0.01 & \textbf{0.50 (7332.43\%)} & \textbf{0.12 (1702.30\%)} & \textbf{0.01 (0.29\%)} & \textbf{0.00 (79.86\%)} \\ \hline
conrel & 5.00 & \textbf{2.00 (60.00\%)} & 5.00 (0.00\%) & 5.00 (0.00\%) & 5.00 (0.00\%) \\ \hline
avg\_loop\_weight & 352.50 & \textbf{4.20 (98.81\%)} & \textbf{36.00 (89.79\%)} & \textbf{404.83 (14.85\%)} & \textbf{4023.33 (1041.37\%)} \\ \hline
go\_path\_ratio & 0.83 & 0.83 (0.00\%) & 0.83 (0.00\%) & 0.83 (0.00\%) & 0.83 (0.00\%) \\ \hline
avg\_go\_path\_len & 1.00 & \textbf{1.40 (40.00\%)} & 1.00 (0.00\%) & 1.00 (0.00\%) & 1.00 (0.00\%) \\ \hline
max\_go\_path\_len & 1.00 & \textbf{2.00 (100.00\%)} & 1.00 (0.00\%) & 1.00 (0.00\%) & 1.00 (0.00\%) \\ \hline
num\_sinks & 2.00 & 2.00 (0.00\%) & 2.00 (0.00\%) & 2.00 (0.00\%) & 2.00 (0.00\%) \\ \hline
num\_sources & 2.00 & 2.00 (0.00\%) & 2.00 (0.00\%) & 2.00 (0.00\%) & 2.00 (0.00\%) \\ \hline
funnel\_num & 2.00 & 2.00 (0.00\%) & 2.00 (0.00\%) & 2.00 (0.00\%) & 2.00 (0.00\%) \\ \hline
mean\_funnel\_size & 5.00 & \textbf{4.50 (10.00\%)} & 5.00 (0.00\%) & 5.00 (0.00\%) & 5.00 (0.00\%) \\ \hline
max\_funnel\_size & 5.00 & 5.00 (0.00\%) & 5.00 (0.00\%) & 5.00 (0.00\%) & 5.00 (0.00\%) \\ \hline
max\_out\_degree & 5.00 & \textbf{4.00 (20.00\%)} & 5.00 (0.00\%) & 5.00 (0.00\%) & 5.00 (0.00\%) \\ \hline
avg\_out\_degree & 4.83 & \textbf{2.83 (41.38\%)} & \textbf{4.50 (6.90\%)} & \textbf{4.67 (3.45\%)} & 4.83 (0.00\%) \\ \hline
max\_in\_degree & 5.00 & 5.00 (0.00\%) & 5.00 (0.00\%) & 5.00 (0.00\%) & 5.00 (0.00\%) \\ \hline
avg\_in\_degree & 4.83 & \textbf{2.83 (41.38\%)} & \textbf{4.50 (6.90\%)} & \textbf{4.67 (3.45\%)} & 4.83 (0.00\%) \\ \hline
density & 1.17 & \textbf{0.73 (37.14\%)} & \textbf{1.10 (5.71\%)} & \textbf{1.13 (2.86\%)} & 1.17 (0.00\%) \\ \hline
num\_cc & 1.00 & 1.00 (0.00\%) & 1.00 (0.00\%) & 1.00 (0.00\%) & 1.00 (0.00\%) \\ \hline
largest\_cc & 6.00 & 6.00 (0.00\%) & 6.00 (0.00\%) & 6.00 (0.00\%) & 6.00 (0.00\%) \\ \hline
largest\_cc\_radius & 1.00 & 1.00 (0.00\%) & 1.00 (0.00\%) & 1.00 (0.00\%) & 1.00 (0.00\%) \\ \hline
avg\_path\_len & 1.03 & \textbf{1.15 (11.29\%)} & \textbf{1.10 (6.45\%)} & \textbf{1.07 (3.23\%)} & 1.03 (0.00\%) \\ \hline
largest\_clique\_size & 6.00 & \textbf{5.00 (16.67\%)} & 6.00 (0.00\%) & 6.00 (0.00\%) & 6.00 (0.00\%) \\ \hline
reciprocity & 0.97 & \textbf{0.47 (51.26\%)} & \textbf{0.89 (7.94\%)} & \textbf{0.93 (3.83\%)} & 0.97 (0.00\%) \\ \hline
\end{tabular}
}
\end{table}

\begin{table}
\centering
\caption{Instancja uniform\_11 - porównanie wartości miar uzyskanych z próbkowania snowball z wartościami z przeglądu zupełnego.}
\label{tab:comp_uniform_11_snowball}
\footnotesize
\resizebox{\textwidth}{!}{
\begin{tabular}{|l|l|l|l|l|l|}
\hline
\textbf{metryka} & \textbf{p. zupełny} & \textbf{snowball\_1} & \textbf{snowball\_10} & \textbf{snowball\_100} & \textbf{snowball\_1000} \\ \hline
opt\_count & 3628800.00 & 401.00 & 1408.00 & 10898.00 & 101698.00 \\ \hline
oracle\_count & 979464676.00 & 72105.00 & 253110.00 & 1986930.00 & 18375555.00 \\ \hline
node\_count & 4.00 & 4.00 (0.00\%) & 4.00 (0.00\%) & 4.00 (0.00\%) & 4.00 (0.00\%) \\ \hline
edge\_count & 16.00 & \textbf{10.00 (37.50\%)} & \textbf{14.00 (12.50\%)} & \textbf{14.00 (12.50\%)} & \textbf{14.00 (12.50\%)} \\ \hline
num\_subsinks & 2.00 & \textbf{3.00 (50.00\%)} & 2.00 (0.00\%) & 2.00 (0.00\%) & 2.00 (0.00\%) \\ \hline
edge\_to\_node & 4.00 & \textbf{2.50 (37.50\%)} & \textbf{3.50 (12.50\%)} & \textbf{3.50 (12.50\%)} & \textbf{3.50 (12.50\%)} \\ \hline
avg\_fitness & 2632.50 & 2632.50 (0.00\%) & 2632.50 (0.00\%) & 2632.50 (0.00\%) & 2632.50 (0.00\%) \\ \hline
distLO & 0.00 & \textbf{0.10 (3840.85\%)} & \textbf{0.10 (3748.92\%)} & \textbf{0.01 (400.38\%)} & \textbf{0.00 (9.32\%)} \\ \hline
conrel & 3.00 & \textbf{1.00 (66.67\%)} & 3.00 (0.00\%) & 3.00 (0.00\%) & 3.00 (0.00\%) \\ \hline
avg\_loop\_weight & 738.25 & \textbf{120.33 (83.70\%)} & \textbf{306.50 (58.48\%)} & \textbf{2400.00 (225.09\%)} & \textbf{23065.00 (3024.28\%)} \\ \hline
go\_path\_ratio & 0.75 & \textbf{0.50 (33.33\%)} & 0.75 (0.00\%) & 0.75 (0.00\%) & 0.75 (0.00\%) \\ \hline
avg\_go\_path\_len & 1.00 & 1.00 (0.00\%) & 1.00 (0.00\%) & 1.00 (0.00\%) & 1.00 (0.00\%) \\ \hline
max\_go\_path\_len & 1.00 & 1.00 (0.00\%) & 1.00 (0.00\%) & 1.00 (0.00\%) & 1.00 (0.00\%) \\ \hline
num\_sinks & 2.00 & \textbf{3.00 (50.00\%)} & 2.00 (0.00\%) & 2.00 (0.00\%) & 2.00 (0.00\%) \\ \hline
num\_sources & 2.00 & 2.00 (0.00\%) & 2.00 (0.00\%) & 2.00 (0.00\%) & 2.00 (0.00\%) \\ \hline
funnel\_num & 2.00 & \textbf{3.00 (50.00\%)} & 2.00 (0.00\%) & 2.00 (0.00\%) & 2.00 (0.00\%) \\ \hline
mean\_funnel\_size & 3.00 & \textbf{1.67 (44.44\%)} & 3.00 (0.00\%) & 3.00 (0.00\%) & 3.00 (0.00\%) \\ \hline
max\_funnel\_size & 3.00 & \textbf{2.00 (33.33\%)} & 3.00 (0.00\%) & 3.00 (0.00\%) & 3.00 (0.00\%) \\ \hline
max\_out\_degree & 3.00 & 3.00 (0.00\%) & 3.00 (0.00\%) & 3.00 (0.00\%) & 3.00 (0.00\%) \\ \hline
avg\_out\_degree & 3.00 & \textbf{1.75 (41.67\%)} & \textbf{2.50 (16.67\%)} & \textbf{2.50 (16.67\%)} & \textbf{2.50 (16.67\%)} \\ \hline
max\_in\_degree & 3.00 & \textbf{2.00 (33.33\%)} & 3.00 (0.00\%) & 3.00 (0.00\%) & 3.00 (0.00\%) \\ \hline
avg\_in\_degree & 3.00 & \textbf{1.75 (41.67\%)} & \textbf{2.50 (16.67\%)} & \textbf{2.50 (16.67\%)} & \textbf{2.50 (16.67\%)} \\ \hline
density & 1.33 & \textbf{0.83 (37.50\%)} & \textbf{1.17 (12.50\%)} & \textbf{1.17 (12.50\%)} & \textbf{1.17 (12.50\%)} \\ \hline
num\_cc & 1.00 & 1.00 (0.00\%) & 1.00 (0.00\%) & 1.00 (0.00\%) & 1.00 (0.00\%) \\ \hline
largest\_cc & 4.00 & 4.00 (0.00\%) & 4.00 (0.00\%) & 4.00 (0.00\%) & 4.00 (0.00\%) \\ \hline
largest\_cc\_radius & 1.00 & 1.00 (0.00\%) & 1.00 (0.00\%) & 1.00 (0.00\%) & 1.00 (0.00\%) \\ \hline
avg\_path\_len & 1.00 & \textbf{1.22 (22.22\%)} & \textbf{1.17 (16.67\%)} & \textbf{1.17 (16.67\%)} & \textbf{1.17 (16.67\%)} \\ \hline
largest\_clique\_size & 4.00 & \textbf{3.00 (25.00\%)} & 4.00 (0.00\%) & 4.00 (0.00\%) & 4.00 (0.00\%) \\ \hline
reciprocity & 1.00 & \textbf{0.57 (42.86\%)} & \textbf{0.80 (20.00\%)} & \textbf{0.80 (20.00\%)} & \textbf{0.80 (20.00\%)} \\ \hline
\end{tabular}
}
\end{table}

\begin{table}
\centering
\caption{Instancja uniform\_11 - porównanie wartości miar uzyskanych z próbkowania dwufazowego z wartościami z przeglądu zupełnego.}
\label{tab:comp_uniform_11_twophase}
\footnotesize
\resizebox{\textwidth}{!}{
\begin{tabular}{|l|l|l|l|l|l|}
\hline
\textbf{metryka} & \textbf{p. zupełny} & \textbf{dwufazowy\_10} & \textbf{dwufazowy\_100} & \textbf{dwufazowy\_1000} & \textbf{dwufazowy\_10000} \\ \hline
opt\_count & 3628800.00 & 116.00 & 10013.00 & 1000010.00 & 100000006.00 \\ \hline
oracle\_count & 979464676.00 & 38001.00 & 4086571.00 & 416049316.00 & 41687797175.00 \\ \hline
node\_count & 4.00 & 4.00 (0.00\%) & 4.00 (0.00\%) & 4.00 (0.00\%) & 4.00 (0.00\%) \\ \hline
edge\_count & 16.00 & \textbf{14.00 (12.50\%)} & \textbf{15.00 (6.25\%)} & 16.00 (0.00\%) & 16.00 (0.00\%) \\ \hline
num\_subsinks & 2.00 & 2.00 (0.00\%) & 2.00 (0.00\%) & 2.00 (0.00\%) & 2.00 (0.00\%) \\ \hline
edge\_to\_node & 4.00 & \textbf{3.50 (12.50\%)} & \textbf{3.75 (6.25\%)} & 4.00 (0.00\%) & 4.00 (0.00\%) \\ \hline
avg\_fitness & 2632.50 & 2632.50 (0.00\%) & 2632.50 (0.00\%) & 2632.50 (0.00\%) & 2632.50 (0.00\%) \\ \hline
distLO & 0.00 & \textbf{0.26 (9820.37\%)} & \textbf{0.03 (1125.95\%)} & \textbf{0.00 (63.37\%)} & \textbf{0.00 (81.38\%)} \\ \hline
conrel & 3.00 & 3.00 (0.00\%) & 3.00 (0.00\%) & 3.00 (0.00\%) & 3.00 (0.00\%) \\ \hline
avg\_loop\_weight & 738.25 & \textbf{4.50 (99.39\%)} & \textbf{48.50 (93.43\%)} & \textbf{451.50 (38.84\%)} & \textbf{4685.00 (534.61\%)} \\ \hline
go\_path\_ratio & 0.75 & 0.75 (0.00\%) & 0.75 (0.00\%) & 0.75 (0.00\%) & 0.75 (0.00\%) \\ \hline
avg\_go\_path\_len & 1.00 & 1.00 (0.00\%) & 1.00 (0.00\%) & 1.00 (0.00\%) & 1.00 (0.00\%) \\ \hline
max\_go\_path\_len & 1.00 & 1.00 (0.00\%) & 1.00 (0.00\%) & 1.00 (0.00\%) & 1.00 (0.00\%) \\ \hline
num\_sinks & 2.00 & 2.00 (0.00\%) & 2.00 (0.00\%) & 2.00 (0.00\%) & 2.00 (0.00\%) \\ \hline
num\_sources & 2.00 & 2.00 (0.00\%) & 2.00 (0.00\%) & 2.00 (0.00\%) & 2.00 (0.00\%) \\ \hline
funnel\_num & 2.00 & 2.00 (0.00\%) & 2.00 (0.00\%) & 2.00 (0.00\%) & 2.00 (0.00\%) \\ \hline
mean\_funnel\_size & 3.00 & 3.00 (0.00\%) & 3.00 (0.00\%) & 3.00 (0.00\%) & 3.00 (0.00\%) \\ \hline
max\_funnel\_size & 3.00 & 3.00 (0.00\%) & 3.00 (0.00\%) & 3.00 (0.00\%) & 3.00 (0.00\%) \\ \hline
max\_out\_degree & 3.00 & 3.00 (0.00\%) & 3.00 (0.00\%) & 3.00 (0.00\%) & 3.00 (0.00\%) \\ \hline
avg\_out\_degree & 3.00 & \textbf{2.50 (16.67\%)} & \textbf{2.75 (8.33\%)} & 3.00 (0.00\%) & 3.00 (0.00\%) \\ \hline
max\_in\_degree & 3.00 & 3.00 (0.00\%) & 3.00 (0.00\%) & 3.00 (0.00\%) & 3.00 (0.00\%) \\ \hline
avg\_in\_degree & 3.00 & \textbf{2.50 (16.67\%)} & \textbf{2.75 (8.33\%)} & 3.00 (0.00\%) & 3.00 (0.00\%) \\ \hline
density & 1.33 & \textbf{1.17 (12.50\%)} & \textbf{1.25 (6.25\%)} & 1.33 (0.00\%) & 1.33 (0.00\%) \\ \hline
num\_cc & 1.00 & 1.00 (0.00\%) & 1.00 (0.00\%) & 1.00 (0.00\%) & 1.00 (0.00\%) \\ \hline
largest\_cc & 4.00 & 4.00 (0.00\%) & 4.00 (0.00\%) & 4.00 (0.00\%) & 4.00 (0.00\%) \\ \hline
largest\_cc\_radius & 1.00 & 1.00 (0.00\%) & 1.00 (0.00\%) & 1.00 (0.00\%) & 1.00 (0.00\%) \\ \hline
avg\_path\_len & 1.00 & \textbf{1.17 (16.67\%)} & \textbf{1.08 (8.33\%)} & 1.00 (0.00\%) & 1.00 (0.00\%) \\ \hline
largest\_clique\_size & 4.00 & 4.00 (0.00\%) & 4.00 (0.00\%) & 4.00 (0.00\%) & 4.00 (0.00\%) \\ \hline
reciprocity & 1.00 & \textbf{0.80 (20.00\%)} & \textbf{0.91 (9.09\%)} & 1.00 (0.00\%) & 1.00 (0.00\%) \\ \hline
\end{tabular}
}
\end{table}

\begin{table}
\centering
\caption{Instancja cliques\_7 - porównanie wartości miar uzyskanych z próbkowania snowball z wartościami z przeglądu zupełnego.}
\label{tab:comp_cliques_7_snowball}
\footnotesize
\resizebox{\textwidth}{!}{
\begin{tabular}{|l|l|l|l|l|l|}
\hline
\textbf{metryka} & \textbf{p. zupełny} & \textbf{snowball\_1} & \textbf{snowball\_10} & \textbf{snowball\_100} & \textbf{snowball\_1000} \\ \hline
opt\_count & 720.00 & 301.00 & 1210.00 & 10300.00 & 101200.00 \\ \hline
oracle\_count & 46570.00 & 19467.00 & 79989.00 & 681198.00 & 6671805.00 \\ \hline
node\_count & 2.00 & 2.00 (0.00\%) & 2.00 (0.00\%) & 2.00 (0.00\%) & 2.00 (0.00\%) \\ \hline
edge\_count & 4.00 & 4.00 (0.00\%) & 4.00 (0.00\%) & 4.00 (0.00\%) & 4.00 (0.00\%) \\ \hline
num\_subsinks & 2.00 & 2.00 (0.00\%) & 2.00 (0.00\%) & 2.00 (0.00\%) & 2.00 (0.00\%) \\ \hline
edge\_to\_node & 2.00 & 2.00 (0.00\%) & 2.00 (0.00\%) & 2.00 (0.00\%) & 2.00 (0.00\%) \\ \hline
avg\_fitness & 2401.00 & 2401.00 (0.00\%) & 2401.00 (0.00\%) & 2401.00 (0.00\%) & 2401.00 (0.00\%) \\ \hline
distLO & 0.02 & \textbf{0.03 (33.15\%)} & \textbf{0.00 (78.81\%)} & \textbf{0.00 (96.68\%)} & \textbf{0.00 (99.70\%)} \\ \hline
conrel & 1.00 & 1.00 (0.00\%) & 1.00 (0.00\%) & 1.00 (0.00\%) & 1.00 (0.00\%) \\ \hline
avg\_loop\_weight & 100.00 & \textbf{121.50 (21.50\%)} & \textbf{489.00 (389.00\%)} & \textbf{4266.00 (4166.00\%)} & \textbf{41638.50 (41538.50\%)} \\ \hline
go\_path\_ratio & 0.50 & 0.50 (0.00\%) & 0.50 (0.00\%) & 0.50 (0.00\%) & 0.50 (0.00\%) \\ \hline
avg\_go\_path\_len & 1.00 & 1.00 (0.00\%) & 1.00 (0.00\%) & 1.00 (0.00\%) & 1.00 (0.00\%) \\ \hline
max\_go\_path\_len & 1.00 & 1.00 (0.00\%) & 1.00 (0.00\%) & 1.00 (0.00\%) & 1.00 (0.00\%) \\ \hline
num\_sinks & 2.00 & 2.00 (0.00\%) & 2.00 (0.00\%) & 2.00 (0.00\%) & 2.00 (0.00\%) \\ \hline
num\_sources & 2.00 & 2.00 (0.00\%) & 2.00 (0.00\%) & 2.00 (0.00\%) & 2.00 (0.00\%) \\ \hline
funnel\_num & 2.00 & 2.00 (0.00\%) & 2.00 (0.00\%) & 2.00 (0.00\%) & 2.00 (0.00\%) \\ \hline
mean\_funnel\_size & 1.00 & 1.00 (0.00\%) & 1.00 (0.00\%) & 1.00 (0.00\%) & 1.00 (0.00\%) \\ \hline
max\_funnel\_size & 1.00 & 1.00 (0.00\%) & 1.00 (0.00\%) & 1.00 (0.00\%) & 1.00 (0.00\%) \\ \hline
max\_out\_degree & 1.00 & 1.00 (0.00\%) & 1.00 (0.00\%) & 1.00 (0.00\%) & 1.00 (0.00\%) \\ \hline
avg\_out\_degree & 1.00 & 1.00 (0.00\%) & 1.00 (0.00\%) & 1.00 (0.00\%) & 1.00 (0.00\%) \\ \hline
max\_in\_degree & 1.00 & 1.00 (0.00\%) & 1.00 (0.00\%) & 1.00 (0.00\%) & 1.00 (0.00\%) \\ \hline
avg\_in\_degree & 1.00 & 1.00 (0.00\%) & 1.00 (0.00\%) & 1.00 (0.00\%) & 1.00 (0.00\%) \\ \hline
density & 2.00 & 2.00 (0.00\%) & 2.00 (0.00\%) & 2.00 (0.00\%) & 2.00 (0.00\%) \\ \hline
num\_cc & 1.00 & 1.00 (0.00\%) & 1.00 (0.00\%) & 1.00 (0.00\%) & 1.00 (0.00\%) \\ \hline
largest\_cc & 2.00 & 2.00 (0.00\%) & 2.00 (0.00\%) & 2.00 (0.00\%) & 2.00 (0.00\%) \\ \hline
largest\_cc\_radius & 1.00 & 1.00 (0.00\%) & 1.00 (0.00\%) & 1.00 (0.00\%) & 1.00 (0.00\%) \\ \hline
avg\_path\_len & 1.00 & 1.00 (0.00\%) & 1.00 (0.00\%) & 1.00 (0.00\%) & 1.00 (0.00\%) \\ \hline
largest\_clique\_size & 2.00 & 2.00 (0.00\%) & 2.00 (0.00\%) & 2.00 (0.00\%) & 2.00 (0.00\%) \\ \hline
reciprocity & 1.00 & 1.00 (0.00\%) & 1.00 (0.00\%) & 1.00 (0.00\%) & 1.00 (0.00\%) \\ \hline
\end{tabular}
}
\end{table}

\begin{table}
\centering
\caption{Instancja cliques\_7 - porównanie wartości miar uzyskanych z próbkowania dwufazowego z wartościami z przeglądu zupełnego.}
\label{tab:comp_cliques_7_twophase}
\footnotesize
\resizebox{\textwidth}{!}{
\begin{tabular}{|l|l|l|l|l|l|}
\hline
\textbf{metryka} & \textbf{p. zupełny} & \textbf{dwufazowy\_10} & \textbf{dwufazowy\_100} & \textbf{dwufazowy\_1000} & \textbf{dwufazowy\_10000} \\ \hline
opt\_count & 720.00 & 102.00 & 10004.00 & 1000003.00 & 100000002.00 \\ \hline
oracle\_count & 46570.00 & 8360.00 & 891012.00 & 89532059.00 & 8961919004.00 \\ \hline
node\_count & 2.00 & 2.00 (0.00\%) & 2.00 (0.00\%) & 2.00 (0.00\%) & 2.00 (0.00\%) \\ \hline
edge\_count & 4.00 & 4.00 (0.00\%) & 4.00 (0.00\%) & 4.00 (0.00\%) & 4.00 (0.00\%) \\ \hline
num\_subsinks & 2.00 & 2.00 (0.00\%) & 2.00 (0.00\%) & 2.00 (0.00\%) & 2.00 (0.00\%) \\ \hline
edge\_to\_node & 2.00 & 2.00 (0.00\%) & 2.00 (0.00\%) & 2.00 (0.00\%) & 2.00 (0.00\%) \\ \hline
avg\_fitness & 2401.00 & 2401.00 (0.00\%) & 2401.00 (0.00\%) & 2401.00 (0.00\%) & 2401.00 (0.00\%) \\ \hline
distLO & 0.02 & \textbf{0.60 (2486.90\%)} & \textbf{0.07 (187.43\%)} & \textbf{0.00 (82.76\%)} & \textbf{0.00 (99.21\%)} \\ \hline
conrel & 1.00 & 1.00 (0.00\%) & 1.00 (0.00\%) & 1.00 (0.00\%) & 1.00 (0.00\%) \\ \hline
avg\_loop\_weight & 100.00 & \textbf{7.00 (93.00\%)} & \textbf{68.00 (32.00\%)} & \textbf{713.50 (613.50\%)} & \textbf{7138.00 (7038.00\%)} \\ \hline
go\_path\_ratio & 0.50 & 0.50 (0.00\%) & 0.50 (0.00\%) & 0.50 (0.00\%) & 0.50 (0.00\%) \\ \hline
avg\_go\_path\_len & 1.00 & 1.00 (0.00\%) & 1.00 (0.00\%) & 1.00 (0.00\%) & 1.00 (0.00\%) \\ \hline
max\_go\_path\_len & 1.00 & 1.00 (0.00\%) & 1.00 (0.00\%) & 1.00 (0.00\%) & 1.00 (0.00\%) \\ \hline
num\_sinks & 2.00 & 2.00 (0.00\%) & 2.00 (0.00\%) & 2.00 (0.00\%) & 2.00 (0.00\%) \\ \hline
num\_sources & 2.00 & 2.00 (0.00\%) & 2.00 (0.00\%) & 2.00 (0.00\%) & 2.00 (0.00\%) \\ \hline
funnel\_num & 2.00 & 2.00 (0.00\%) & 2.00 (0.00\%) & 2.00 (0.00\%) & 2.00 (0.00\%) \\ \hline
mean\_funnel\_size & 1.00 & 1.00 (0.00\%) & 1.00 (0.00\%) & 1.00 (0.00\%) & 1.00 (0.00\%) \\ \hline
max\_funnel\_size & 1.00 & 1.00 (0.00\%) & 1.00 (0.00\%) & 1.00 (0.00\%) & 1.00 (0.00\%) \\ \hline
max\_out\_degree & 1.00 & 1.00 (0.00\%) & 1.00 (0.00\%) & 1.00 (0.00\%) & 1.00 (0.00\%) \\ \hline
avg\_out\_degree & 1.00 & 1.00 (0.00\%) & 1.00 (0.00\%) & 1.00 (0.00\%) & 1.00 (0.00\%) \\ \hline
max\_in\_degree & 1.00 & 1.00 (0.00\%) & 1.00 (0.00\%) & 1.00 (0.00\%) & 1.00 (0.00\%) \\ \hline
avg\_in\_degree & 1.00 & 1.00 (0.00\%) & 1.00 (0.00\%) & 1.00 (0.00\%) & 1.00 (0.00\%) \\ \hline
density & 2.00 & 2.00 (0.00\%) & 2.00 (0.00\%) & 2.00 (0.00\%) & 2.00 (0.00\%) \\ \hline
num\_cc & 1.00 & 1.00 (0.00\%) & 1.00 (0.00\%) & 1.00 (0.00\%) & 1.00 (0.00\%) \\ \hline
largest\_cc & 2.00 & 2.00 (0.00\%) & 2.00 (0.00\%) & 2.00 (0.00\%) & 2.00 (0.00\%) \\ \hline
largest\_cc\_radius & 1.00 & 1.00 (0.00\%) & 1.00 (0.00\%) & 1.00 (0.00\%) & 1.00 (0.00\%) \\ \hline
avg\_path\_len & 1.00 & 1.00 (0.00\%) & 1.00 (0.00\%) & 1.00 (0.00\%) & 1.00 (0.00\%) \\ \hline
largest\_clique\_size & 2.00 & 2.00 (0.00\%) & 2.00 (0.00\%) & 2.00 (0.00\%) & 2.00 (0.00\%) \\ \hline
reciprocity & 1.00 & 1.00 (0.00\%) & 1.00 (0.00\%) & 1.00 (0.00\%) & 1.00 (0.00\%) \\ \hline
\end{tabular}
}
\end{table}

\begin{table}
\centering
\caption{Instancja cliques\_8 - porównanie wartości miar uzyskanych z próbkowania snowball z wartościami z przeglądu zupełnego.}
\label{tab:comp_cliques_8_snowball}
\footnotesize
\resizebox{\textwidth}{!}{
\begin{tabular}{|l|l|l|l|l|l|}
\hline
\textbf{metryka} & \textbf{p. zupełny} & \textbf{snowball\_1} & \textbf{snowball\_10} & \textbf{snowball\_100} & \textbf{snowball\_1000} \\ \hline
opt\_count & 5040.00 & 501.00 & 1508.00 & 10598.00 & 101498.00 \\ \hline
oracle\_count & 419679.00 & 42252.00 & 127932.00 & 895076.00 & 8632344.00 \\ \hline
node\_count & 4.00 & 4.00 (0.00\%) & 4.00 (0.00\%) & 4.00 (0.00\%) & 4.00 (0.00\%) \\ \hline
edge\_count & 16.00 & \textbf{15.00 (6.25\%)} & 16.00 (0.00\%) & 16.00 (0.00\%) & 16.00 (0.00\%) \\ \hline
num\_subsinks & 2.00 & 2.00 (0.00\%) & 2.00 (0.00\%) & 2.00 (0.00\%) & 2.00 (0.00\%) \\ \hline
edge\_to\_node & 4.00 & \textbf{3.75 (6.25\%)} & 4.00 (0.00\%) & 4.00 (0.00\%) & 4.00 (0.00\%) \\ \hline
avg\_fitness & 1718.00 & 1718.00 (0.00\%) & 1718.00 (0.00\%) & 1718.00 (0.00\%) & 1718.00 (0.00\%) \\ \hline
distLO & 0.01 & \textbf{0.06 (388.30\%)} & \textbf{0.01 (14.48\%)} & \textbf{0.01 (56.23\%)} & \textbf{0.00 (97.32\%)} \\ \hline
conrel & 3.00 & 3.00 (0.00\%) & 3.00 (0.00\%) & 3.00 (0.00\%) & 3.00 (0.00\%) \\ \hline
avg\_loop\_weight & 141.75 & \textbf{82.25 (41.98\%)} & \textbf{247.00 (74.25\%)} & \textbf{1991.75 (1305.11\%)} & \textbf{19076.00 (13357.50\%)} \\ \hline
go\_path\_ratio & 0.75 & 0.75 (0.00\%) & 0.75 (0.00\%) & 0.75 (0.00\%) & 0.75 (0.00\%) \\ \hline
avg\_go\_path\_len & 1.00 & 1.00 (0.00\%) & 1.00 (0.00\%) & 1.00 (0.00\%) & 1.00 (0.00\%) \\ \hline
max\_go\_path\_len & 1.00 & 1.00 (0.00\%) & 1.00 (0.00\%) & 1.00 (0.00\%) & 1.00 (0.00\%) \\ \hline
num\_sinks & 2.00 & 2.00 (0.00\%) & 2.00 (0.00\%) & 2.00 (0.00\%) & 2.00 (0.00\%) \\ \hline
num\_sources & 2.00 & 2.00 (0.00\%) & 2.00 (0.00\%) & 2.00 (0.00\%) & 2.00 (0.00\%) \\ \hline
funnel\_num & 2.00 & 2.00 (0.00\%) & 2.00 (0.00\%) & 2.00 (0.00\%) & 2.00 (0.00\%) \\ \hline
mean\_funnel\_size & 3.00 & 3.00 (0.00\%) & 3.00 (0.00\%) & 3.00 (0.00\%) & 3.00 (0.00\%) \\ \hline
max\_funnel\_size & 3.00 & 3.00 (0.00\%) & 3.00 (0.00\%) & 3.00 (0.00\%) & 3.00 (0.00\%) \\ \hline
max\_out\_degree & 3.00 & 3.00 (0.00\%) & 3.00 (0.00\%) & 3.00 (0.00\%) & 3.00 (0.00\%) \\ \hline
avg\_out\_degree & 3.00 & \textbf{2.75 (8.33\%)} & 3.00 (0.00\%) & 3.00 (0.00\%) & 3.00 (0.00\%) \\ \hline
max\_in\_degree & 3.00 & 3.00 (0.00\%) & 3.00 (0.00\%) & 3.00 (0.00\%) & 3.00 (0.00\%) \\ \hline
avg\_in\_degree & 3.00 & \textbf{2.75 (8.33\%)} & 3.00 (0.00\%) & 3.00 (0.00\%) & 3.00 (0.00\%) \\ \hline
density & 1.33 & \textbf{1.25 (6.25\%)} & 1.33 (0.00\%) & 1.33 (0.00\%) & 1.33 (0.00\%) \\ \hline
num\_cc & 1.00 & 1.00 (0.00\%) & 1.00 (0.00\%) & 1.00 (0.00\%) & 1.00 (0.00\%) \\ \hline
largest\_cc & 4.00 & 4.00 (0.00\%) & 4.00 (0.00\%) & 4.00 (0.00\%) & 4.00 (0.00\%) \\ \hline
largest\_cc\_radius & 1.00 & 1.00 (0.00\%) & 1.00 (0.00\%) & 1.00 (0.00\%) & 1.00 (0.00\%) \\ \hline
avg\_path\_len & 1.00 & \textbf{1.08 (8.33\%)} & 1.00 (0.00\%) & 1.00 (0.00\%) & 1.00 (0.00\%) \\ \hline
largest\_clique\_size & 4.00 & 4.00 (0.00\%) & 4.00 (0.00\%) & 4.00 (0.00\%) & 4.00 (0.00\%) \\ \hline
reciprocity & 1.00 & \textbf{0.91 (9.09\%)} & 1.00 (0.00\%) & 1.00 (0.00\%) & 1.00 (0.00\%) \\ \hline
\end{tabular}
}
\end{table}

\begin{table}
\centering
\caption{Instancja cliques\_8 - porównanie wartości miar uzyskanych z próbkowania dwufazowego z wartościami z przeglądu zupełnego.}
\label{tab:comp_cliques_8_twophase}
\footnotesize
\resizebox{\textwidth}{!}{
\begin{tabular}{|l|l|l|l|l|l|}
\hline
\textbf{metryka} & \textbf{p. zupełny} & \textbf{dwufazowy\_10} & \textbf{dwufazowy\_100} & \textbf{dwufazowy\_1000} & \textbf{dwufazowy\_10000} \\ \hline
opt\_count & 5040.00 & 110.00 & 10008.00 & 1000011.00 & 100000011.00 \\ \hline
oracle\_count & 419679.00 & 12004.00 & 1309047.00 & 133434867.00 & 13373288761.00 \\ \hline
node\_count & 4.00 & 4.00 (0.00\%) & 4.00 (0.00\%) & 4.00 (0.00\%) & 4.00 (0.00\%) \\ \hline
edge\_count & 16.00 & \textbf{10.00 (37.50\%)} & 16.00 (0.00\%) & 16.00 (0.00\%) & 16.00 (0.00\%) \\ \hline
num\_subsinks & 2.00 & 2.00 (0.00\%) & 2.00 (0.00\%) & 2.00 (0.00\%) & 2.00 (0.00\%) \\ \hline
edge\_to\_node & 4.00 & \textbf{2.50 (37.50\%)} & 4.00 (0.00\%) & 4.00 (0.00\%) & 4.00 (0.00\%) \\ \hline
avg\_fitness & 1718.00 & 1718.00 (0.00\%) & 1718.00 (0.00\%) & 1718.00 (0.00\%) & 1718.00 (0.00\%) \\ \hline
distLO & 0.01 & \textbf{0.59 (4534.43\%)} & \textbf{0.03 (124.40\%)} & \textbf{0.00 (63.32\%)} & \textbf{0.00 (96.45\%)} \\ \hline
conrel & 3.00 & 3.00 (0.00\%) & 3.00 (0.00\%) & 3.00 (0.00\%) & 3.00 (0.00\%) \\ \hline
avg\_loop\_weight & 141.75 & \textbf{6.00 (95.77\%)} & \textbf{41.25 (70.90\%)} & \textbf{454.50 (220.63\%)} & \textbf{4537.50 (3101.06\%)} \\ \hline
go\_path\_ratio & 0.75 & 0.75 (0.00\%) & 0.75 (0.00\%) & 0.75 (0.00\%) & 0.75 (0.00\%) \\ \hline
avg\_go\_path\_len & 1.00 & 1.00 (0.00\%) & 1.00 (0.00\%) & 1.00 (0.00\%) & 1.00 (0.00\%) \\ \hline
max\_go\_path\_len & 1.00 & 1.00 (0.00\%) & 1.00 (0.00\%) & 1.00 (0.00\%) & 1.00 (0.00\%) \\ \hline
num\_sinks & 2.00 & 2.00 (0.00\%) & 2.00 (0.00\%) & 2.00 (0.00\%) & 2.00 (0.00\%) \\ \hline
num\_sources & 2.00 & 2.00 (0.00\%) & 2.00 (0.00\%) & 2.00 (0.00\%) & 2.00 (0.00\%) \\ \hline
funnel\_num & 2.00 & 2.00 (0.00\%) & 2.00 (0.00\%) & 2.00 (0.00\%) & 2.00 (0.00\%) \\ \hline
mean\_funnel\_size & 3.00 & \textbf{2.50 (16.67\%)} & 3.00 (0.00\%) & 3.00 (0.00\%) & 3.00 (0.00\%) \\ \hline
max\_funnel\_size & 3.00 & 3.00 (0.00\%) & 3.00 (0.00\%) & 3.00 (0.00\%) & 3.00 (0.00\%) \\ \hline
max\_out\_degree & 3.00 & 3.00 (0.00\%) & 3.00 (0.00\%) & 3.00 (0.00\%) & 3.00 (0.00\%) \\ \hline
avg\_out\_degree & 3.00 & \textbf{1.50 (50.00\%)} & 3.00 (0.00\%) & 3.00 (0.00\%) & 3.00 (0.00\%) \\ \hline
max\_in\_degree & 3.00 & 3.00 (0.00\%) & 3.00 (0.00\%) & 3.00 (0.00\%) & 3.00 (0.00\%) \\ \hline
avg\_in\_degree & 3.00 & \textbf{1.50 (50.00\%)} & 3.00 (0.00\%) & 3.00 (0.00\%) & 3.00 (0.00\%) \\ \hline
density & 1.33 & \textbf{0.83 (37.50\%)} & 1.33 (0.00\%) & 1.33 (0.00\%) & 1.33 (0.00\%) \\ \hline
num\_cc & 1.00 & 1.00 (0.00\%) & 1.00 (0.00\%) & 1.00 (0.00\%) & 1.00 (0.00\%) \\ \hline
largest\_cc & 4.00 & 4.00 (0.00\%) & 4.00 (0.00\%) & 4.00 (0.00\%) & 4.00 (0.00\%) \\ \hline
largest\_cc\_radius & 1.00 & 1.00 (0.00\%) & 1.00 (0.00\%) & 1.00 (0.00\%) & 1.00 (0.00\%) \\ \hline
avg\_path\_len & 1.00 & \textbf{1.14 (14.29\%)} & 1.00 (0.00\%) & 1.00 (0.00\%) & 1.00 (0.00\%) \\ \hline
largest\_clique\_size & 4.00 & \textbf{3.00 (25.00\%)} & 4.00 (0.00\%) & 4.00 (0.00\%) & 4.00 (0.00\%) \\ \hline
reciprocity & 1.00 & \textbf{0.33 (66.67\%)} & 1.00 (0.00\%) & 1.00 (0.00\%) & 1.00 (0.00\%) \\ \hline
\end{tabular}
}
\end{table}

\begin{table}
\centering
\caption{Instancja cliques\_9 - porównanie wartości miar uzyskanych z próbkowania snowball z wartościami z przeglądu zupełnego.}
\label{tab:comp_cliques_9_snowball}
\footnotesize
\resizebox{\textwidth}{!}{
\begin{tabular}{|l|l|l|l|l|l|}
\hline
\textbf{metryka} & \textbf{p. zupełny} & \textbf{snowball\_1} & \textbf{snowball\_10} & \textbf{snowball\_100} & \textbf{snowball\_1000} \\ \hline
opt\_count & 40320.00 & 401.00 & 1310.00 & 10300.00 & 101200.00 \\ \hline
oracle\_count & 6306952.00 & 47880.00 & 156960.00 & 1235592.00 & 12167676.00 \\ \hline
node\_count & 2.00 & 2.00 (0.00\%) & 2.00 (0.00\%) & 2.00 (0.00\%) & 2.00 (0.00\%) \\ \hline
edge\_count & 4.00 & 4.00 (0.00\%) & 4.00 (0.00\%) & 4.00 (0.00\%) & 4.00 (0.00\%) \\ \hline
num\_subsinks & 2.00 & 2.00 (0.00\%) & 2.00 (0.00\%) & 2.00 (0.00\%) & 2.00 (0.00\%) \\ \hline
edge\_to\_node & 2.00 & 2.00 (0.00\%) & 2.00 (0.00\%) & 2.00 (0.00\%) & 2.00 (0.00\%) \\ \hline
avg\_fitness & 1458.00 & 1458.00 (0.00\%) & 1458.00 (0.00\%) & 1458.00 (0.00\%) & 1458.00 (0.00\%) \\ \hline
distLO & 0.00 & \textbf{0.03 (900.91\%)} & \textbf{0.01 (194.47\%)} & \textbf{0.00 (61.40\%)} & \textbf{0.00 (95.67\%)} \\ \hline
conrel & 1.00 & 1.00 (0.00\%) & 1.00 (0.00\%) & 1.00 (0.00\%) & 1.00 (0.00\%) \\ \hline
avg\_loop\_weight & 388.00 & \textbf{181.00 (53.35\%)} & \textbf{589.00 (51.80\%)} & \textbf{4638.50 (1095.49\%)} & \textbf{45549.50 (11639.56\%)} \\ \hline
go\_path\_ratio & 0.50 & 0.50 (0.00\%) & 0.50 (0.00\%) & 0.50 (0.00\%) & 0.50 (0.00\%) \\ \hline
avg\_go\_path\_len & 1.00 & 1.00 (0.00\%) & 1.00 (0.00\%) & 1.00 (0.00\%) & 1.00 (0.00\%) \\ \hline
max\_go\_path\_len & 1.00 & 1.00 (0.00\%) & 1.00 (0.00\%) & 1.00 (0.00\%) & 1.00 (0.00\%) \\ \hline
num\_sinks & 2.00 & 2.00 (0.00\%) & 2.00 (0.00\%) & 2.00 (0.00\%) & 2.00 (0.00\%) \\ \hline
num\_sources & 2.00 & 2.00 (0.00\%) & 2.00 (0.00\%) & 2.00 (0.00\%) & 2.00 (0.00\%) \\ \hline
funnel\_num & 2.00 & 2.00 (0.00\%) & 2.00 (0.00\%) & 2.00 (0.00\%) & 2.00 (0.00\%) \\ \hline
mean\_funnel\_size & 1.00 & 1.00 (0.00\%) & 1.00 (0.00\%) & 1.00 (0.00\%) & 1.00 (0.00\%) \\ \hline
max\_funnel\_size & 1.00 & 1.00 (0.00\%) & 1.00 (0.00\%) & 1.00 (0.00\%) & 1.00 (0.00\%) \\ \hline
max\_out\_degree & 1.00 & 1.00 (0.00\%) & 1.00 (0.00\%) & 1.00 (0.00\%) & 1.00 (0.00\%) \\ \hline
avg\_out\_degree & 1.00 & 1.00 (0.00\%) & 1.00 (0.00\%) & 1.00 (0.00\%) & 1.00 (0.00\%) \\ \hline
max\_in\_degree & 1.00 & 1.00 (0.00\%) & 1.00 (0.00\%) & 1.00 (0.00\%) & 1.00 (0.00\%) \\ \hline
avg\_in\_degree & 1.00 & 1.00 (0.00\%) & 1.00 (0.00\%) & 1.00 (0.00\%) & 1.00 (0.00\%) \\ \hline
density & 2.00 & 2.00 (0.00\%) & 2.00 (0.00\%) & 2.00 (0.00\%) & 2.00 (0.00\%) \\ \hline
num\_cc & 1.00 & 1.00 (0.00\%) & 1.00 (0.00\%) & 1.00 (0.00\%) & 1.00 (0.00\%) \\ \hline
largest\_cc & 2.00 & 2.00 (0.00\%) & 2.00 (0.00\%) & 2.00 (0.00\%) & 2.00 (0.00\%) \\ \hline
largest\_cc\_radius & 1.00 & 1.00 (0.00\%) & 1.00 (0.00\%) & 1.00 (0.00\%) & 1.00 (0.00\%) \\ \hline
avg\_path\_len & 1.00 & 1.00 (0.00\%) & 1.00 (0.00\%) & 1.00 (0.00\%) & 1.00 (0.00\%) \\ \hline
largest\_clique\_size & 2.00 & 2.00 (0.00\%) & 2.00 (0.00\%) & 2.00 (0.00\%) & 2.00 (0.00\%) \\ \hline
reciprocity & 1.00 & 1.00 (0.00\%) & 1.00 (0.00\%) & 1.00 (0.00\%) & 1.00 (0.00\%) \\ \hline
\end{tabular}
}
\end{table}

\begin{table}
\centering
\caption{Instancja cliques\_9 - porównanie wartości miar uzyskanych z próbkowania dwufazowego z wartościami z przeglądu zupełnego.}
\label{tab:comp_cliques_9_twophase}
\footnotesize
\resizebox{\textwidth}{!}{
\begin{tabular}{|l|l|l|l|l|l|}
\hline
\textbf{metryka} & \textbf{p. zupełny} & \textbf{dwufazowy\_10} & \textbf{dwufazowy\_100} & \textbf{dwufazowy\_1000} & \textbf{dwufazowy\_10000} \\ \hline
opt\_count & 40320.00 & 102.00 & 10002.00 & 1000003.00 & 100000002.00 \\ \hline
oracle\_count & 6306952.00 & 21917.00 & 2274360.00 & 229637590.00 & 22990985284.00 \\ \hline
node\_count & 2.00 & 2.00 (0.00\%) & 2.00 (0.00\%) & 2.00 (0.00\%) & 2.00 (0.00\%) \\ \hline
edge\_count & 4.00 & \textbf{3.00 (25.00\%)} & \textbf{3.00 (25.00\%)} & 4.00 (0.00\%) & 4.00 (0.00\%) \\ \hline
num\_subsinks & 2.00 & 2.00 (0.00\%) & 2.00 (0.00\%) & 2.00 (0.00\%) & 2.00 (0.00\%) \\ \hline
edge\_to\_node & 2.00 & \textbf{1.50 (25.00\%)} & \textbf{1.50 (25.00\%)} & 2.00 (0.00\%) & 2.00 (0.00\%) \\ \hline
avg\_fitness & 1458.00 & 1458.00 (0.00\%) & 1458.00 (0.00\%) & 1458.00 (0.00\%) & 1458.00 (0.00\%) \\ \hline
distLO & 0.00 & \textbf{0.12 (3673.92\%)} & \textbf{0.03 (676.98\%)} & \textbf{0.01 (252.29\%)} & \textbf{0.00 (54.87\%)} \\ \hline
conrel & 1.00 & 1.00 (0.00\%) & 1.00 (0.00\%) & 1.00 (0.00\%) & 1.00 (0.00\%) \\ \hline
avg\_loop\_weight & 388.00 & \textbf{6.50 (98.32\%)} & \textbf{70.00 (81.96\%)} & \textbf{696.50 (79.51\%)} & \textbf{6983.00 (1699.74\%)} \\ \hline
go\_path\_ratio & 0.50 & 0.50 (0.00\%) & \textbf{0.00 (100.00\%)} & 0.50 (0.00\%) & 0.50 (0.00\%) \\ \hline
avg\_go\_path\_len & 1.00 & 1.00 (0.00\%) & \textbf{nan (nan\%)} & 1.00 (0.00\%) & 1.00 (0.00\%) \\ \hline
max\_go\_path\_len & 1.00 & 1.00 (0.00\%) & \textbf{0.00 (100.00\%)} & 1.00 (0.00\%) & 1.00 (0.00\%) \\ \hline
num\_sinks & 2.00 & 2.00 (0.00\%) & 2.00 (0.00\%) & 2.00 (0.00\%) & 2.00 (0.00\%) \\ \hline
num\_sources & 2.00 & 2.00 (0.00\%) & 2.00 (0.00\%) & 2.00 (0.00\%) & 2.00 (0.00\%) \\ \hline
funnel\_num & 2.00 & 2.00 (0.00\%) & 2.00 (0.00\%) & 2.00 (0.00\%) & 2.00 (0.00\%) \\ \hline
mean\_funnel\_size & 1.00 & 1.00 (0.00\%) & 1.00 (0.00\%) & 1.00 (0.00\%) & 1.00 (0.00\%) \\ \hline
max\_funnel\_size & 1.00 & 1.00 (0.00\%) & 1.00 (0.00\%) & 1.00 (0.00\%) & 1.00 (0.00\%) \\ \hline
max\_out\_degree & 1.00 & 1.00 (0.00\%) & 1.00 (0.00\%) & 1.00 (0.00\%) & 1.00 (0.00\%) \\ \hline
avg\_out\_degree & 1.00 & \textbf{0.50 (50.00\%)} & \textbf{0.50 (50.00\%)} & 1.00 (0.00\%) & 1.00 (0.00\%) \\ \hline
max\_in\_degree & 1.00 & 1.00 (0.00\%) & 1.00 (0.00\%) & 1.00 (0.00\%) & 1.00 (0.00\%) \\ \hline
avg\_in\_degree & 1.00 & \textbf{0.50 (50.00\%)} & \textbf{0.50 (50.00\%)} & 1.00 (0.00\%) & 1.00 (0.00\%) \\ \hline
density & 2.00 & \textbf{1.50 (25.00\%)} & \textbf{1.50 (25.00\%)} & 2.00 (0.00\%) & 2.00 (0.00\%) \\ \hline
num\_cc & 1.00 & 1.00 (0.00\%) & 1.00 (0.00\%) & 1.00 (0.00\%) & 1.00 (0.00\%) \\ \hline
largest\_cc & 2.00 & 2.00 (0.00\%) & 2.00 (0.00\%) & 2.00 (0.00\%) & 2.00 (0.00\%) \\ \hline
largest\_cc\_radius & 1.00 & 1.00 (0.00\%) & 1.00 (0.00\%) & 1.00 (0.00\%) & 1.00 (0.00\%) \\ \hline
avg\_path\_len & 1.00 & 1.00 (0.00\%) & 1.00 (0.00\%) & 1.00 (0.00\%) & 1.00 (0.00\%) \\ \hline
largest\_clique\_size & 2.00 & 2.00 (0.00\%) & 2.00 (0.00\%) & 2.00 (0.00\%) & 2.00 (0.00\%) \\ \hline
reciprocity & 1.00 & \textbf{0.00 (100.00\%)} & \textbf{0.00 (100.00\%)} & 1.00 (0.00\%) & 1.00 (0.00\%) \\ \hline
\end{tabular}
}
\end{table}

\begin{table}
\centering
\caption{Instancja cliques\_10 - porównanie wartości miar uzyskanych z próbkowania snowball z wartościami z przeglądu zupełnego.}
\label{tab:comp_cliques_10_snowball}
\footnotesize
\resizebox{\textwidth}{!}{
\begin{tabular}{|l|l|l|l|l|l|}
\hline
\textbf{metryka} & \textbf{p. zupełny} & \textbf{snowball\_1} & \textbf{snowball\_10} & \textbf{snowball\_100} & \textbf{snowball\_1000} \\ \hline
opt\_count & 362880.00 & 601.00 & 1808.00 & 10898.00 & 101698.00 \\ \hline
oracle\_count & 66863792.00 & 88020.00 & 263925.00 & 1626795.00 & 15107760.00 \\ \hline
node\_count & 4.00 & 4.00 (0.00\%) & 4.00 (0.00\%) & 4.00 (0.00\%) & 4.00 (0.00\%) \\ \hline
edge\_count & 16.00 & \textbf{13.00 (18.75\%)} & \textbf{15.00 (6.25\%)} & 16.00 (0.00\%) & 16.00 (0.00\%) \\ \hline
num\_subsinks & 2.00 & 2.00 (0.00\%) & 2.00 (0.00\%) & 2.00 (0.00\%) & 2.00 (0.00\%) \\ \hline
edge\_to\_node & 4.00 & \textbf{3.25 (18.75\%)} & \textbf{3.75 (6.25\%)} & 4.00 (0.00\%) & 4.00 (0.00\%) \\ \hline
avg\_fitness & 3356.50 & 3356.50 (0.00\%) & 3356.50 (0.00\%) & 3356.50 (0.00\%) & 3356.50 (0.00\%) \\ \hline
distLO & 0.01 & \textbf{0.03 (196.42\%)} & \textbf{0.07 (530.44\%)} & \textbf{0.01 (3.83\%)} & \textbf{0.00 (82.48\%)} \\ \hline
conrel & 3.00 & \textbf{1.00 (66.67\%)} & 3.00 (0.00\%) & 3.00 (0.00\%) & 3.00 (0.00\%) \\ \hline
avg\_loop\_weight & 422.00 & \textbf{119.50 (71.68\%)} & \textbf{364.75 (13.57\%)} & \textbf{2219.75 (426.01\%)} & \textbf{20851.00 (4841.00\%)} \\ \hline
go\_path\_ratio & 0.75 & 0.75 (0.00\%) & 0.75 (0.00\%) & 0.75 (0.00\%) & 0.75 (0.00\%) \\ \hline
avg\_go\_path\_len & 1.00 & \textbf{1.33 (33.33\%)} & 1.00 (0.00\%) & 1.00 (0.00\%) & 1.00 (0.00\%) \\ \hline
max\_go\_path\_len & 1.00 & \textbf{2.00 (100.00\%)} & 1.00 (0.00\%) & 1.00 (0.00\%) & 1.00 (0.00\%) \\ \hline
num\_sinks & 2.00 & 2.00 (0.00\%) & 2.00 (0.00\%) & 2.00 (0.00\%) & 2.00 (0.00\%) \\ \hline
num\_sources & 2.00 & 2.00 (0.00\%) & 2.00 (0.00\%) & 2.00 (0.00\%) & 2.00 (0.00\%) \\ \hline
funnel\_num & 2.00 & 2.00 (0.00\%) & 2.00 (0.00\%) & 2.00 (0.00\%) & 2.00 (0.00\%) \\ \hline
mean\_funnel\_size & 3.00 & \textbf{2.50 (16.67\%)} & 3.00 (0.00\%) & 3.00 (0.00\%) & 3.00 (0.00\%) \\ \hline
max\_funnel\_size & 3.00 & 3.00 (0.00\%) & 3.00 (0.00\%) & 3.00 (0.00\%) & 3.00 (0.00\%) \\ \hline
max\_out\_degree & 3.00 & 3.00 (0.00\%) & 3.00 (0.00\%) & 3.00 (0.00\%) & 3.00 (0.00\%) \\ \hline
avg\_out\_degree & 3.00 & \textbf{2.25 (25.00\%)} & \textbf{2.75 (8.33\%)} & 3.00 (0.00\%) & 3.00 (0.00\%) \\ \hline
max\_in\_degree & 3.00 & 3.00 (0.00\%) & 3.00 (0.00\%) & 3.00 (0.00\%) & 3.00 (0.00\%) \\ \hline
avg\_in\_degree & 3.00 & \textbf{2.25 (25.00\%)} & \textbf{2.75 (8.33\%)} & 3.00 (0.00\%) & 3.00 (0.00\%) \\ \hline
density & 1.33 & \textbf{1.08 (18.75\%)} & \textbf{1.25 (6.25\%)} & 1.33 (0.00\%) & 1.33 (0.00\%) \\ \hline
num\_cc & 1.00 & 1.00 (0.00\%) & 1.00 (0.00\%) & 1.00 (0.00\%) & 1.00 (0.00\%) \\ \hline
largest\_cc & 4.00 & 4.00 (0.00\%) & 4.00 (0.00\%) & 4.00 (0.00\%) & 4.00 (0.00\%) \\ \hline
largest\_cc\_radius & 1.00 & 1.00 (0.00\%) & 1.00 (0.00\%) & 1.00 (0.00\%) & 1.00 (0.00\%) \\ \hline
avg\_path\_len & 1.00 & \textbf{1.25 (25.00\%)} & \textbf{1.08 (8.33\%)} & 1.00 (0.00\%) & 1.00 (0.00\%) \\ \hline
largest\_clique\_size & 4.00 & \textbf{3.00 (25.00\%)} & 4.00 (0.00\%) & 4.00 (0.00\%) & 4.00 (0.00\%) \\ \hline
reciprocity & 1.00 & \textbf{0.89 (11.11\%)} & \textbf{0.91 (9.09\%)} & 1.00 (0.00\%) & 1.00 (0.00\%) \\ \hline
\end{tabular}
}
\end{table}

\begin{table}
\centering
\caption{Instancja cliques\_10 - porównanie wartości miar uzyskanych z próbkowania dwufazowego z wartościami z przeglądu zupełnego.}
\label{tab:comp_cliques_10_twophase}
\footnotesize
\resizebox{\textwidth}{!}{
\begin{tabular}{|l|l|l|l|l|l|}
\hline
\textbf{metryka} & \textbf{p. zupełny} & \textbf{dwufazowy\_10} & \textbf{dwufazowy\_100} & \textbf{dwufazowy\_1000} & \textbf{dwufazowy\_10000} \\ \hline
opt\_count & 362880.00 & 107.00 & 10005.00 & 1000004.00 & 100000007.00 \\ \hline
oracle\_count & 66863792.00 & 24357.00 & 2963711.00 & 303071290.00 & 30369385855.00 \\ \hline
node\_count & 4.00 & 4.00 (0.00\%) & 4.00 (0.00\%) & 4.00 (0.00\%) & 4.00 (0.00\%) \\ \hline
edge\_count & 16.00 & \textbf{13.00 (18.75\%)} & 16.00 (0.00\%) & 16.00 (0.00\%) & 16.00 (0.00\%) \\ \hline
num\_subsinks & 2.00 & 2.00 (0.00\%) & 2.00 (0.00\%) & 2.00 (0.00\%) & 2.00 (0.00\%) \\ \hline
edge\_to\_node & 4.00 & \textbf{3.25 (18.75\%)} & 4.00 (0.00\%) & 4.00 (0.00\%) & 4.00 (0.00\%) \\ \hline
avg\_fitness & 3356.50 & 3356.50 (0.00\%) & 3356.50 (0.00\%) & 3356.50 (0.00\%) & 3356.50 (0.00\%) \\ \hline
distLO & 0.01 & \textbf{0.47 (4042.35\%)} & \textbf{0.28 (2361.09\%)} & \textbf{0.01 (9.67\%)} & \textbf{0.00 (90.69\%)} \\ \hline
conrel & 3.00 & 3.00 (0.00\%) & 3.00 (0.00\%) & 3.00 (0.00\%) & 3.00 (0.00\%) \\ \hline
avg\_loop\_weight & 422.00 & \textbf{5.50 (98.70\%)} & \textbf{42.25 (89.99\%)} & \textbf{441.50 (4.62\%)} & \textbf{4308.00 (920.85\%)} \\ \hline
go\_path\_ratio & 0.75 & 0.75 (0.00\%) & 0.75 (0.00\%) & 0.75 (0.00\%) & 0.75 (0.00\%) \\ \hline
avg\_go\_path\_len & 1.00 & 1.00 (0.00\%) & 1.00 (0.00\%) & 1.00 (0.00\%) & 1.00 (0.00\%) \\ \hline
max\_go\_path\_len & 1.00 & 1.00 (0.00\%) & 1.00 (0.00\%) & 1.00 (0.00\%) & 1.00 (0.00\%) \\ \hline
num\_sinks & 2.00 & 2.00 (0.00\%) & 2.00 (0.00\%) & 2.00 (0.00\%) & 2.00 (0.00\%) \\ \hline
num\_sources & 2.00 & 2.00 (0.00\%) & 2.00 (0.00\%) & 2.00 (0.00\%) & 2.00 (0.00\%) \\ \hline
funnel\_num & 2.00 & 2.00 (0.00\%) & 2.00 (0.00\%) & 2.00 (0.00\%) & 2.00 (0.00\%) \\ \hline
mean\_funnel\_size & 3.00 & 3.00 (0.00\%) & 3.00 (0.00\%) & 3.00 (0.00\%) & 3.00 (0.00\%) \\ \hline
max\_funnel\_size & 3.00 & 3.00 (0.00\%) & 3.00 (0.00\%) & 3.00 (0.00\%) & 3.00 (0.00\%) \\ \hline
max\_out\_degree & 3.00 & 3.00 (0.00\%) & 3.00 (0.00\%) & 3.00 (0.00\%) & 3.00 (0.00\%) \\ \hline
avg\_out\_degree & 3.00 & \textbf{2.25 (25.00\%)} & 3.00 (0.00\%) & 3.00 (0.00\%) & 3.00 (0.00\%) \\ \hline
max\_in\_degree & 3.00 & 3.00 (0.00\%) & 3.00 (0.00\%) & 3.00 (0.00\%) & 3.00 (0.00\%) \\ \hline
avg\_in\_degree & 3.00 & \textbf{2.25 (25.00\%)} & 3.00 (0.00\%) & 3.00 (0.00\%) & 3.00 (0.00\%) \\ \hline
density & 1.33 & \textbf{1.08 (18.75\%)} & 1.33 (0.00\%) & 1.33 (0.00\%) & 1.33 (0.00\%) \\ \hline
num\_cc & 1.00 & 1.00 (0.00\%) & 1.00 (0.00\%) & 1.00 (0.00\%) & 1.00 (0.00\%) \\ \hline
largest\_cc & 4.00 & 4.00 (0.00\%) & 4.00 (0.00\%) & 4.00 (0.00\%) & 4.00 (0.00\%) \\ \hline
largest\_cc\_radius & 1.00 & 1.00 (0.00\%) & 1.00 (0.00\%) & 1.00 (0.00\%) & 1.00 (0.00\%) \\ \hline
avg\_path\_len & 1.00 & 1.00 (0.00\%) & 1.00 (0.00\%) & 1.00 (0.00\%) & 1.00 (0.00\%) \\ \hline
largest\_clique\_size & 4.00 & 4.00 (0.00\%) & 4.00 (0.00\%) & 4.00 (0.00\%) & 4.00 (0.00\%) \\ \hline
reciprocity & 1.00 & \textbf{0.67 (33.33\%)} & 1.00 (0.00\%) & 1.00 (0.00\%) & 1.00 (0.00\%) \\ \hline
\end{tabular}
}
\end{table}

\begin{table}
\centering
\caption{Instancja cliques\_11 - porównanie wartości miar uzyskanych z próbkowania snowball z wartościami z przeglądu zupełnego.}
\label{tab:comp_cliques_11_snowball}
\footnotesize
\resizebox{\textwidth}{!}{
\begin{tabular}{|l|l|l|l|l|l|}
\hline
\textbf{metryka} & \textbf{p. zupełny} & \textbf{snowball\_1} & \textbf{snowball\_10} & \textbf{snowball\_100} & \textbf{snowball\_1000} \\ \hline
opt\_count & 3628800.00 & 1001.00 & 1906.00 & 11996.00 & 102096.00 \\ \hline
oracle\_count & 941261556.00 & 182710.00 & 340065.00 & 2181245.00 & 18669915.00 \\ \hline
node\_count & 6.00 & 6.00 (0.00\%) & 6.00 (0.00\%) & 6.00 (0.00\%) & 6.00 (0.00\%) \\ \hline
edge\_count & 36.00 & \textbf{32.00 (11.11\%)} & \textbf{35.00 (2.78\%)} & \textbf{35.00 (2.78\%)} & 36.00 (0.00\%) \\ \hline
num\_subsinks & 2.00 & 2.00 (0.00\%) & 2.00 (0.00\%) & 2.00 (0.00\%) & 2.00 (0.00\%) \\ \hline
edge\_to\_node & 6.00 & \textbf{5.33 (11.11\%)} & \textbf{5.83 (2.78\%)} & \textbf{5.83 (2.78\%)} & 6.00 (0.00\%) \\ \hline
avg\_fitness & 991.67 & 991.67 (0.00\%) & 991.67 (0.00\%) & 991.67 (0.00\%) & 991.67 (0.00\%) \\ \hline
distLO & 0.01 & \textbf{0.26 (2667.10\%)} & \textbf{0.15 (1466.64\%)} & \textbf{0.02 (130.30\%)} & \textbf{0.00 (85.78\%)} \\ \hline
conrel & 5.00 & 5.00 (0.00\%) & 5.00 (0.00\%) & 5.00 (0.00\%) & 5.00 (0.00\%) \\ \hline
avg\_loop\_weight & 729.67 & \textbf{141.83 (80.56\%)} & \textbf{254.33 (65.14\%)} & \textbf{1638.50 (124.55\%)} & \textbf{13907.67 (1806.03\%)} \\ \hline
go\_path\_ratio & 0.83 & 0.83 (0.00\%) & 0.83 (0.00\%) & 0.83 (0.00\%) & 0.83 (0.00\%) \\ \hline
avg\_go\_path\_len & 1.00 & \textbf{1.20 (20.00\%)} & 1.00 (0.00\%) & 1.00 (0.00\%) & 1.00 (0.00\%) \\ \hline
max\_go\_path\_len & 1.00 & \textbf{2.00 (100.00\%)} & 1.00 (0.00\%) & 1.00 (0.00\%) & 1.00 (0.00\%) \\ \hline
num\_sinks & 2.00 & 2.00 (0.00\%) & 2.00 (0.00\%) & 2.00 (0.00\%) & 2.00 (0.00\%) \\ \hline
num\_sources & 4.00 & 4.00 (0.00\%) & 4.00 (0.00\%) & 4.00 (0.00\%) & 4.00 (0.00\%) \\ \hline
funnel\_num & 2.00 & 2.00 (0.00\%) & 2.00 (0.00\%) & 2.00 (0.00\%) & 2.00 (0.00\%) \\ \hline
mean\_funnel\_size & 5.00 & \textbf{4.50 (10.00\%)} & 5.00 (0.00\%) & 5.00 (0.00\%) & 5.00 (0.00\%) \\ \hline
max\_funnel\_size & 5.00 & 5.00 (0.00\%) & 5.00 (0.00\%) & 5.00 (0.00\%) & 5.00 (0.00\%) \\ \hline
max\_out\_degree & 5.00 & 5.00 (0.00\%) & 5.00 (0.00\%) & 5.00 (0.00\%) & 5.00 (0.00\%) \\ \hline
avg\_out\_degree & 5.00 & \textbf{4.33 (13.33\%)} & \textbf{4.83 (3.33\%)} & \textbf{4.83 (3.33\%)} & 5.00 (0.00\%) \\ \hline
max\_in\_degree & 5.00 & 5.00 (0.00\%) & 5.00 (0.00\%) & 5.00 (0.00\%) & 5.00 (0.00\%) \\ \hline
avg\_in\_degree & 5.00 & \textbf{4.33 (13.33\%)} & \textbf{4.83 (3.33\%)} & \textbf{4.83 (3.33\%)} & 5.00 (0.00\%) \\ \hline
density & 1.20 & \textbf{1.07 (11.11\%)} & \textbf{1.17 (2.78\%)} & \textbf{1.17 (2.78\%)} & 1.20 (0.00\%) \\ \hline
num\_cc & 1.00 & 1.00 (0.00\%) & 1.00 (0.00\%) & 1.00 (0.00\%) & 1.00 (0.00\%) \\ \hline
largest\_cc & 6.00 & 6.00 (0.00\%) & 6.00 (0.00\%) & 6.00 (0.00\%) & 6.00 (0.00\%) \\ \hline
largest\_cc\_radius & 1.00 & 1.00 (0.00\%) & 1.00 (0.00\%) & 1.00 (0.00\%) & 1.00 (0.00\%) \\ \hline
avg\_path\_len & 1.00 & \textbf{1.13 (13.33\%)} & \textbf{1.03 (3.33\%)} & \textbf{1.03 (3.33\%)} & 1.00 (0.00\%) \\ \hline
largest\_clique\_size & 6.00 & 6.00 (0.00\%) & 6.00 (0.00\%) & 6.00 (0.00\%) & 6.00 (0.00\%) \\ \hline
reciprocity & 1.00 & \textbf{0.85 (15.38\%)} & \textbf{0.97 (3.45\%)} & \textbf{0.97 (3.45\%)} & 1.00 (0.00\%) \\ \hline
\end{tabular}
}
\end{table}

\begin{table}
\centering
\caption{Instancja cliques\_11 - porównanie wartości miar uzyskanych z próbkowania dwufazowego z wartościami z przeglądu zupełnego.}
\label{tab:comp_cliques_11_twophase}
\footnotesize
\resizebox{\textwidth}{!}{
\begin{tabular}{|l|l|l|l|l|l|}
\hline
\textbf{metryka} & \textbf{p. zupełny} & \textbf{dwufazowy\_10} & \textbf{dwufazowy\_100} & \textbf{dwufazowy\_1000} & \textbf{dwufazowy\_10000} \\ \hline
opt\_count & 3628800.00 & 115.00 & 10023.00 & 1000021.00 & 100000019.00 \\ \hline
oracle\_count & 941261556.00 & 30938.00 & 3968052.00 & 410317830.00 & 41179616838.00 \\ \hline
node\_count & 6.00 & 6.00 (0.00\%) & 6.00 (0.00\%) & 6.00 (0.00\%) & 6.00 (0.00\%) \\ \hline
edge\_count & 36.00 & \textbf{23.00 (36.11\%)} & \textbf{31.00 (13.89\%)} & 36.00 (0.00\%) & 36.00 (0.00\%) \\ \hline
num\_subsinks & 2.00 & \textbf{3.00 (50.00\%)} & 2.00 (0.00\%) & 2.00 (0.00\%) & 2.00 (0.00\%) \\ \hline
edge\_to\_node & 6.00 & \textbf{3.83 (36.11\%)} & \textbf{5.17 (13.89\%)} & 6.00 (0.00\%) & 6.00 (0.00\%) \\ \hline
avg\_fitness & 991.67 & 991.67 (0.00\%) & 991.67 (0.00\%) & 991.67 (0.00\%) & 991.67 (0.00\%) \\ \hline
distLO & 0.01 & \textbf{0.83 (8602.17\%)} & \textbf{0.10 (906.79\%)} & \textbf{0.02 (125.38\%)} & \textbf{0.02 (84.77\%)} \\ \hline
conrel & 5.00 & 5.00 (0.00\%) & 5.00 (0.00\%) & 5.00 (0.00\%) & 5.00 (0.00\%) \\ \hline
avg\_loop\_weight & 729.67 & \textbf{4.67 (99.36\%)} & \textbf{41.50 (94.31\%)} & \textbf{432.50 (40.73\%)} & \textbf{4333.17 (493.86\%)} \\ \hline
go\_path\_ratio & 0.83 & \textbf{0.67 (20.00\%)} & 0.83 (0.00\%) & 0.83 (0.00\%) & 0.83 (0.00\%) \\ \hline
avg\_go\_path\_len & 1.00 & \textbf{1.50 (50.00\%)} & 1.00 (0.00\%) & 1.00 (0.00\%) & 1.00 (0.00\%) \\ \hline
max\_go\_path\_len & 1.00 & \textbf{2.00 (100.00\%)} & 1.00 (0.00\%) & 1.00 (0.00\%) & 1.00 (0.00\%) \\ \hline
num\_sinks & 2.00 & \textbf{3.00 (50.00\%)} & 2.00 (0.00\%) & 2.00 (0.00\%) & 2.00 (0.00\%) \\ \hline
num\_sources & 4.00 & 4.00 (0.00\%) & 4.00 (0.00\%) & 4.00 (0.00\%) & 4.00 (0.00\%) \\ \hline
funnel\_num & 2.00 & \textbf{3.00 (50.00\%)} & 2.00 (0.00\%) & 2.00 (0.00\%) & 2.00 (0.00\%) \\ \hline
mean\_funnel\_size & 5.00 & \textbf{2.33 (53.33\%)} & \textbf{4.50 (10.00\%)} & 5.00 (0.00\%) & 5.00 (0.00\%) \\ \hline
max\_funnel\_size & 5.00 & \textbf{3.00 (40.00\%)} & 5.00 (0.00\%) & 5.00 (0.00\%) & 5.00 (0.00\%) \\ \hline
max\_out\_degree & 5.00 & \textbf{4.00 (20.00\%)} & 5.00 (0.00\%) & 5.00 (0.00\%) & 5.00 (0.00\%) \\ \hline
avg\_out\_degree & 5.00 & \textbf{2.83 (43.33\%)} & \textbf{4.17 (16.67\%)} & 5.00 (0.00\%) & 5.00 (0.00\%) \\ \hline
max\_in\_degree & 5.00 & 5.00 (0.00\%) & 5.00 (0.00\%) & 5.00 (0.00\%) & 5.00 (0.00\%) \\ \hline
avg\_in\_degree & 5.00 & \textbf{2.83 (43.33\%)} & \textbf{4.17 (16.67\%)} & 5.00 (0.00\%) & 5.00 (0.00\%) \\ \hline
density & 1.20 & \textbf{0.77 (36.11\%)} & \textbf{1.03 (13.89\%)} & 1.20 (0.00\%) & 1.20 (0.00\%) \\ \hline
num\_cc & 1.00 & 1.00 (0.00\%) & 1.00 (0.00\%) & 1.00 (0.00\%) & 1.00 (0.00\%) \\ \hline
largest\_cc & 6.00 & 6.00 (0.00\%) & 6.00 (0.00\%) & 6.00 (0.00\%) & 6.00 (0.00\%) \\ \hline
largest\_cc\_radius & 1.00 & 1.00 (0.00\%) & 1.00 (0.00\%) & 1.00 (0.00\%) & 1.00 (0.00\%) \\ \hline
avg\_path\_len & 1.00 & \textbf{1.36 (36.00\%)} & \textbf{1.17 (16.67\%)} & 1.00 (0.00\%) & 1.00 (0.00\%) \\ \hline
largest\_clique\_size & 6.00 & \textbf{5.00 (16.67\%)} & 6.00 (0.00\%) & 6.00 (0.00\%) & 6.00 (0.00\%) \\ \hline
reciprocity & 1.00 & \textbf{0.35 (64.71\%)} & \textbf{0.80 (20.00\%)} & 1.00 (0.00\%) & 1.00 (0.00\%) \\ \hline
\end{tabular}
}
\end{table}

\begin{table}
\centering
\caption{Instancja grid\_7 - porównanie wartości miar uzyskanych z próbkowania snowball z wartościami z przeglądu zupełnego.}
\label{tab:comp_grid_7_snowball}
\footnotesize
\resizebox{\textwidth}{!}{
\begin{tabular}{|l|l|l|l|l|l|}
\hline
\textbf{metryka} & \textbf{p. zupełny} & \textbf{snowball\_1} & \textbf{snowball\_10} & \textbf{snowball\_100} & \textbf{snowball\_1000} \\ \hline
opt\_count & 720.00 & 301.00 & 1210.00 & 10300.00 & 101200.00 \\ \hline
oracle\_count & 40111.00 & 18564.00 & 72912.00 & 624981.00 & 6124188.00 \\ \hline
node\_count & 2.00 & 2.00 (0.00\%) & 2.00 (0.00\%) & 2.00 (0.00\%) & 2.00 (0.00\%) \\ \hline
edge\_count & 4.00 & 4.00 (0.00\%) & 4.00 (0.00\%) & 4.00 (0.00\%) & 4.00 (0.00\%) \\ \hline
num\_subsinks & 2.00 & 2.00 (0.00\%) & 2.00 (0.00\%) & 2.00 (0.00\%) & 2.00 (0.00\%) \\ \hline
edge\_to\_node & 2.00 & 2.00 (0.00\%) & 2.00 (0.00\%) & 2.00 (0.00\%) & 2.00 (0.00\%) \\ \hline
avg\_fitness & 741.00 & 741.00 (0.00\%) & 741.00 (0.00\%) & 741.00 (0.00\%) & 741.00 (0.00\%) \\ \hline
distLO & 0.01 & \textbf{0.03 (124.07\%)} & \textbf{0.01 (49.98\%)} & \textbf{0.00 (93.94\%)} & \textbf{0.00 (99.43\%)} \\ \hline
conrel & 1.00 & 1.00 (0.00\%) & 1.00 (0.00\%) & 1.00 (0.00\%) & 1.00 (0.00\%) \\ \hline
avg\_loop\_weight & 99.50 & \textbf{127.50 (28.14\%)} & \textbf{507.50 (410.05\%)} & \textbf{4289.00 (4210.55\%)} & \textbf{42077.50 (42188.94\%)} \\ \hline
go\_path\_ratio & 0.50 & 0.50 (0.00\%) & 0.50 (0.00\%) & 0.50 (0.00\%) & 0.50 (0.00\%) \\ \hline
avg\_go\_path\_len & 1.00 & 1.00 (0.00\%) & 1.00 (0.00\%) & 1.00 (0.00\%) & 1.00 (0.00\%) \\ \hline
max\_go\_path\_len & 1.00 & 1.00 (0.00\%) & 1.00 (0.00\%) & 1.00 (0.00\%) & 1.00 (0.00\%) \\ \hline
num\_sinks & 2.00 & 2.00 (0.00\%) & 2.00 (0.00\%) & 2.00 (0.00\%) & 2.00 (0.00\%) \\ \hline
num\_sources & 2.00 & 2.00 (0.00\%) & 2.00 (0.00\%) & 2.00 (0.00\%) & 2.00 (0.00\%) \\ \hline
funnel\_num & 2.00 & 2.00 (0.00\%) & 2.00 (0.00\%) & 2.00 (0.00\%) & 2.00 (0.00\%) \\ \hline
mean\_funnel\_size & 1.00 & 1.00 (0.00\%) & 1.00 (0.00\%) & 1.00 (0.00\%) & 1.00 (0.00\%) \\ \hline
max\_funnel\_size & 1.00 & 1.00 (0.00\%) & 1.00 (0.00\%) & 1.00 (0.00\%) & 1.00 (0.00\%) \\ \hline
max\_out\_degree & 1.00 & 1.00 (0.00\%) & 1.00 (0.00\%) & 1.00 (0.00\%) & 1.00 (0.00\%) \\ \hline
avg\_out\_degree & 1.00 & 1.00 (0.00\%) & 1.00 (0.00\%) & 1.00 (0.00\%) & 1.00 (0.00\%) \\ \hline
max\_in\_degree & 1.00 & 1.00 (0.00\%) & 1.00 (0.00\%) & 1.00 (0.00\%) & 1.00 (0.00\%) \\ \hline
avg\_in\_degree & 1.00 & 1.00 (0.00\%) & 1.00 (0.00\%) & 1.00 (0.00\%) & 1.00 (0.00\%) \\ \hline
density & 2.00 & 2.00 (0.00\%) & 2.00 (0.00\%) & 2.00 (0.00\%) & 2.00 (0.00\%) \\ \hline
num\_cc & 1.00 & 1.00 (0.00\%) & 1.00 (0.00\%) & 1.00 (0.00\%) & 1.00 (0.00\%) \\ \hline
largest\_cc & 2.00 & 2.00 (0.00\%) & 2.00 (0.00\%) & 2.00 (0.00\%) & 2.00 (0.00\%) \\ \hline
largest\_cc\_radius & 1.00 & 1.00 (0.00\%) & 1.00 (0.00\%) & 1.00 (0.00\%) & 1.00 (0.00\%) \\ \hline
avg\_path\_len & 1.00 & 1.00 (0.00\%) & 1.00 (0.00\%) & 1.00 (0.00\%) & 1.00 (0.00\%) \\ \hline
largest\_clique\_size & 2.00 & 2.00 (0.00\%) & 2.00 (0.00\%) & 2.00 (0.00\%) & 2.00 (0.00\%) \\ \hline
reciprocity & 1.00 & 1.00 (0.00\%) & 1.00 (0.00\%) & 1.00 (0.00\%) & 1.00 (0.00\%) \\ \hline
\end{tabular}
}
\end{table}

\begin{table}
\centering
\caption{Instancja grid\_7 - porównanie wartości miar uzyskanych z próbkowania dwufazowego z wartościami z przeglądu zupełnego.}
\label{tab:comp_grid_7_twophase}
\footnotesize
\resizebox{\textwidth}{!}{
\begin{tabular}{|l|l|l|l|l|l|}
\hline
\textbf{metryka} & \textbf{p. zupełny} & \textbf{dwufazowy\_10} & \textbf{dwufazowy\_100} & \textbf{dwufazowy\_1000} & \textbf{dwufazowy\_10000} \\ \hline
opt\_count & 720.00 & 102.00 & 10002.00 & 1000003.00 & 100000002.00 \\ \hline
oracle\_count & 40111.00 & 7651.00 & 836455.00 & 84307312.00 & 8437454946.00 \\ \hline
node\_count & 2.00 & 2.00 (0.00\%) & 2.00 (0.00\%) & 2.00 (0.00\%) & 2.00 (0.00\%) \\ \hline
edge\_count & 4.00 & 4.00 (0.00\%) & 4.00 (0.00\%) & 4.00 (0.00\%) & 4.00 (0.00\%) \\ \hline
num\_subsinks & 2.00 & 2.00 (0.00\%) & 2.00 (0.00\%) & 2.00 (0.00\%) & 2.00 (0.00\%) \\ \hline
edge\_to\_node & 2.00 & 2.00 (0.00\%) & 2.00 (0.00\%) & 2.00 (0.00\%) & 2.00 (0.00\%) \\ \hline
avg\_fitness & 741.00 & 741.00 (0.00\%) & 741.00 (0.00\%) & 741.00 (0.00\%) & 741.00 (0.00\%) \\ \hline
distLO & 0.01 & \textbf{0.16 (1266.24\%)} & \textbf{0.02 (30.04\%)} & \textbf{0.01 (51.58\%)} & \textbf{0.00 (98.65\%)} \\ \hline
conrel & 1.00 & 1.00 (0.00\%) & 1.00 (0.00\%) & 1.00 (0.00\%) & 1.00 (0.00\%) \\ \hline
avg\_loop\_weight & 99.50 & \textbf{6.50 (93.47\%)} & \textbf{73.00 (26.63\%)} & \textbf{725.00 (628.64\%)} & \textbf{7075.50 (7011.06\%)} \\ \hline
go\_path\_ratio & 0.50 & 0.50 (0.00\%) & 0.50 (0.00\%) & 0.50 (0.00\%) & 0.50 (0.00\%) \\ \hline
avg\_go\_path\_len & 1.00 & 1.00 (0.00\%) & 1.00 (0.00\%) & 1.00 (0.00\%) & 1.00 (0.00\%) \\ \hline
max\_go\_path\_len & 1.00 & 1.00 (0.00\%) & 1.00 (0.00\%) & 1.00 (0.00\%) & 1.00 (0.00\%) \\ \hline
num\_sinks & 2.00 & 2.00 (0.00\%) & 2.00 (0.00\%) & 2.00 (0.00\%) & 2.00 (0.00\%) \\ \hline
num\_sources & 2.00 & 2.00 (0.00\%) & 2.00 (0.00\%) & 2.00 (0.00\%) & 2.00 (0.00\%) \\ \hline
funnel\_num & 2.00 & 2.00 (0.00\%) & 2.00 (0.00\%) & 2.00 (0.00\%) & 2.00 (0.00\%) \\ \hline
mean\_funnel\_size & 1.00 & 1.00 (0.00\%) & 1.00 (0.00\%) & 1.00 (0.00\%) & 1.00 (0.00\%) \\ \hline
max\_funnel\_size & 1.00 & 1.00 (0.00\%) & 1.00 (0.00\%) & 1.00 (0.00\%) & 1.00 (0.00\%) \\ \hline
max\_out\_degree & 1.00 & 1.00 (0.00\%) & 1.00 (0.00\%) & 1.00 (0.00\%) & 1.00 (0.00\%) \\ \hline
avg\_out\_degree & 1.00 & 1.00 (0.00\%) & 1.00 (0.00\%) & 1.00 (0.00\%) & 1.00 (0.00\%) \\ \hline
max\_in\_degree & 1.00 & 1.00 (0.00\%) & 1.00 (0.00\%) & 1.00 (0.00\%) & 1.00 (0.00\%) \\ \hline
avg\_in\_degree & 1.00 & 1.00 (0.00\%) & 1.00 (0.00\%) & 1.00 (0.00\%) & 1.00 (0.00\%) \\ \hline
density & 2.00 & 2.00 (0.00\%) & 2.00 (0.00\%) & 2.00 (0.00\%) & 2.00 (0.00\%) \\ \hline
num\_cc & 1.00 & 1.00 (0.00\%) & 1.00 (0.00\%) & 1.00 (0.00\%) & 1.00 (0.00\%) \\ \hline
largest\_cc & 2.00 & 2.00 (0.00\%) & 2.00 (0.00\%) & 2.00 (0.00\%) & 2.00 (0.00\%) \\ \hline
largest\_cc\_radius & 1.00 & 1.00 (0.00\%) & 1.00 (0.00\%) & 1.00 (0.00\%) & 1.00 (0.00\%) \\ \hline
avg\_path\_len & 1.00 & 1.00 (0.00\%) & 1.00 (0.00\%) & 1.00 (0.00\%) & 1.00 (0.00\%) \\ \hline
largest\_clique\_size & 2.00 & 2.00 (0.00\%) & 2.00 (0.00\%) & 2.00 (0.00\%) & 2.00 (0.00\%) \\ \hline
reciprocity & 1.00 & 1.00 (0.00\%) & 1.00 (0.00\%) & 1.00 (0.00\%) & 1.00 (0.00\%) \\ \hline
\end{tabular}
}
\end{table}

\begin{table}
\centering
\caption{Instancja grid\_8 - porównanie wartości miar uzyskanych z próbkowania snowball z wartościami z przeglądu zupełnego.}
\label{tab:comp_grid_8_snowball}
\footnotesize
\resizebox{\textwidth}{!}{
\begin{tabular}{|l|l|l|l|l|l|}
\hline
\textbf{metryka} & \textbf{p. zupełny} & \textbf{snowball\_1} & \textbf{snowball\_10} & \textbf{snowball\_100} & \textbf{snowball\_1000} \\ \hline
opt\_count & 5040.00 & 401.00 & 1210.00 & 10300.00 & 101200.00 \\ \hline
oracle\_count & 448412.00 & 34216.00 & 102984.00 & 868392.00 & 8524068.00 \\ \hline
node\_count & 2.00 & 2.00 (0.00\%) & 2.00 (0.00\%) & 2.00 (0.00\%) & 2.00 (0.00\%) \\ \hline
edge\_count & 4.00 & 4.00 (0.00\%) & 4.00 (0.00\%) & 4.00 (0.00\%) & 4.00 (0.00\%) \\ \hline
num\_subsinks & 2.00 & 2.00 (0.00\%) & 2.00 (0.00\%) & 2.00 (0.00\%) & 2.00 (0.00\%) \\ \hline
edge\_to\_node & 2.00 & 2.00 (0.00\%) & 2.00 (0.00\%) & 2.00 (0.00\%) & 2.00 (0.00\%) \\ \hline
avg\_fitness & 800.00 & 800.00 (0.00\%) & 800.00 (0.00\%) & 800.00 (0.00\%) & 800.00 (0.00\%) \\ \hline
distLO & 0.01 & \textbf{0.03 (370.57\%)} & \textbf{0.01 (65.35\%)} & \textbf{0.00 (82.66\%)} & \textbf{0.00 (98.44\%)} \\ \hline
conrel & 1.00 & 1.00 (0.00\%) & 1.00 (0.00\%) & 1.00 (0.00\%) & 1.00 (0.00\%) \\ \hline
avg\_loop\_weight & 214.00 & \textbf{180.50 (15.65\%)} & \textbf{541.00 (152.80\%)} & \textbf{4540.00 (2021.50\%)} & \textbf{44727.50 (20800.70\%)} \\ \hline
go\_path\_ratio & 0.50 & 0.50 (0.00\%) & 0.50 (0.00\%) & 0.50 (0.00\%) & 0.50 (0.00\%) \\ \hline
avg\_go\_path\_len & 1.00 & 1.00 (0.00\%) & 1.00 (0.00\%) & 1.00 (0.00\%) & 1.00 (0.00\%) \\ \hline
max\_go\_path\_len & 1.00 & 1.00 (0.00\%) & 1.00 (0.00\%) & 1.00 (0.00\%) & 1.00 (0.00\%) \\ \hline
num\_sinks & 2.00 & 2.00 (0.00\%) & 2.00 (0.00\%) & 2.00 (0.00\%) & 2.00 (0.00\%) \\ \hline
num\_sources & 2.00 & 2.00 (0.00\%) & 2.00 (0.00\%) & 2.00 (0.00\%) & 2.00 (0.00\%) \\ \hline
funnel\_num & 2.00 & 2.00 (0.00\%) & 2.00 (0.00\%) & 2.00 (0.00\%) & 2.00 (0.00\%) \\ \hline
mean\_funnel\_size & 1.00 & 1.00 (0.00\%) & 1.00 (0.00\%) & 1.00 (0.00\%) & 1.00 (0.00\%) \\ \hline
max\_funnel\_size & 1.00 & 1.00 (0.00\%) & 1.00 (0.00\%) & 1.00 (0.00\%) & 1.00 (0.00\%) \\ \hline
max\_out\_degree & 1.00 & 1.00 (0.00\%) & 1.00 (0.00\%) & 1.00 (0.00\%) & 1.00 (0.00\%) \\ \hline
avg\_out\_degree & 1.00 & 1.00 (0.00\%) & 1.00 (0.00\%) & 1.00 (0.00\%) & 1.00 (0.00\%) \\ \hline
max\_in\_degree & 1.00 & 1.00 (0.00\%) & 1.00 (0.00\%) & 1.00 (0.00\%) & 1.00 (0.00\%) \\ \hline
avg\_in\_degree & 1.00 & 1.00 (0.00\%) & 1.00 (0.00\%) & 1.00 (0.00\%) & 1.00 (0.00\%) \\ \hline
density & 2.00 & 2.00 (0.00\%) & 2.00 (0.00\%) & 2.00 (0.00\%) & 2.00 (0.00\%) \\ \hline
num\_cc & 1.00 & 1.00 (0.00\%) & 1.00 (0.00\%) & 1.00 (0.00\%) & 1.00 (0.00\%) \\ \hline
largest\_cc & 2.00 & 2.00 (0.00\%) & 2.00 (0.00\%) & 2.00 (0.00\%) & 2.00 (0.00\%) \\ \hline
largest\_cc\_radius & 1.00 & 1.00 (0.00\%) & 1.00 (0.00\%) & 1.00 (0.00\%) & 1.00 (0.00\%) \\ \hline
avg\_path\_len & 1.00 & 1.00 (0.00\%) & 1.00 (0.00\%) & 1.00 (0.00\%) & 1.00 (0.00\%) \\ \hline
largest\_clique\_size & 2.00 & 2.00 (0.00\%) & 2.00 (0.00\%) & 2.00 (0.00\%) & 2.00 (0.00\%) \\ \hline
reciprocity & 1.00 & 1.00 (0.00\%) & 1.00 (0.00\%) & 1.00 (0.00\%) & 1.00 (0.00\%) \\ \hline
\end{tabular}
}
\end{table}

\begin{table}
\centering
\caption{Instancja grid\_8 - porównanie wartości miar uzyskanych z próbkowania dwufazowego z wartościami z przeglądu zupełnego.}
\label{tab:comp_grid_8_twophase}
\footnotesize
\resizebox{\textwidth}{!}{
\begin{tabular}{|l|l|l|l|l|l|}
\hline
\textbf{metryka} & \textbf{p. zupełny} & \textbf{dwufazowy\_10} & \textbf{dwufazowy\_100} & \textbf{dwufazowy\_1000} & \textbf{dwufazowy\_10000} \\ \hline
opt\_count & 5040.00 & 103.00 & 10003.00 & 1000003.00 & 100000003.00 \\ \hline
oracle\_count & 448412.00 & 11936.00 & 1341516.00 & 135233837.00 & 13535783234.00 \\ \hline
node\_count & 2.00 & 2.00 (0.00\%) & 2.00 (0.00\%) & 2.00 (0.00\%) & 2.00 (0.00\%) \\ \hline
edge\_count & 4.00 & \textbf{3.00 (25.00\%)} & 4.00 (0.00\%) & 4.00 (0.00\%) & 4.00 (0.00\%) \\ \hline
num\_subsinks & 2.00 & 2.00 (0.00\%) & 2.00 (0.00\%) & 2.00 (0.00\%) & 2.00 (0.00\%) \\ \hline
edge\_to\_node & 2.00 & \textbf{1.50 (25.00\%)} & 2.00 (0.00\%) & 2.00 (0.00\%) & 2.00 (0.00\%) \\ \hline
avg\_fitness & 800.00 & 800.00 (0.00\%) & 800.00 (0.00\%) & 800.00 (0.00\%) & 800.00 (0.00\%) \\ \hline
distLO & 0.01 & \textbf{0.13 (1952.38\%)} & \textbf{0.07 (992.50\%)} & \textbf{0.01 (2.32\%)} & \textbf{0.00 (97.32\%)} \\ \hline
conrel & 1.00 & 1.00 (0.00\%) & 1.00 (0.00\%) & 1.00 (0.00\%) & 1.00 (0.00\%) \\ \hline
avg\_loop\_weight & 214.00 & \textbf{7.00 (96.73\%)} & \textbf{75.50 (64.72\%)} & \textbf{757.00 (253.74\%)} & \textbf{7526.50 (3417.06\%)} \\ \hline
go\_path\_ratio & 0.50 & 0.50 (0.00\%) & 0.50 (0.00\%) & 0.50 (0.00\%) & 0.50 (0.00\%) \\ \hline
avg\_go\_path\_len & 1.00 & 1.00 (0.00\%) & 1.00 (0.00\%) & 1.00 (0.00\%) & 1.00 (0.00\%) \\ \hline
max\_go\_path\_len & 1.00 & 1.00 (0.00\%) & 1.00 (0.00\%) & 1.00 (0.00\%) & 1.00 (0.00\%) \\ \hline
num\_sinks & 2.00 & 2.00 (0.00\%) & 2.00 (0.00\%) & 2.00 (0.00\%) & 2.00 (0.00\%) \\ \hline
num\_sources & 2.00 & 2.00 (0.00\%) & 2.00 (0.00\%) & 2.00 (0.00\%) & 2.00 (0.00\%) \\ \hline
funnel\_num & 2.00 & 2.00 (0.00\%) & 2.00 (0.00\%) & 2.00 (0.00\%) & 2.00 (0.00\%) \\ \hline
mean\_funnel\_size & 1.00 & 1.00 (0.00\%) & 1.00 (0.00\%) & 1.00 (0.00\%) & 1.00 (0.00\%) \\ \hline
max\_funnel\_size & 1.00 & 1.00 (0.00\%) & 1.00 (0.00\%) & 1.00 (0.00\%) & 1.00 (0.00\%) \\ \hline
max\_out\_degree & 1.00 & 1.00 (0.00\%) & 1.00 (0.00\%) & 1.00 (0.00\%) & 1.00 (0.00\%) \\ \hline
avg\_out\_degree & 1.00 & \textbf{0.50 (50.00\%)} & 1.00 (0.00\%) & 1.00 (0.00\%) & 1.00 (0.00\%) \\ \hline
max\_in\_degree & 1.00 & 1.00 (0.00\%) & 1.00 (0.00\%) & 1.00 (0.00\%) & 1.00 (0.00\%) \\ \hline
avg\_in\_degree & 1.00 & \textbf{0.50 (50.00\%)} & 1.00 (0.00\%) & 1.00 (0.00\%) & 1.00 (0.00\%) \\ \hline
density & 2.00 & \textbf{1.50 (25.00\%)} & 2.00 (0.00\%) & 2.00 (0.00\%) & 2.00 (0.00\%) \\ \hline
num\_cc & 1.00 & 1.00 (0.00\%) & 1.00 (0.00\%) & 1.00 (0.00\%) & 1.00 (0.00\%) \\ \hline
largest\_cc & 2.00 & 2.00 (0.00\%) & 2.00 (0.00\%) & 2.00 (0.00\%) & 2.00 (0.00\%) \\ \hline
largest\_cc\_radius & 1.00 & 1.00 (0.00\%) & 1.00 (0.00\%) & 1.00 (0.00\%) & 1.00 (0.00\%) \\ \hline
avg\_path\_len & 1.00 & 1.00 (0.00\%) & 1.00 (0.00\%) & 1.00 (0.00\%) & 1.00 (0.00\%) \\ \hline
largest\_clique\_size & 2.00 & 2.00 (0.00\%) & 2.00 (0.00\%) & 2.00 (0.00\%) & 2.00 (0.00\%) \\ \hline
reciprocity & 1.00 & \textbf{0.00 (100.00\%)} & 1.00 (0.00\%) & 1.00 (0.00\%) & 1.00 (0.00\%) \\ \hline
\end{tabular}
}
\end{table}

\begin{table}
\centering
\caption{Instancja grid\_9 - porównanie wartości miar uzyskanych z próbkowania snowball z wartościami z przeglądu zupełnego.}
\label{tab:comp_grid_9_snowball}
\footnotesize
\resizebox{\textwidth}{!}{
\begin{tabular}{|l|l|l|l|l|l|}
\hline
\textbf{metryka} & \textbf{p. zupełny} & \textbf{snowball\_1} & \textbf{snowball\_10} & \textbf{snowball\_100} & \textbf{snowball\_1000} \\ \hline
opt\_count & 40320.00 & 2501.00 & 5401.00 & 17386.00 & 108686.00 \\ \hline
oracle\_count & 4548760.00 & 240012.00 & 530136.00 & 1697868.00 & 10488780.00 \\ \hline
node\_count & 16.00 & 16.00 (0.00\%) & 16.00 (0.00\%) & 16.00 (0.00\%) & 16.00 (0.00\%) \\ \hline
edge\_count & 236.00 & \textbf{131.00 (44.49\%)} & \textbf{191.00 (19.07\%)} & \textbf{228.00 (3.39\%)} & 236.00 (0.00\%) \\ \hline
num\_subsinks & 16.00 & 16.00 (0.00\%) & 16.00 (0.00\%) & 16.00 (0.00\%) & 16.00 (0.00\%) \\ \hline
edge\_to\_node & 14.75 & \textbf{8.19 (44.49\%)} & \textbf{11.94 (19.07\%)} & \textbf{14.25 (3.39\%)} & 14.75 (0.00\%) \\ \hline
avg\_fitness & 941.00 & 941.00 (0.00\%) & 941.00 (0.00\%) & 941.00 (0.00\%) & 941.00 (0.00\%) \\ \hline
distLO & 0.01 & \textbf{0.34 (2442.55\%)} & \textbf{0.28 (1974.80\%)} & \textbf{0.05 (281.88\%)} & \textbf{0.01 (34.25\%)} \\ \hline
conrel & 15.00 & \textbf{4.33 (71.11\%)} & 15.00 (0.00\%) & 15.00 (0.00\%) & 15.00 (0.00\%) \\ \hline
avg\_loop\_weight & 119.62 & \textbf{90.25 (24.56\%)} & \textbf{152.38 (27.38\%)} & \textbf{496.12 (314.73\%)} & \textbf{3191.19 (2567.66\%)} \\ \hline
go\_path\_ratio & 0.94 & \textbf{0.69 (26.67\%)} & 0.94 (0.00\%) & 0.94 (0.00\%) & 0.94 (0.00\%) \\ \hline
avg\_go\_path\_len & 1.00 & \textbf{1.09 (9.09\%)} & 1.00 (0.00\%) & 1.00 (0.00\%) & 1.00 (0.00\%) \\ \hline
max\_go\_path\_len & 1.00 & \textbf{2.00 (100.00\%)} & 1.00 (0.00\%) & 1.00 (0.00\%) & 1.00 (0.00\%) \\ \hline
num\_sinks & 16.00 & 16.00 (0.00\%) & 16.00 (0.00\%) & 16.00 (0.00\%) & 16.00 (0.00\%) \\ \hline
num\_sources & 16.00 & 16.00 (0.00\%) & 16.00 (0.00\%) & 16.00 (0.00\%) & 16.00 (0.00\%) \\ \hline
funnel\_num & 16.00 & 16.00 (0.00\%) & 16.00 (0.00\%) & 16.00 (0.00\%) & 16.00 (0.00\%) \\ \hline
mean\_funnel\_size & 1.00 & 1.00 (0.00\%) & 1.00 (0.00\%) & 1.00 (0.00\%) & 1.00 (0.00\%) \\ \hline
max\_funnel\_size & 1.00 & 1.00 (0.00\%) & 1.00 (0.00\%) & 1.00 (0.00\%) & 1.00 (0.00\%) \\ \hline
max\_out\_degree & 15.00 & \textbf{12.00 (20.00\%)} & \textbf{14.00 (6.67\%)} & 15.00 (0.00\%) & 15.00 (0.00\%) \\ \hline
avg\_out\_degree & 13.75 & \textbf{7.44 (45.91\%)} & \textbf{10.94 (20.45\%)} & \textbf{13.25 (3.64\%)} & 13.75 (0.00\%) \\ \hline
max\_in\_degree & 15.00 & \textbf{11.00 (26.67\%)} & 15.00 (0.00\%) & 15.00 (0.00\%) & 15.00 (0.00\%) \\ \hline
avg\_in\_degree & 13.75 & \textbf{7.44 (45.91\%)} & \textbf{10.94 (20.45\%)} & \textbf{13.25 (3.64\%)} & 13.75 (0.00\%) \\ \hline
density & 0.98 & \textbf{0.55 (44.49\%)} & \textbf{0.80 (19.07\%)} & \textbf{0.95 (3.39\%)} & 0.98 (0.00\%) \\ \hline
num\_cc & 1.00 & 1.00 (0.00\%) & 1.00 (0.00\%) & 1.00 (0.00\%) & 1.00 (0.00\%) \\ \hline
largest\_cc & 16.00 & 16.00 (0.00\%) & 16.00 (0.00\%) & 16.00 (0.00\%) & 16.00 (0.00\%) \\ \hline
largest\_cc\_radius & 1.00 & \textbf{2.00 (100.00\%)} & 1.00 (0.00\%) & 1.00 (0.00\%) & 1.00 (0.00\%) \\ \hline
avg\_path\_len & 1.08 & \textbf{1.34 (23.59\%)} & \textbf{1.27 (17.31\%)} & \textbf{1.12 (3.08\%)} & 1.08 (0.00\%) \\ \hline
largest\_clique\_size & 14.00 & \textbf{8.00 (42.86\%)} & \textbf{10.00 (28.57\%)} & \textbf{12.00 (14.29\%)} & \textbf{12.00 (14.29\%)} \\ \hline
reciprocity & 0.93 & \textbf{0.72 (22.06\%)} & \textbf{0.79 (14.96\%)} & \textbf{0.97 (4.79\%)} & \textbf{1.00 (7.84\%)} \\ \hline
\end{tabular}
}
\end{table}

\begin{table}
\centering
\caption{Instancja grid\_9 - porównanie wartości miar uzyskanych z próbkowania dwufazowego z wartościami z przeglądu zupełnego.}
\label{tab:comp_grid_9_twophase}
\footnotesize
\resizebox{\textwidth}{!}{
\begin{tabular}{|l|l|l|l|l|l|}
\hline
\textbf{metryka} & \textbf{p. zupełny} & \textbf{dwufazowy\_10} & \textbf{dwufazowy\_100} & \textbf{dwufazowy\_1000} & \textbf{dwufazowy\_10000} \\ \hline
opt\_count & 40320.00 & 119.00 & 10177.00 & 1000177.00 & 100000110.00 \\ \hline
oracle\_count & 4548760.00 & 11307.00 & 1588061.00 & 168868459.00 & 17011086742.00 \\ \hline
node\_count & 16.00 & \textbf{10.00 (37.50\%)} & 16.00 (0.00\%) & 16.00 (0.00\%) & 16.00 (0.00\%) \\ \hline
edge\_count & 236.00 & \textbf{39.00 (83.47\%)} & \textbf{160.00 (32.20\%)} & \textbf{223.00 (5.51\%)} & 236.00 (0.00\%) \\ \hline
num\_subsinks & 16.00 & \textbf{10.00 (37.50\%)} & 16.00 (0.00\%) & 16.00 (0.00\%) & 16.00 (0.00\%) \\ \hline
edge\_to\_node & 14.75 & \textbf{3.90 (73.56\%)} & \textbf{10.00 (32.20\%)} & \textbf{13.94 (5.51\%)} & 14.75 (0.00\%) \\ \hline
avg\_fitness & 941.00 & 941.00 (0.00\%) & 941.00 (0.00\%) & 941.00 (0.00\%) & 941.00 (0.00\%) \\ \hline
distLO & 0.01 & \textbf{0.56 (4094.46\%)} & \textbf{0.56 (4138.42\%)} & \textbf{0.01 (32.69\%)} & \textbf{0.00 (93.35\%)} \\ \hline
conrel & 15.00 & \textbf{1.50 (90.00\%)} & 15.00 (0.00\%) & 15.00 (0.00\%) & 15.00 (0.00\%) \\ \hline
avg\_loop\_weight & 119.62 & \textbf{3.00 (97.49\%)} & \textbf{23.06 (80.72\%)} & \textbf{226.19 (89.08\%)} & \textbf{2257.38 (1787.04\%)} \\ \hline
go\_path\_ratio & 0.94 & \textbf{0.90 (4.00\%)} & 0.94 (0.00\%) & 0.94 (0.00\%) & 0.94 (0.00\%) \\ \hline
avg\_go\_path\_len & 1.00 & \textbf{1.56 (55.56\%)} & \textbf{2.60 (160.00\%)} & 1.00 (0.00\%) & 1.00 (0.00\%) \\ \hline
max\_go\_path\_len & 1.00 & \textbf{2.00 (100.00\%)} & \textbf{3.00 (200.00\%)} & 1.00 (0.00\%) & 1.00 (0.00\%) \\ \hline
num\_sinks & 16.00 & \textbf{10.00 (37.50\%)} & 16.00 (0.00\%) & 16.00 (0.00\%) & 16.00 (0.00\%) \\ \hline
num\_sources & 16.00 & \textbf{10.00 (37.50\%)} & 16.00 (0.00\%) & 16.00 (0.00\%) & 16.00 (0.00\%) \\ \hline
funnel\_num & 16.00 & \textbf{10.00 (37.50\%)} & 16.00 (0.00\%) & 16.00 (0.00\%) & 16.00 (0.00\%) \\ \hline
mean\_funnel\_size & 1.00 & 1.00 (0.00\%) & 1.00 (0.00\%) & 1.00 (0.00\%) & 1.00 (0.00\%) \\ \hline
max\_funnel\_size & 1.00 & 1.00 (0.00\%) & 1.00 (0.00\%) & 1.00 (0.00\%) & 1.00 (0.00\%) \\ \hline
max\_out\_degree & 15.00 & \textbf{5.00 (66.67\%)} & \textbf{14.00 (6.67\%)} & 15.00 (0.00\%) & 15.00 (0.00\%) \\ \hline
avg\_out\_degree & 13.75 & \textbf{2.90 (78.91\%)} & \textbf{9.00 (34.55\%)} & \textbf{12.94 (5.91\%)} & 13.75 (0.00\%) \\ \hline
max\_in\_degree & 15.00 & \textbf{8.00 (46.67\%)} & 15.00 (0.00\%) & 15.00 (0.00\%) & 15.00 (0.00\%) \\ \hline
avg\_in\_degree & 13.75 & \textbf{2.90 (78.91\%)} & \textbf{9.00 (34.55\%)} & \textbf{12.94 (5.91\%)} & 13.75 (0.00\%) \\ \hline
density & 0.98 & \textbf{0.43 (55.93\%)} & \textbf{0.67 (32.20\%)} & \textbf{0.93 (5.51\%)} & 0.98 (0.00\%) \\ \hline
num\_cc & 1.00 & 1.00 (0.00\%) & 1.00 (0.00\%) & 1.00 (0.00\%) & 1.00 (0.00\%) \\ \hline
largest\_cc & 16.00 & \textbf{10.00 (37.50\%)} & 16.00 (0.00\%) & 16.00 (0.00\%) & 16.00 (0.00\%) \\ \hline
largest\_cc\_radius & 1.00 & \textbf{2.00 (100.00\%)} & 1.00 (0.00\%) & 1.00 (0.00\%) & 1.00 (0.00\%) \\ \hline
avg\_path\_len & 1.08 & \textbf{1.74 (60.47\%)} & \textbf{1.45 (34.23\%)} & \textbf{1.14 (5.00\%)} & 1.08 (0.00\%) \\ \hline
largest\_clique\_size & 14.00 & \textbf{4.00 (71.43\%)} & \textbf{10.00 (28.57\%)} & 14.00 (0.00\%) & 14.00 (0.00\%) \\ \hline
reciprocity & 0.93 & \textbf{0.28 (70.25\%)} & \textbf{0.57 (38.59\%)} & \textbf{0.89 (4.14\%)} & 0.93 (0.00\%) \\ \hline
\end{tabular}
}
\end{table}

\begin{table}
\centering
\caption{Instancja grid\_10 - porównanie wartości miar uzyskanych z próbkowania snowball z wartościami z przeglądu zupełnego.}
\label{tab:comp_grid_10_snowball}
\footnotesize
\resizebox{\textwidth}{!}{
\begin{tabular}{|l|l|l|l|l|l|}
\hline
\textbf{metryka} & \textbf{p. zupełny} & \textbf{snowball\_1} & \textbf{snowball\_10} & \textbf{snowball\_100} & \textbf{snowball\_1000} \\ \hline
opt\_count & 362880.00 & 2301.00 & 5301.00 & 19184.00 & 111184.00 \\ \hline
oracle\_count & 60644109.00 & 297585.00 & 679185.00 & 2476485.00 & 14224680.00 \\ \hline
node\_count & 18.00 & 18.00 (0.00\%) & 18.00 (0.00\%) & 18.00 (0.00\%) & 18.00 (0.00\%) \\ \hline
edge\_count & 276.00 & \textbf{143.00 (48.19\%)} & \textbf{206.00 (25.36\%)} & \textbf{252.00 (8.70\%)} & \textbf{275.00 (0.36\%)} \\ \hline
num\_subsinks & 14.00 & \textbf{17.00 (21.43\%)} & 14.00 (0.00\%) & 14.00 (0.00\%) & 14.00 (0.00\%) \\ \hline
edge\_to\_node & 15.33 & \textbf{7.94 (48.19\%)} & \textbf{11.44 (25.36\%)} & \textbf{14.00 (8.70\%)} & \textbf{15.28 (0.36\%)} \\ \hline
avg\_fitness & 1100.44 & 1100.44 (0.00\%) & 1100.44 (0.00\%) & 1100.44 (0.00\%) & 1100.44 (0.00\%) \\ \hline
distLO & 0.13 & \textbf{0.23 (75.72\%)} & \textbf{0.18 (34.40\%)} & \textbf{0.13 (1.30\%)} & \textbf{0.02 (81.95\%)} \\ \hline
conrel & 17.00 & \textbf{3.50 (79.41\%)} & \textbf{8.00 (52.94\%)} & \textbf{5.00 (70.59\%)} & 17.00 (0.00\%) \\ \hline
avg\_loop\_weight & 222.94 & \textbf{97.38 (56.32\%)} & \textbf{163.72 (26.56\%)} & \textbf{587.56 (163.54\%)} & \textbf{3454.78 (1449.61\%)} \\ \hline
go\_path\_ratio & 0.94 & \textbf{0.67 (29.41\%)} & 0.94 (0.00\%) & 0.94 (0.00\%) & 0.94 (0.00\%) \\ \hline
avg\_go\_path\_len & 1.00 & \textbf{1.25 (25.00\%)} & \textbf{1.06 (5.88\%)} & \textbf{1.12 (11.76\%)} & 1.00 (0.00\%) \\ \hline
max\_go\_path\_len & 1.00 & \textbf{2.00 (100.00\%)} & \textbf{2.00 (100.00\%)} & \textbf{2.00 (100.00\%)} & 1.00 (0.00\%) \\ \hline
num\_sinks & 14.00 & \textbf{17.00 (21.43\%)} & 14.00 (0.00\%) & 14.00 (0.00\%) & 14.00 (0.00\%) \\ \hline
num\_sources & 2.00 & \textbf{5.00 (150.00\%)} & 2.00 (0.00\%) & 2.00 (0.00\%) & 2.00 (0.00\%) \\ \hline
funnel\_num & 14.00 & \textbf{17.00 (21.43\%)} & 14.00 (0.00\%) & 14.00 (0.00\%) & 14.00 (0.00\%) \\ \hline
mean\_funnel\_size & 4.79 & \textbf{1.76 (63.13\%)} & \textbf{4.14 (13.43\%)} & 4.79 (0.00\%) & \textbf{4.86 (1.49\%)} \\ \hline
max\_funnel\_size & 5.00 & \textbf{2.00 (60.00\%)} & 5.00 (0.00\%) & 5.00 (0.00\%) & 5.00 (0.00\%) \\ \hline
max\_out\_degree & 17.00 & \textbf{15.00 (11.76\%)} & \textbf{14.00 (17.65\%)} & \textbf{16.00 (5.88\%)} & 17.00 (0.00\%) \\ \hline
avg\_out\_degree & 14.33 & \textbf{7.22 (49.61\%)} & \textbf{10.44 (27.13\%)} & \textbf{13.00 (9.30\%)} & \textbf{14.28 (0.39\%)} \\ \hline
max\_in\_degree & 17.00 & \textbf{12.00 (29.41\%)} & \textbf{16.00 (5.88\%)} & 17.00 (0.00\%) & 17.00 (0.00\%) \\ \hline
avg\_in\_degree & 14.33 & \textbf{7.22 (49.61\%)} & \textbf{10.44 (27.13\%)} & \textbf{13.00 (9.30\%)} & \textbf{14.28 (0.39\%)} \\ \hline
density & 0.90 & \textbf{0.47 (48.19\%)} & \textbf{0.67 (25.36\%)} & \textbf{0.82 (8.70\%)} & \textbf{0.90 (0.36\%)} \\ \hline
num\_cc & 1.00 & 1.00 (0.00\%) & 1.00 (0.00\%) & 1.00 (0.00\%) & 1.00 (0.00\%) \\ \hline
largest\_cc & 18.00 & 18.00 (0.00\%) & 18.00 (0.00\%) & 18.00 (0.00\%) & 18.00 (0.00\%) \\ \hline
largest\_cc\_radius & 1.00 & \textbf{2.00 (100.00\%)} & \textbf{2.00 (100.00\%)} & 1.00 (0.00\%) & 1.00 (0.00\%) \\ \hline
avg\_path\_len & 1.16 & \textbf{1.43 (23.99\%)} & \textbf{1.40 (20.90\%)} & \textbf{1.24 (6.78\%)} & \textbf{1.16 (0.28\%)} \\ \hline
largest\_clique\_size & 15.00 & \textbf{8.00 (46.67\%)} & \textbf{9.00 (40.00\%)} & \textbf{12.00 (20.00\%)} & \textbf{14.00 (6.67\%)} \\ \hline
reciprocity & 0.85 & \textbf{0.62 (27.83\%)} & \textbf{0.76 (11.42\%)} & \textbf{0.81 (4.78\%)} & \textbf{0.87 (2.21\%)} \\ \hline
\end{tabular}
}
\end{table}

\begin{table}
\centering
\caption{Instancja grid\_10 - porównanie wartości miar uzyskanych z próbkowania dwufazowego z wartościami z przeglądu zupełnego.}
\label{tab:comp_grid_10_twophase}
\footnotesize
\resizebox{\textwidth}{!}{
\begin{tabular}{|l|l|l|l|l|l|}
\hline
\textbf{metryka} & \textbf{p. zupełny} & \textbf{dwufazowy\_10} & \textbf{dwufazowy\_100} & \textbf{dwufazowy\_1000} & \textbf{dwufazowy\_10000} \\ \hline
opt\_count & 362880.00 & 115.00 & 10080.00 & 1000167.00 & 100000139.00 \\ \hline
oracle\_count & 60644109.00 & 16288.00 & 2279193.00 & 248253253.00 & 25043586053.00 \\ \hline
node\_count & 18.00 & \textbf{9.00 (50.00\%)} & 18.00 (0.00\%) & 18.00 (0.00\%) & 18.00 (0.00\%) \\ \hline
edge\_count & 276.00 & \textbf{34.00 (87.68\%)} & \textbf{185.00 (32.97\%)} & \textbf{253.00 (8.33\%)} & 276.00 (0.00\%) \\ \hline
num\_subsinks & 14.00 & \textbf{8.00 (42.86\%)} & 14.00 (0.00\%) & 14.00 (0.00\%) & 14.00 (0.00\%) \\ \hline
edge\_to\_node & 15.33 & \textbf{3.78 (75.36\%)} & \textbf{10.28 (32.97\%)} & \textbf{14.06 (8.33\%)} & 15.33 (0.00\%) \\ \hline
avg\_fitness & 1100.44 & \textbf{1086.67 (1.25\%)} & 1100.44 (0.00\%) & 1100.44 (0.00\%) & 1100.44 (0.00\%) \\ \hline
distLO & 0.13 & \textbf{0.75 (469.82\%)} & \textbf{0.59 (350.50\%)} & \textbf{0.12 (7.95\%)} & \textbf{0.03 (79.33\%)} \\ \hline
conrel & 17.00 & \textbf{0.50 (97.06\%)} & \textbf{2.00 (88.24\%)} & \textbf{8.00 (52.94\%)} & 17.00 (0.00\%) \\ \hline
avg\_loop\_weight & 222.94 & \textbf{4.33 (98.06\%)} & \textbf{24.67 (88.94\%)} & \textbf{249.44 (11.89\%)} & \textbf{2470.22 (1008.00\%)} \\ \hline
go\_path\_ratio & 0.94 & \textbf{0.89 (5.88\%)} & 0.94 (0.00\%) & 0.94 (0.00\%) & 0.94 (0.00\%) \\ \hline
avg\_go\_path\_len & 1.00 & \textbf{2.00 (100.00\%)} & \textbf{1.59 (58.82\%)} & \textbf{1.06 (5.88\%)} & 1.00 (0.00\%) \\ \hline
max\_go\_path\_len & 1.00 & \textbf{3.00 (200.00\%)} & \textbf{2.00 (100.00\%)} & \textbf{2.00 (100.00\%)} & 1.00 (0.00\%) \\ \hline
num\_sinks & 14.00 & \textbf{8.00 (42.86\%)} & 14.00 (0.00\%) & 14.00 (0.00\%) & 14.00 (0.00\%) \\ \hline
num\_sources & 2.00 & \textbf{6.00 (200.00\%)} & \textbf{5.00 (150.00\%)} & 2.00 (0.00\%) & 2.00 (0.00\%) \\ \hline
funnel\_num & 14.00 & \textbf{8.00 (42.86\%)} & 14.00 (0.00\%) & 14.00 (0.00\%) & 14.00 (0.00\%) \\ \hline
mean\_funnel\_size & 4.79 & \textbf{1.38 (71.27\%)} & \textbf{3.57 (25.37\%)} & \textbf{4.29 (10.45\%)} & 4.79 (0.00\%) \\ \hline
max\_funnel\_size & 5.00 & \textbf{2.00 (60.00\%)} & 5.00 (0.00\%) & 5.00 (0.00\%) & 5.00 (0.00\%) \\ \hline
max\_out\_degree & 17.00 & \textbf{4.00 (76.47\%)} & \textbf{13.00 (23.53\%)} & 17.00 (0.00\%) & 17.00 (0.00\%) \\ \hline
avg\_out\_degree & 14.33 & \textbf{2.78 (80.62\%)} & \textbf{9.28 (35.27\%)} & \textbf{13.06 (8.91\%)} & 14.33 (0.00\%) \\ \hline
max\_in\_degree & 17.00 & \textbf{8.00 (52.94\%)} & 17.00 (0.00\%) & 17.00 (0.00\%) & 17.00 (0.00\%) \\ \hline
avg\_in\_degree & 14.33 & \textbf{2.78 (80.62\%)} & \textbf{9.28 (35.27\%)} & \textbf{13.06 (8.91\%)} & 14.33 (0.00\%) \\ \hline
density & 0.90 & \textbf{0.47 (47.64\%)} & \textbf{0.60 (32.97\%)} & \textbf{0.83 (8.33\%)} & 0.90 (0.00\%) \\ \hline
num\_cc & 1.00 & 1.00 (0.00\%) & 1.00 (0.00\%) & 1.00 (0.00\%) & 1.00 (0.00\%) \\ \hline
largest\_cc & 18.00 & \textbf{9.00 (50.00\%)} & 18.00 (0.00\%) & 18.00 (0.00\%) & 18.00 (0.00\%) \\ \hline
largest\_cc\_radius & 1.00 & 1.00 (0.00\%) & 1.00 (0.00\%) & 1.00 (0.00\%) & 1.00 (0.00\%) \\ \hline
avg\_path\_len & 1.16 & \textbf{1.75 (50.85\%)} & \textbf{1.61 (39.55\%)} & \textbf{1.24 (7.06\%)} & 1.16 (0.00\%) \\ \hline
largest\_clique\_size & 15.00 & \textbf{4.00 (73.33\%)} & \textbf{10.00 (33.33\%)} & \textbf{13.00 (13.33\%)} & 15.00 (0.00\%) \\ \hline
reciprocity & 0.85 & \textbf{0.32 (62.47\%)} & \textbf{0.50 (41.01\%)} & \textbf{0.77 (9.18\%)} & 0.85 (0.00\%) \\ \hline
\end{tabular}
}
\end{table}

\begin{table}
\centering
\caption{Instancja grid\_11 - porównanie wartości miar uzyskanych z próbkowania snowball z wartościami z przeglądu zupełnego.}
\label{tab:comp_grid_11_snowball}
\footnotesize
\resizebox{\textwidth}{!}{
\begin{tabular}{|l|l|l|l|l|l|}
\hline
\textbf{metryka} & \textbf{p. zupełny} & \textbf{snowball\_1} & \textbf{snowball\_10} & \textbf{snowball\_100} & \textbf{snowball\_1000} \\ \hline
opt\_count & 3628800.00 & 1401.00 & 16501.00 & 40156.00 & 142756.00 \\ \hline
oracle\_count & 852898736.00 & 221870.00 & 2660240.00 & 6491430.00 & 22904695.00 \\ \hline
node\_count & 46.00 & \textbf{34.00 (26.09\%)} & 46.00 (0.00\%) & 46.00 (0.00\%) & 46.00 (0.00\%) \\ \hline
edge\_count & 1467.00 & \textbf{141.00 (90.39\%)} & \textbf{846.00 (42.33\%)} & \textbf{1082.00 (26.24\%)} & \textbf{1230.00 (16.16\%)} \\ \hline
num\_subsinks & 18.00 & \textbf{31.00 (72.22\%)} & \textbf{19.00 (5.56\%)} & 18.00 (0.00\%) & 18.00 (0.00\%) \\ \hline
edge\_to\_node & 31.89 & \textbf{4.15 (87.00\%)} & \textbf{18.39 (42.33\%)} & \textbf{23.52 (26.24\%)} & \textbf{26.74 (16.16\%)} \\ \hline
avg\_fitness & 1194.57 & \textbf{1182.06 (1.05\%)} & 1194.57 (0.00\%) & 1194.57 (0.00\%) & 1194.57 (0.00\%) \\ \hline
distLO & 0.26 & \textbf{0.55 (112.61\%)} & \textbf{0.26 (0.76\%)} & \textbf{0.17 (35.04\%)} & \textbf{0.16 (38.38\%)} \\ \hline
conrel & 8.20 & \textbf{0.42 (94.92\%)} & \textbf{2.07 (74.80\%)} & \textbf{2.29 (72.13\%)} & \textbf{2.54 (69.04\%)} \\ \hline
avg\_loop\_weight & 223.83 & \textbf{69.67 (68.87\%)} & \textbf{183.93 (17.82\%)} & \textbf{427.70 (91.08\%)} & \textbf{1708.43 (663.29\%)} \\ \hline
go\_path\_ratio & 0.98 & \textbf{0.32 (66.93\%)} & \textbf{0.96 (2.22\%)} & 0.98 (0.00\%) & 0.98 (0.00\%) \\ \hline
avg\_go\_path\_len & 1.09 & \textbf{1.45 (33.58\%)} & \textbf{1.36 (25.23\%)} & \textbf{1.29 (18.37\%)} & \textbf{1.27 (16.33\%)} \\ \hline
max\_go\_path\_len & 2.00 & 2.00 (0.00\%) & 2.00 (0.00\%) & 2.00 (0.00\%) & 2.00 (0.00\%) \\ \hline
num\_sinks & 18.00 & \textbf{31.00 (72.22\%)} & \textbf{19.00 (5.56\%)} & 18.00 (0.00\%) & 18.00 (0.00\%) \\ \hline
num\_sources & 5.00 & \textbf{24.00 (380.00\%)} & \textbf{10.00 (100.00\%)} & \textbf{8.00 (60.00\%)} & \textbf{7.00 (40.00\%)} \\ \hline
funnel\_num & 18.00 & \textbf{31.00 (72.22\%)} & \textbf{19.00 (5.56\%)} & 18.00 (0.00\%) & 18.00 (0.00\%) \\ \hline
mean\_funnel\_size & 23.89 & \textbf{1.65 (93.11\%)} & \textbf{15.63 (34.57\%)} & \textbf{19.44 (18.60\%)} & \textbf{20.50 (14.19\%)} \\ \hline
max\_funnel\_size & 29.00 & \textbf{4.00 (86.21\%)} & \textbf{26.00 (10.34\%)} & \textbf{28.00 (3.45\%)} & 29.00 (0.00\%) \\ \hline
max\_out\_degree & 43.00 & \textbf{15.00 (65.12\%)} & \textbf{26.00 (39.53\%)} & \textbf{34.00 (20.93\%)} & \textbf{37.00 (13.95\%)} \\ \hline
avg\_out\_degree & 30.89 & \textbf{3.79 (87.72\%)} & \textbf{17.41 (43.63\%)} & \textbf{22.52 (27.09\%)} & \textbf{25.74 (16.68\%)} \\ \hline
max\_in\_degree & 45.00 & \textbf{10.00 (77.78\%)} & \textbf{37.00 (17.78\%)} & \textbf{41.00 (8.89\%)} & \textbf{44.00 (2.22\%)} \\ \hline
avg\_in\_degree & 30.89 & \textbf{3.79 (87.72\%)} & \textbf{17.41 (43.63\%)} & \textbf{22.52 (27.09\%)} & \textbf{25.74 (16.68\%)} \\ \hline
density & 0.71 & \textbf{0.13 (82.27\%)} & \textbf{0.41 (42.33\%)} & \textbf{0.52 (26.24\%)} & \textbf{0.59 (16.16\%)} \\ \hline
num\_cc & 1.00 & 1.00 (0.00\%) & 1.00 (0.00\%) & 1.00 (0.00\%) & 1.00 (0.00\%) \\ \hline
largest\_cc & 46.00 & \textbf{34.00 (26.09\%)} & 46.00 (0.00\%) & 46.00 (0.00\%) & 46.00 (0.00\%) \\ \hline
largest\_cc\_radius & 1.00 & \textbf{2.00 (100.00\%)} & \textbf{2.00 (100.00\%)} & \textbf{2.00 (100.00\%)} & \textbf{2.00 (100.00\%)} \\ \hline
avg\_path\_len & 1.31 & \textbf{1.87 (42.07\%)} & \textbf{1.65 (25.57\%)} & \textbf{1.51 (14.82\%)} & \textbf{1.43 (8.75\%)} \\ \hline
largest\_clique\_size & 29.00 & \textbf{7.00 (75.86\%)} & \textbf{12.00 (58.62\%)} & \textbf{14.00 (51.72\%)} & \textbf{16.00 (44.83\%)} \\ \hline
reciprocity & 0.70 & \textbf{0.37 (46.70\%)} & \textbf{0.57 (18.45\%)} & \textbf{0.66 (5.98\%)} & \textbf{0.73 (4.53\%)} \\ \hline
\end{tabular}
}
\end{table}

\begin{table}
\centering
\caption{Instancja grid\_11 - porównanie wartości miar uzyskanych z próbkowania dwufazowego z wartościami z przeglądu zupełnego.}
\label{tab:comp_grid_11_twophase}
\footnotesize
\resizebox{\textwidth}{!}{
\begin{tabular}{|l|l|l|l|l|l|}
\hline
\textbf{metryka} & \textbf{p. zupełny} & \textbf{dwufazowy\_10} & \textbf{dwufazowy\_100} & \textbf{dwufazowy\_1000} & \textbf{dwufazowy\_10000} \\ \hline
opt\_count & 3628800.00 & 117.00 & 10686.00 & 1001548.00 & 100003412.00 \\ \hline
oracle\_count & 852898736.00 & 20088.00 & 2734752.00 & 330689157.00 & 33809658289.00 \\ \hline
node\_count & 46.00 & \textbf{10.00 (78.26\%)} & 46.00 (0.00\%) & 46.00 (0.00\%) & 46.00 (0.00\%) \\ \hline
edge\_count & 1467.00 & \textbf{36.00 (97.55\%)} & \textbf{789.00 (46.22\%)} & \textbf{1263.00 (13.91\%)} & \textbf{1463.00 (0.27\%)} \\ \hline
num\_subsinks & 18.00 & \textbf{10.00 (44.44\%)} & 18.00 (0.00\%) & 18.00 (0.00\%) & 18.00 (0.00\%) \\ \hline
edge\_to\_node & 31.89 & \textbf{3.60 (88.71\%)} & \textbf{17.15 (46.22\%)} & \textbf{27.46 (13.91\%)} & \textbf{31.80 (0.27\%)} \\ \hline
avg\_fitness & 1194.57 & \textbf{1141.00 (4.48\%)} & 1194.57 (0.00\%) & 1194.57 (0.00\%) & 1194.57 (0.00\%) \\ \hline
distLO & 0.26 & \textbf{0.48 (85.20\%)} & \textbf{0.26 (0.35\%)} & \textbf{0.29 (10.45\%)} & \textbf{0.00 (98.68\%)} \\ \hline
conrel & 8.20 & \textbf{1.50 (81.71\%)} & \textbf{10.50 (28.05\%)} & \textbf{3.60 (56.10\%)} & \textbf{45.00 (448.78\%)} \\ \hline
avg\_loop\_weight & 223.83 & \textbf{4.00 (98.21\%)} & \textbf{16.33 (92.71\%)} & \textbf{171.80 (23.24\%)} & \textbf{1688.13 (654.22\%)} \\ \hline
go\_path\_ratio & 0.98 & \textbf{0.90 (8.00\%)} & 0.98 (0.00\%) & 0.98 (0.00\%) & 0.98 (0.00\%) \\ \hline
avg\_go\_path\_len & 1.09 & \textbf{1.78 (63.27\%)} & 1.09 (0.00\%) & \textbf{1.40 (28.57\%)} & \textbf{1.00 (8.16\%)} \\ \hline
max\_go\_path\_len & 2.00 & \textbf{3.00 (50.00\%)} & 2.00 (0.00\%) & 2.00 (0.00\%) & \textbf{1.00 (50.00\%)} \\ \hline
num\_sinks & 18.00 & \textbf{10.00 (44.44\%)} & 18.00 (0.00\%) & 18.00 (0.00\%) & 18.00 (0.00\%) \\ \hline
num\_sources & 5.00 & \textbf{10.00 (100.00\%)} & \textbf{17.00 (240.00\%)} & \textbf{6.00 (20.00\%)} & 5.00 (0.00\%) \\ \hline
funnel\_num & 18.00 & \textbf{10.00 (44.44\%)} & 18.00 (0.00\%) & 18.00 (0.00\%) & 18.00 (0.00\%) \\ \hline
mean\_funnel\_size & 23.89 & \textbf{1.00 (95.81\%)} & \textbf{15.94 (33.26\%)} & \textbf{21.94 (8.14\%)} & 23.89 (0.00\%) \\ \hline
max\_funnel\_size & 29.00 & \textbf{1.00 (96.55\%)} & 29.00 (0.00\%) & 29.00 (0.00\%) & 29.00 (0.00\%) \\ \hline
max\_out\_degree & 43.00 & \textbf{5.00 (88.37\%)} & \textbf{25.00 (41.86\%)} & \textbf{36.00 (16.28\%)} & 43.00 (0.00\%) \\ \hline
avg\_out\_degree & 30.89 & \textbf{2.60 (91.58\%)} & \textbf{16.15 (47.71\%)} & \textbf{26.46 (14.36\%)} & \textbf{30.80 (0.28\%)} \\ \hline
max\_in\_degree & 45.00 & \textbf{5.00 (88.89\%)} & \textbf{44.00 (2.22\%)} & 45.00 (0.00\%) & 45.00 (0.00\%) \\ \hline
avg\_in\_degree & 30.89 & \textbf{2.60 (91.58\%)} & \textbf{16.15 (47.71\%)} & \textbf{26.46 (14.36\%)} & \textbf{30.80 (0.28\%)} \\ \hline
density & 0.71 & \textbf{0.40 (43.56\%)} & \textbf{0.38 (46.22\%)} & \textbf{0.61 (13.91\%)} & \textbf{0.71 (0.27\%)} \\ \hline
num\_cc & 1.00 & 1.00 (0.00\%) & 1.00 (0.00\%) & 1.00 (0.00\%) & 1.00 (0.00\%) \\ \hline
largest\_cc & 46.00 & \textbf{10.00 (78.26\%)} & 46.00 (0.00\%) & 46.00 (0.00\%) & 46.00 (0.00\%) \\ \hline
largest\_cc\_radius & 1.00 & \textbf{2.00 (100.00\%)} & 1.00 (0.00\%) & 1.00 (0.00\%) & 1.00 (0.00\%) \\ \hline
avg\_path\_len & 1.31 & \textbf{2.22 (69.18\%)} & \textbf{1.90 (44.35\%)} & \textbf{1.41 (7.50\%)} & \textbf{1.32 (0.15\%)} \\ \hline
largest\_clique\_size & 29.00 & \textbf{5.00 (82.76\%)} & \textbf{13.00 (55.17\%)} & \textbf{23.00 (20.69\%)} & 29.00 (0.00\%) \\ \hline
reciprocity & 0.70 & \textbf{0.23 (66.94\%)} & \textbf{0.37 (46.79\%)} & \textbf{0.61 (13.13\%)} & \textbf{0.70 (0.12\%)} \\ \hline
\end{tabular}
}
\end{table}



TODO: Komentarz do wyników.

\subsection{Badanie stabilności dla średnich instancji}
Instancje \textbf{burma14}, \textbf{ulysses22} ze zbioru tsplib oraz wygenerowane instancje
\textbf{cliques\_20}, \textbf{uniform\_20} oraz \textbf{grid\_20} są zbyt duże, by wykonać dla nich przegląd zupełny,
ale zbyt małe, żeby próbkować je z takimi samymi parametrami, jak większe instancje.
Z tego powodu badania przeprowadzone dla tych instancji umieszczono w osobnej sekcji.
Dla instancji tego typu wykonano bardziej szczegółowe próbkowanie z mniejszymi interwałami zapisu.

Próbkowanie algorytmem snowball wykonano z następującymi parametrami:
\begin{itemize}
    \item $w_{len}$ - długość losowego spaceru - 10000
    \item $m$ - liczba prób przeszukania sąsiedztwa - 100
    \item $depth$ - głębokość przeszukiwania - 3
\end{itemize}
Zapis wyników wykonywano przy każdym znalezionym wierzchołku.
Wyjątkiem była instancja grid\_20 - tu zapis wykonywano co 100 wierzchołków z powodu dużej ilości optimów lokalnych.

Próbkowanie algorytmem dwufazowym przeprowadzono z następującymi parametrami:
\begin{itemize}
    \item $n_{max}$ - żądana liczba wierzchołków - 1
    \item $n_{att}$ - liczba prób generowania wierzchołków - 1000
    \item $e_{att}$ - liczba prób generowania krawędzi - 10
    \item $n_{runs}$ - liczba powtórzeń - 1000
\end{itemize}
Zapis wyników wykonywano po zakończeniu każdego powtórzenia.

Utworzono wykresy przedstawiające zależność wartości miar od liczby spróbkowanych wierzchołków.
Przedstawiono je na rysunkach od \ref{fig:small_edge_count} do \ref{fig:small_missed}.

\def\metricsSmall{
    {edge_count/Liczba krawędzi},
    {edge_to_node/Stosunek liczby krawędzi do liczby wierzchołków},
    {assortativity_deg/Współczynnik różnorodności grafu},
    {avg_fitness/Średnia wartość funkcji celu w znalezionych optimach lokalnych},
    {conrel/Współczynnik conrel},
    {density/Gęstość grafu},
    {distLO/Współczynnik distLO},
    {reciprocity/Współczynnik wzajemności grafu},
    {largest_clique_size/Rozmiar największej kliki w grafie},
    {num_sources/Liczba źródeł w grafie},
    {num_sinks/Liczba ścieków w grafie},
    {num_subsinks/Liczba subsinks w grafie},
    {avg_in_degree/Średni stopień wchodzący wierzchołków w grafie},
    {max_in_degree/Maksymalny stopień wchodzący wśród wierzchołków w grafie},
    {max_out_degree/Średni stopień wychodzący wierzchołków w grafie},
    {avg_out_degree/Maksymalny stopień wychodzący wśród wierzchołków w grafie},
    {avg_loop_weight/Średnia waga pętli w grafie},
    {avg_path_len/Średnia waga pętli w grafie},
    {avg_go_path_len/Średnia długość istniejących ścieżek do globalnego optimum},
    {max_go_path_len/Długość najdłuższej istniejącej ścieżki do globalnego optimum},
    {go_path_ratio/Stosunek liczby wierzchołków, z których istnieje ścieżka do globalnego optimum do liczby wszystkich wierzchołków},
    {funnel_num/Liczba lejów w przestrzeni},
    {max_funnel_size/Rozmiar największego leja w przestrzeni},
    {mean_funnel_size/Średni rozmiar lejów w przestrzeni},
    {num_cc/Liczba spójnych podgrafów},
    {largest_cc/Rozmiar największego spójnego podgrafu},
    {largest_cc_radius/Promień największego spójnego podgrafu}}


\foreach \metric/\cap in \metricsSmall{
    \begin{figure}[p]
        \centering
        \includegraphics[width=\textwidth]{chapters/experiments/img/merged_plots/per1_all/\metric.png}
        \caption{\cap \space w zależności od liczby wierzchołków}
        \label{fig:small_\metric}
    \end{figure}
}

\begin{figure}[h!]
    \centering
    \includegraphics[width=\textwidth]{chapters/experiments/img/merged_plots/per1_all/missed.png}
    \caption{Liczba nieudanych prób tworzenia krawędzi w zależności od liczby wierzchołków - próbkowanie dwufazowe}
    \label{fig:small_missed}
\end{figure}

TODO: Komentarz do wyników

\newpage

\subsection{Badanie stabilności dla dużych instancji}
W tej sekcji opisano badania przeprowadzone dla instancji o rozmiarze 48 i większym.
Próbkowanie przeprowadzono przez ustaloną z góry liczbę powtórzeń, zapisując stan przestrzeni po zakończeniu każdego powtórzenia.

Próbkowanie algorytmem snowball wykonano z następującymi parametrami:
\begin{itemize}
    \item $w_{len}$ - długość losowego spaceru - 10000
    \item $m$ - liczba prób przeszukania sąsiedztwa - 100
    \item $depth$ - głębokość przeszukiwania - 3
\end{itemize}

Próbkowanie algorytmem \textit{snowball} prowadzono do zakończenia losowego spaceru lub osiągnięcia liczby 100000 wierzchołków.
Dla instancji \textbf{cliques\_50} z powodu małej liczby optimów lokalnych zapis wykonywano co 100 znalezionych wierzchołków.
Dla pozostałych instancji stan próbkowania zapisywany był co 1000 znalezionych wierzchołków.

Próbkowanie algorytmem dwufazowym przeprowadzono z następującymi parametrami:
\begin{itemize}
    \item $n_{max}$ - żądana liczba wierzchołków - 1000
    \item $n_{att}$ - liczba prób generowania wierzchołków - 1000
    \item $e_{att}$ - liczba prób generowania krawędzi - 1000
    \item $n_{runs}$ - liczba powtórzeń - 100
\end{itemize}

Po zakończeniu próbkowania dla każdego zapisanego stanu przestrzeni obliczono wartości badanych miar. W ten sposób uzyskano
wartości miar dla różnych etapów próbkowania przestrzeni.

Utworzono wykresy przedstawiające zależność wartości miar od liczby spróbkowanych wierzchołków.
Przedstawiono je na rysunkach od \ref{fig:main_snowball_edge_count} do \ref{fig:main_twophase_missed}.

\def\metrics{
    {edge_count/Liczba krawędzi},
    {edge_to_node/Stosunek liczby krawędzi do liczby wierzchołków},
    {assortativity_deg/Współczynnik różnorodności grafu},
    {avg_fitness/Średnia wartość funkcji celu w znalezionych optimach lokalnych},
    {conrel/Współczynnik conrel},
    {density/Gęstość grafu},
    {distLO/Współczynnik distLO},
    {reciprocity/Współczynnik wzajemności grafu},
    {largest_clique_size/Rozmiar największej kliki w grafie},
    {num_sources/Liczba źródeł w grafie},
    {num_sinks/Liczba ścieków w grafie},
    {num_subsinks/Liczba subsinks w grafie},
    {avg_in_degree/Średni stopień wchodzący wierzchołków w grafie},
    {max_in_degree/Maksymalny stopień wchodzący wśród wierzchołków w grafie},
    {max_out_degree/Średni stopień wychodzący wierzchołków w grafie},
    {avg_out_degree/Maksymalny stopień wychodzący wśród wierzchołków w grafie},
    {avg_loop_weight/Średnia waga pętli w grafie},
    {avg_path_len/Średnia waga pętli w grafie},
    {avg_go_path_len/Średnia długość istniejących ścieżek do globalnego optimum},
    {max_go_path_len/Długość najdłuższej istniejącej ścieżki do globalnego optimum},
    {go_path_ratio/Stosunek liczby wierzchołków, z których istnieje ścieżka do globalnego optimum do liczby wszystkich wierzchołków},
    {funnel_num/Liczba lejów w przestrzeni},
    {max_funnel_size/Rozmiar największego leja w przestrzeni},
    {mean_funnel_size/Średni rozmiar lejów w przestrzeni},
    {num_cc/Liczba spójnych podgrafów},
    {largest_cc/Rozmiar największego spójnego podgrafu},
    {largest_cc_radius/Promień największego spójnego podgrafu}}

\foreach \metric/\cap in \metrics{
    \begin{figure}[p]
        \centering
        \includegraphics[width=\textwidth]{chapters/experiments/img/merged_plots/main_snowball/\metric.png}
        \caption{\cap \space w zależności od liczby wierzchołków - próbkowanie snowball}
        \label{fig:main_snowball_\metric}
    \end{figure}
}

TODO: wstawić główne wyniki z twophase, jak już będą, Komentarz do wyników