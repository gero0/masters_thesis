\chapter{Przegląd literatury}

W tym rozdziale przedstawiony jest przegląd najnowszych publikacji z dziedziny badania przestrzeni rozwiązań.
Pierwsza sekcja opisuje praktyczne zastosowania analizy przestrzeni rozwiązań znalezione w literaturze.
Druga sekcja skupia się na przedstawionych w cytowanych pracach algorytmach próbkowania.

\section{Cele badania przestrzeni rozwiązań}
Analiza przestrzeni rozwiązań pozwala na lepsze poznanie niektórych cech problemu, a także zbadanie, w jakim stopniu cechy te zależne są od
rodzaju badanej instancji. Przedmiotem badań z tej dziedziny najczęściej są znane problemy NP-trudne, takie jak problem komiwojażera czy problem kwadratowego przydziału.
Jednym z proponowanych zastosowań analizy przestrzeni rozwiązań jest wykorzystanie jej do wyboru najlepszego algorytmu heurystycznego
dla danej instancji problemu. W pracy \textit{Local Optima Networks in Solving Algorithm Selection Problem for TSP} \cite{DBLP:conf/depcos/BozejkoGNAB18}
sieci optimów lokalnych wygenerowane na podstawie próbkowania przestrzeni rozwiązań wykorzystano do nauczenia modeli regresji, których zadaniem było
przewidzenie, który algorytm heurystyczny da lepsze wyniki dla danej instancji problemu.
Badanie wykazało, że analiza przestrzeni rozwiązań może zostać z powodzeniem wykorzystana w tym celu.
Problemem pozostaje natomiast długi czas trwania próbkowania przestrzeni, zwykle dłuższy niż czas działania samego algorytmu optymalizacji.
W pracy \textit{Mapping the global structure of TSP Fitness landscapes} \cite{DBLP:journals/heuristics/OchoaV18} zbadano przestrzeń rozwiązań
dla różnych instancji problemu komiwojażera. Zauważono, że instancje wygenerowane losowo zwykle mają mniejszą neutralność i mniej globalnych optimów od
instancji rzeczywistych. Zaobserwowano również, że sposób rozłożenia miast (w klastrach, równomierny) wpływa na wzajemną korelację wartości różnych miar.

W ostatnich latach w podobny sposób przeprowadzono analizy przestrzeni konfiguracji wieloparametrowych algorytmów optymalizacyjnych.
W pracy \textit{Understanding Parameter Spaces using Local Optima Networks: A Case Study on Particle Swarm Optimization} \cite{DBLP:conf/gecco/CleghornO21}
wykorzystano sieci lokalnych optimów, oraz pochodne struktury CMLON do analizy i wizualizacji przestrzeni parametrów algorytmu roju cząstek.
Analiza wykazała istnienie dużej ilości lokalnych optimów, niską neutralność, oraz istnienie wielu ścieków (ang sinks) nie znajdujących się w optimum globalnym,
Sugeruje to, że naiwne metody dobierania parametrów mogą łatwo doprowadzić do suboptymalnej konfiguracji i w efekcie nie otrzymania najlepszych wyników.
Podobne badanie wykonano dla przestrzeni parametrów procesu AutoML w pracy \textit{Understanding AutoML Search Spaces with Local Optima Networks} \cite{DBLP:conf/gecco/TeixeiraP22}.
AutoML jest procesem automatyzacji konfiguracji procesów (ang. pipelines) uczenia maszynowego obejmującym m. in. wybór zastosowanego przetwarzania wstępnego,
właściwego algorytmu uczenia, oraz jego hiperparametrów. W pracy zbadano przestrzeń konfiguracji AutoML dla zadania klasyfikacji.
Jako funkcję celu przyjęto uśrednioną wartość metryki F-score dla danej konfiguracji procesu.
Wartości F-score uzyskano poprzez przetestowanie procesu na kilku zbiorach danych.
Dla każdej z badanych przestrzeni konfiguracji utworzono trzy sieci LON. Każda z sieci była oparta o inny model krawędzi.
Przeszukano wszystkie możliwe rozwiązania, ale ze względu na złożoność obliczeniową sieć LON budowano poprzez przeglądanie
ograniczonego sąsiedztwa każdego z wierzchołków. Badanie wykonano dla kilku wielkości sąsiedztwa --- 20, 30, 50 i 100 sąsiadów.
Zauważono wpływ modelu krawędzi na powstałą sieć --- w sieciach z krawędziami typu basin-transition nie zauważono obecności ścieków,
natomiast istniało wiele źródeł. Sieci oparte o escape edges miały z kolei dużo ścieków i mało źródeł.
W przypadku perturbation edges w sieci nie było ścieków ani źródeł niezależnie od rozmiaru sąsiedztwa.
Zauważono, że w badanych przypadkach optima globalne skupiały się w pewnym rejonie przestrzeni rozwiązań, w niewielkiej od siebie odległości.
Nie były rozłożone równomiernie. Sugeruje to, że w problemie konfiguracji procesu uczenia maszynowego, najlepsze ze wszystkich możliwych
konfiguracji niewiele się od siebie różnią.


\section{Próbkowanie przestrzeni rozwiązań}
Przegląd zupełny przestrzeni rozwiązań problemów trudnych obliczeniowo jest w praktyce niemożliwy, poza bardzo małymi instancjami problemu.
Z tego powodu analizę tą przeprowadza się na części przestrzeni zbadanej w procesie próbkowania. \\
W tym miejscu pojawiają się pytanie: w jaki sposób próbkować przestrzeń, aby cechy jej zbadanego fragmentu
jak najlepiej odzwierciedlały cechy całej przestrzeni?

Próbkowanie przestrzeni rozwiązań wykonuje się zwykle poprzez zastosowanie pewnej odmiany iteracyjnego przeszukiwania lokalnego
(ang. ILS --- Iterated Local Search) \cite{DBLP:journals/corr/OchoaVDT14},
a wyniki zapisuje się w postaci sieci optimów lokalnych.
Najczęściej stosowane algorytmy próbkowania oparte o ILS można podzielić na trzy kategorie:
Próbkowanie dwufazowe, \textit{Markov-chain} oraz \textit{Snowball}.

Próbkowanie dwufazowe składa się z dwóch faz --- najpierw próbkowane są wierzchołki poprzez  generowanie
losowego rozwiązania, a następnie wykorzystaniu algorytmu optymalizacji lokalnej do znalezienia najbliższego
lokalnego optimum. Następuje po tym wylosowanie nowego punktu i powtórzenie procedury.
Druga faza to faza próbkowania krawędzi, polegająca na poddaniu znalezionych wcześniej rozwiązań operacji perturbacji, a następnie wykonaniu optymalizacji
lokalnej. Jeśli otrzymane w ten sposób rozwiązanie jest innym lokalnym optimum znajdującym się w zbiorze wierzchołków, to dodawana jest odpowiednia krawędź.
Metoda ta po raz pierwszy została zaprezentowana w pracy \textit{Data-Driven Local Optima Network Characterization of QAPLIB Instances} \cite{10.1145/2576768.2598275}
do analizy problemu kwadratowego przypisania.
W pracy \textit{Local Optima Networks in Solving Algorithm Selection Problem for TSP} \cite{DBLP:conf/depcos/BozejkoGNAB18}
algorytm oparty o tę samą metodę został wykorzystany do badania przestrzeni problemu komiwojażera.
Zastosowana tam odmiana wykorzystuje algorytm 2-opt do optymalizacji lokalnej i procedurę \textit{2-exchange} jako operację perturbacji.

Metoda \textit{Markov-chain} została zaprezentowana w pracy \textit{Perturbation Strength and the Global Structure of QAP Fitness Landscapes} \cite{markovchain}.
Zaczyna się od wybrania losowego rozwiązania i jego optymalizacji. Następnie otrzymane optimum lokalne zostaje poddane operacji perturbacji
i otrzymywane w ten sposób rozwiązanie staje się nowym punktem startowym. Procedura jest powtarzana przez określoną liczbę iteracji, z każdą iteracją
zapisywane są informacje o nowych krawędziach i wierzchołkach.
Metoda została wykorzystana w pracy \textit{Mapping the global structure of TSP fitness landscapes} \cite{DBLP:journals/heuristics/OchoaV18} do badania przestrzeni problemu komiwojażera.
Do optymalizacji lokalnej wykorzystany został algorytm Lin-Kerninghan, a jako operacja perturbacji procedura \textit{Double-bridge}.

Metoda \textit{Snowball} składa się z dwóch etapów wykonywanych naprzemiennie. Pierwszy polega na kilkukrotnym poddaniu rozwiązania początkowego perturbacji
w celu uzyskania sąsiednich rozwiązań, a następnie ich optymalizacji. Procedura jest rekurencyjnie powtarzana dla znalezionych w ten sposób lokalnych optimów
aż do osiągnięcia pewnej z góry ustalonej głębokości. Jedno z lokalnych optimów sąsiadujących z rozwiązaniem początkowym zostaje wybrane jako punkt startowy kolejnej iteracji.
Dodatkowo w pamięci przechowywana jest lista rozwiązań początkowych --- jeśli rozwiązanie już się w nim znajduje, to nie może być wybrane ponownie.
Jeśli nie istnieje rozwiązanie sąsiednie spełniające ten warunek, za kolejne rozwiązanie początkowe przyjmowane jest lokalne optimum otrzymane z optymalizacji rozwiązania losowego.
Algorytm \textit{Snowball} wywodzi się z technik wykorzystywanych w badaniach z dziedziny socjologii, a w kontekście badania przestrzeni rozwiązań
został zaprezentowany po raz pierwszy w pracy \textit{Sampling Local Optima Networks of Large Combinatorial Search Spaces: The QAP Case}
\cite{DBLP:conf/ppsn/VerelDOT18}.

W pracy \textit{Clarifying the Difference in Local Optima Network Sampling Algorithms} \cite{DBLP:conf/evoW/ThomsonOV19}
zostało przeprowadzone statystyczne porównanie algorytmów próbkowania \textit{Snowball}
oraz \textit{Markov-chain}. Eksperyment polegał na spróbkowaniu 30 instancji problemu kwadratowego przypisania, a następnie użyciu zebranych danych
do stworzenia modeli regresji przewidujących jakość rozwiązań, które zostaną uzyskane przez algorytm optymalizacji uruchomiony na instancji.
Wybranymi algorytmami heurystycznymi były \textit{Robust Taboo Search} Taillarda oraz \textit{Improved ILS} Stützla.
Z zebranych danych utworzono grafy LON i zbadano takie właściwości, jak średnia wartość funkcji celu, średni stopień wychodzący
wierzchołków w grafie, promień grafu, liczba lokalnych optimów i liczba krawędzi.
Stworzono modele regresji typu liniowego oraz lasu losowego. Obliczono wartości korelacji pomiędzy właściwościami przestrzeni rozwiązań
a jakością rozwiązań zwracanych przez algorytmy optymalizacji.
Badania wykazały, że dane pozyskane z próbkowania \textit{Markov-chain} były w większym stopniu skorelowane z jakością rozwiązań algorytmów
optymalizacji, a modele regresji utworzone na ich podstawie dokonywały lepszej predykcji niż te utworzone na podstawie danych ze \textit{Snowball}.
Z kolei \textit{Snowball} okazał się bardziej przewidywalny i łatwiejszy w doborze parametrów.

Podobne badanie przeprowadzono w pracy \textit{Inferring Future Landscapes: Sampling the Local Optima Level} \cite{DBLP:journals/ec/ThomsonOVV20}, tym razem na większej liczbie instancji
--- 124 instancji ze zbioru QAPLIB i dodatkowych 60 instancji o rozmiarze N=11.
Dla tych dodatkowych instancji, oprócz próbkowania wykonano również przegląd zupełny.
Zauważono, że algorytm \textit{Snowball} produkuje gęstszą sieć od algorytmu \textit{Markov-chain}.
Dostrzeżono wadę algorytmu \textit{Snowball} --- duży wpływ parametrów próbkowania na wartości metryk opartych o gęstość
i wzory połączeń krawędzi, oraz metryk opisujących struktury lejowe.
Stworzono modele regresji przewidujące jakość rozwiązań generowanych przez heurystyki \textit{Taboo Search} i ILS.
Model przewidujący odpowiedź ILS oparty o dane z algorytmu Snowball był nieco lepszy od modelu opartego o dane z algorytmu ILS.
Między predykcjami modeli przewidujących odpowiedź algorytmu TS nie było znaczącej różnicy.
Wśród wszystkich czterech testowanych modeli cechą przestrzeni będącą najlepszym predyktorem okazała się średnia wartość
funkcji dopasowania (ang. mean fitness).
Porównanie spróbkowanych sieci LON małych instancji z sieciami uzyskanymi
poprzez przegląd zupełny wykazało, że zarówno \textit{Markov-chain} jak i \textit{Snowball}
dobrze aproksymują liczbę wierzchołków i krawędzi w sieci.
\textit{Snowball} odnalazł prawie wszystkie optima lokalne obecne w pełnej sieci.
\textit{Markov-chain} znalazł ich mniej, ale nadal dawał zadowalający wynik.

Z innych prac z dziedziny warto wymienić \textit{First-improvement vs. Best-improvement Local Optima Networks of NK Landscapes} \cite{DBLP:journals/corr/abs-1207-4455}, w którym porównano algorytmy wstępujące (hillclimb)
typu \textit{best-improvement} i \textit{first-improvement} w próbkowaniu przestrzeni NK.
Wykorzystanie algorytmu typu \textit{first-improvement} prowadziło do powstania gęstszej, bardziej kompletnej sieci optimów lokalnych.
Pętle obecne w sieci miały mniejszą wagę w przypadku algorytmu \textit{first-improvement} co sugeruje, że łatwiej wychodzi z optimum lokalnego.
Dodatkową zaletą tego algorytmu jest krótszy czas wykonywania, spowodowany brakiem konieczności przeglądania za każdym razem całego sąsiedztwa
rozwiązania początkowego.

Z kolei w pracy \textit{How Perturbation Strength Shapes the Global Structure of TSP Fitness Landscapes} \cite{DBLP:conf/evoW/McMenemyVO18} zbadano wpływ siły perturbacji (stopnia, w jakim operacja perturbacji zmienia rozwiązanie początkowe)
na właściwości spróbkowanej przestrzeni. Wybranym algorytmem próbkującym był algorytm typu \textit{Markov-chain} pochodzący z artykułu \cite{DBLP:journals/heuristics/OchoaV18}.
Operacją perturbacji tego algorytmu jest procedura double-bridge. Badania przeprowadzono na 180 instancjach o znanych rozwiązaniach optymalnych,
uzyskanych przy pomocy generatora DIMACS TSP . Celem było znalezienie takiego parametru k - siły perturbacji - aby uzyskać jak największy wskaźnik sukcesu.
Za wskaźnik sukcesu przyjęto stosunek liczby przebiegów algorytmu, które znalazły globalne optimum, do wszystkich przebiegów.
Nie znaleziono uniwersalnej najlepszej wartości parametru k. Zauważono jednak, że niższa siła perturbacji sprawdzała się dla instancji,
w których miasta rozmieszczone są w klastrach. Dla instancji z miastami rozmieszczonymi równomiernie zwiększanie wartości k powodowało
zmniejszenie ilości suboptymalnych lejów w przestrzeni.
Zaobserwowano również, że instancje z miastami ułożonymi w klastrach były generalnie prostsze do rozwiązania, niż instancje z rozkładem równomiernym,
a także, że rozkład optimów lokalnych w przestrzeni rozwiązań ma większy wpływ na trudność znalezienia optymalnego rozwiązania dla
danej instancji, niż ich ilość.
